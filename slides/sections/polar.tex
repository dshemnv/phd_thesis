%!TEX root = ../slides.tex

\section[Polar Decoders]{Software Polar Decoders}

\subsection[Polar Codes]{Polar Codes}

\begin{frame}{Polar Codes}
  \begin{figure}[!h]
  \centering
  \begin{tikzpicture}[scale=0.65, every node/.style={transform shape}]
    % \path[use as bounding box] (-2.2, -1.2) rectangle (11.8, 5.2);
    % \draw (-2.2, -1.2) rectangle (11.8, 5.2);

    \tikzset{ txt/.style     ={draw=Paired-1, rounded corners=0pt, minimum height=1cm, minimum width=2.5cm, text=Paired-1, align=center} }
    \tikzset{ rxt/.style     ={draw=Paired-3, rounded corners=0pt, minimum height=1cm, minimum width=2.5cm, text=Paired-3, align=center} }
    \tikzset{ txtplain/.style={draw=Paired-1, fill=Paired-1!50, rounded corners=0pt, minimum height=1cm, minimum width=2.5cm, text=white, align=center} }
    \tikzset{ rxtplain/.style={draw=Paired-3, fill=Paired-3!50, rounded corners=0pt, minimum height=1cm, minimum width=2.5cm, text=white, align=center} }
    \tikzset{ rftx/.style={draw=Paired-1, rounded corners=0pt, minimum height=0.7cm, minimum width=1.0cm, text=black, align=center, rounded corners=2pt, dashed, fill=Paired-1!20} }
    \tikzset{ rfrx/.style={draw=Paired-3, rounded corners=0pt, minimum height=0.7cm, minimum width=1.0cm, text=black, align=center, rounded corners=2pt, dashed, fill=Paired-3!20} }

    \node[style=txtplain, thick   ] (enc)  at ( 4,   4) {Polar\\Encoder};
    \node[style=txt               ] (mod)  at ( 8,   4) {BPSK\\Modulator};
    \node[style=rftx              ] (rftx) at (10.5, 4) {\textbf{RF}};
    \Cloud{(11,2)}{}{0.65};
    \node[text=black, align=center] (chn)  at (10.7, 2) {AWGN\\Channel};
    \node[style=rfrx              ] (rfrx) at (10.5, 0) {\textbf{RF}};
    \node[style=rxt               ] (dem)  at ( 8,   0) {BPSK\\Demodulator};
    \node[style=rxtplain, thick   ] (dec)  at ( 4,   0) {Polar\\Decoder};

    \node[draw=Paired-1, rounded corners=2pt, label={[Paired-1]below:Digital transmitter (BB)}, minimum height=1.5cm, dashed, fit=(enc) (mod)] (tx) {};
    \node[draw=Paired-3, rounded corners=2pt, label={[Paired-3]above:Digital receiver (BB)   }, minimum height=1.5cm, dashed, fit=(dem) (dec)] (rx) {};

    \draw[<-,>=latex] (enc.west) --++ (-1.0,0) node [midway,  above           ] {$\bm{u}$};
    \draw[->,>=latex] (enc)      --   (mod)    node [midway,  above           ] {$\bm{c}$};
    \draw[->,>=latex] (tx.east)  --   (rftx)   node [midway,  above           ] {$\bm{x}$};
    \draw[-         ] (rftx)     --   (11.5,4) node [antenna, above, scale=0.5] {};
    \draw[-         ] (rfrx)     --   (11.5,0) node [antenna, above, scale=0.5] {};
    \draw[<-,>=latex] (rx.east)  --   (rfrx)   node [midway,  above           ] {$\bm{y}$};
    \draw[->,>=latex] (dem)      --   (dec)    node [midway,  above           ] {$\bm{l}$};
    \draw[->,>=latex] (dec.west) --++ (-1.0,0) node [midway,  above           ] {$\bm{\hat{u}}$};
    \draw[- ,>=latex] (mod.east) -- ( 9.5,4);
    \draw[- ,>=latex] (dem.east) -- ( 9.6,0);
  \end{tikzpicture}
  \end{figure}

  \begin{itemize}
    \item<2-> \textbf{Polar codes} have been discovered by \Arikan in 2009~\cite{Arikan2009}
    \item<3-> \textbf{Adopted in the 5G standard} for control channels
    % \item<3-> First code proven to achieve the channel capacity (asymptotically)
  \end{itemize}
  \only<2->{
  \enumcite{Arikan2009}
  }
\end{frame}

\begin{frame}{Coding Scheme}
  \begin{figure}[!h]
  \centering
  \begin{tikzpicture}[scale=0.65, every node/.style={transform shape}]
    \tikzset{ txtplain/.style={draw=Paired-1, fill=Paired-1!50, rounded corners=0pt, minimum height=1cm, minimum width=2.5cm, text=white, align=center} }
    \tikzset{ zoom/.style={draw=Paired-1, dash dot} }

    \node[style=txtplain] (enc) at (0,4.0) {Polar\\Encoder};,

    \draw[<-,>=latex] (enc) --++ (-2.5,0) node [midway, above] {$\bm{u}$};
    \draw[->,>=latex] (enc) --++ ( 2.5,0) node [midway, above] {$\bm{c}$};

    \draw[zoom,-,>=latex] (enc.south west) --++ (-4.25, -1.0);
    \draw[zoom,-,>=latex] (enc.south east) --++ (+4.25, -1.0);

    \only<1>{
    \draw[zoom, fill=Paired-1!20] (-5.5,2.5) rectangle (5.5,-2.5);
    }
    \only<2->{
    \draw[zoom,                 ] (-5.5,2.5) rectangle (5.5,-2.5);
    }

    % \only<2>{
    % \node (g) at (0,0)
    % {$
    % \begin{bmatrix}
    % \\
    % \\
    % \\
    % u_0\\
    % \\
    % u_1\\
    % u_2\\
    % u_3\\
    % \end{bmatrix}^{T}
    % \times
    % \begin{bmatrix}
    % 1 & 0 & 0 & 0 & 0 & 0 & 0 & 0\\
    % 1 & 1 & 0 & 0 & 0 & 0 & 0 & 0\\
    % 1 & 0 & 1 & 0 & 0 & 0 & 0 & 0\\
    % 1 & 1 & 1 & 1 & 0 & 0 & 0 & 0\\
    % 1 & 0 & 0 & 0 & 1 & 0 & 0 & 0\\
    % 1 & 1 & 0 & 0 & 1 & 1 & 0 & 0\\
    % 1 & 0 & 1 & 0 & 1 & 0 & 1 & 0\\
    % 1 & 1 & 1 & 1 & 1 & 1 & 1 & 1\\
    % \end{bmatrix}
    % =
    % \begin{bmatrix}
    % c_0\\
    % c_1\\
    % c_2\\
    % c_3\\
    % c_4\\
    % c_5\\
    % c_6\\
    % c_7\\
    % \end{bmatrix}^{T}
    % $};
    % }

    % \only<3>{
    % \node (g) at (0,0)
    % {$
    % \begin{bmatrix}
    % \textcolor{Paired-1}{0}\\
    % \textcolor{Paired-1}{0}\\
    % \textcolor{Paired-1}{0}\\
    % u_0\\
    % \textcolor{Paired-1}{0}\\
    % u_1\\
    % u_2\\
    % u_3\\
    % \end{bmatrix}^{T}
    % \times
    % \begin{bmatrix}
    % 1 & 0 & 0 & 0 & 0 & 0 & 0 & 0\\
    % 1 & 1 & 0 & 0 & 0 & 0 & 0 & 0\\
    % 1 & 0 & 1 & 0 & 0 & 0 & 0 & 0\\
    % 1 & 1 & 1 & 1 & 0 & 0 & 0 & 0\\
    % 1 & 0 & 0 & 0 & 1 & 0 & 0 & 0\\
    % 1 & 1 & 0 & 0 & 1 & 1 & 0 & 0\\
    % 1 & 0 & 1 & 0 & 1 & 0 & 1 & 0\\
    % 1 & 1 & 1 & 1 & 1 & 1 & 1 & 1\\
    % \end{bmatrix}
    % =
    % \begin{bmatrix}
    % c_0\\
    % c_1\\
    % c_2\\
    % c_3\\
    % c_4\\
    % c_5\\
    % c_6\\
    % c_7\\
    % \end{bmatrix}^{T}
    % $};
    % }

    \only<2->{

    \tikzset{XOR/.style={draw,circle, minimum height=0.35cm,append after command={
            [shorten >=\pgflinewidth, shorten <=\pgflinewidth,]
            (\tikzlastnode.north) edge (\tikzlastnode.south)
            (\tikzlastnode.east) edge (\tikzlastnode.west)
            }
        }
    }

    \tikzset{DOT/.style={circle,fill,inner sep=1.2pt} }

    \newcommand\startx{-3.15}
    \newcommand\startyg{0}
    \newcommand\sed{0.50}
    \newcommand\stasep{1.00}
    \newcommand\xorsep{0.35}

    \node[XOR] (g3_x0) at (\startx,                 \startyg+1.75)       {};
    \node      (g3_e0) at (\startx,                 \startyg+1.25) [DOT] {};

    \node[XOR] (g3_x1) at (\startx,                 \startyg+0.75)       {};
    \node      (g3_e1) at (\startx,                 \startyg+0.25) [DOT] {};

    % ---

    \node[XOR] (g3_x2) at (\startx+\stasep,         \startyg+1.75)       {};
    \node[XOR] (g3_x3) at (\startx+\stasep+\xorsep, \startyg+1.25)       {};

    \node      (g3_e2) at (\startx+\stasep,         \startyg+0.75) [DOT] {};
    \node      (g3_e3) at (\startx+\stasep+\xorsep, \startyg+0.25) [DOT] {};

    % -----

    \node[XOR] (g3_x4) at (\startx,                 \startyg-0.25)       {};
    \node      (g3_e4) at (\startx,                 \startyg-0.75) [DOT] {};

    \node[XOR] (g3_x5) at (\startx,                 \startyg-1.25)       {};
    \node      (g3_e5) at (\startx,                 \startyg-1.75) [DOT] {};

    % ---

    \node[XOR] (g3_x6) at (\startx+\stasep,         \startyg-0.25)       {};
    \node[XOR] (g3_x7) at (\startx+\stasep+\xorsep, \startyg-0.75)       {};

    \node      (g3_e6) at (\startx+\stasep,         \startyg-1.25) [DOT] {};
    \node      (g3_e7) at (\startx+\stasep+\xorsep, \startyg-1.75) [DOT] {};

    % ----------

    \node[XOR] (g3_x8)  at (\startx+\stasep+\stasep+\xorsep,                         \startyg+1.75)       {};
    \node[XOR] (g3_x9)  at (\startx+\stasep+\stasep+\xorsep+\xorsep,                 \startyg+1.25)       {};
    \node[XOR] (g3_x10) at (\startx+\stasep+\stasep+\xorsep+\xorsep+\xorsep,         \startyg+0.75)       {};
    \node[XOR] (g3_x11) at (\startx+\stasep+\stasep+\xorsep+\xorsep+\xorsep+\xorsep, \startyg+0.25)       {};

    \node      (g3_e8)  at (\startx+\stasep+\stasep+\xorsep,                         \startyg-0.25) [DOT] {};
    \node      (g3_e9)  at (\startx+\stasep+\stasep+\xorsep+\xorsep,                 \startyg-0.75) [DOT] {};
    \node      (g3_e10) at (\startx+\stasep+\stasep+\xorsep+\xorsep+\xorsep,         \startyg-1.25) [DOT] {};
    \node      (g3_e11) at (\startx+\stasep+\stasep+\xorsep+\xorsep+\xorsep+\xorsep, \startyg-1.75) [DOT] {};

    \only<1-2>{
    \draw[-,>=latex] (\startx-\sed, \startyg+1.75) node[left] {$\textcolor{white}{0} \textcolor{white}{=} v_0$} -- (g3_x0);
    \draw[-,>=latex] (\startx-\sed, \startyg+1.25) node[left] {$\textcolor{white}{0} \textcolor{white}{=} v_1$} -- (g3_e0);
    }
    \only<3->{
    \draw[-,>=latex] (\startx-\sed, \startyg+1.75) node[left] {$\textcolor{Paired-1}{0} = v_0$} -- (g3_x0);
    \draw[-,>=latex] (\startx-\sed, \startyg+1.25) node[left] {$\textcolor{Paired-1}{0} = v_1$} -- (g3_e0);
    }
    \draw[-,>=latex] (g3_x0)                       -- (g3_e0);
    \draw[-,>=latex] (g3_x0)                       -- (g3_x2);
    \draw[-,>=latex] (g3_e0)                       -- (g3_x3);

    \only<1-2>{
    \draw[-,>=latex] (\startx-\sed, \startyg+0.75) node[left] {$\textcolor{white}{0} \textcolor{white}{=} v_2$} -- (g3_x1);
    }
    \only<3->{
    \draw[-,>=latex] (\startx-\sed, \startyg+0.75) node[left] {$\textcolor{Paired-1}{0} = v_2$} -- (g3_x1);
    }
    \draw[-,>=latex] (\startx-\sed, \startyg+0.25) node[left] {$u_0 = v_3$} -- (g3_e1);
    \draw[-,>=latex] (g3_x1)                       -- (g3_e1);
    \draw[-,>=latex] (g3_x1)                       -- (g3_e2);
    \draw[-,>=latex] (g3_e1)                       -- (g3_e3);
    \draw[-,>=latex] (g3_e2)                       -- (g3_x2);
    \draw[-,>=latex] (g3_e3)                       -- (g3_x3);
    \draw[-,>=latex] (g3_x2)                       -- (g3_x8);
    \draw[-,>=latex] (g3_x3)                       -- (g3_x9);
    \draw[-,>=latex] (g3_e2)                       -- (g3_x10);
    \draw[-,>=latex] (g3_e3)                       -- (g3_x11);

    \only<1-2>{
    \draw[-,>=latex] (\startx-\sed, \startyg-0.25) node[left] {$\textcolor{white}{0} \textcolor{white}{=} v_4$} -- (g3_x4);
    }
    \only<3->{
    \draw[-,>=latex] (\startx-\sed, \startyg-0.25) node[left] {$\textcolor{Paired-1}{0} = v_4$} -- (g3_x4);
    }
    \draw[-,>=latex] (\startx-\sed, \startyg-0.75) node[left] {$u_1 = v_5$} -- (g3_e4);
    \draw[-,>=latex] (g3_x4)                       -- (g3_e4);
    \draw[-,>=latex] (g3_x4)                       -- (g3_x6);
    \draw[-,>=latex] (g3_e4)                       -- (g3_x7);

    \draw[-,>=latex] (\startx-\sed, \startyg-1.25) node[left] {$u_2 = v_6$} -- (g3_x5);
    \draw[-,>=latex] (\startx-\sed, \startyg-1.75) node[left] {$u_3 = v_7$} -- (g3_e5);
    \draw[-,>=latex] (g3_x5)                       -- (g3_e5);
    \draw[-,>=latex] (g3_x5)                       -- (g3_e6);
    \draw[-,>=latex] (g3_e5)                       -- (g3_e7);
    \draw[-,>=latex] (g3_e6)                       -- (g3_x6);
    \draw[-,>=latex] (g3_e7)                       -- (g3_x7);
    \draw[-,>=latex] (g3_x6)                       -- (g3_e8);
    \draw[-,>=latex] (g3_x7)                       -- (g3_e9);
    \draw[-,>=latex] (g3_e6)                       -- (g3_e10);
    \draw[-,>=latex] (g3_e7)                       -- (g3_e11);

    \draw [-,>=latex] (g3_e8)                      -- (g3_x8);
    \draw [-,>=latex] (g3_e9)                      -- (g3_x9);
    \draw [-,>=latex] (g3_e10)                     -- (g3_x10);
    \draw [-,>=latex] (g3_e11)                     -- (g3_x11);
    }

    \only<2-3>{
    % \draw[-,>=latex] (\startx+\stasep+\stasep+\xorsep+\xorsep+\xorsep+\sed, \startyg+1.75) node[right] {$c_0 = v_0 \oplus v_1 \oplus v_2 \oplus v_3 \oplus v_4 \oplus v_5 \oplus v_6 \oplus v_7$} -- (g3_x8);
    \draw[-,>=latex] (\startx+\stasep+\stasep+\xorsep+\xorsep+\xorsep+\xorsep+\sed, \startyg+1.75) node[right] {$c_0 = v_0 \oplus v_1 \oplus ... \oplus v_6 \oplus v_7$} -- (g3_x8);
    \draw[-,>=latex] (\startx+\stasep+\stasep+\xorsep+\xorsep+\xorsep+\xorsep+\sed, \startyg+1.25) node[right] {$c_1 = v_1 \oplus v_3 \oplus v_5 \oplus v_7$} -- (g3_x9);
    \draw[-,>=latex] (\startx+\stasep+\stasep+\xorsep+\xorsep+\xorsep+\xorsep+\sed, \startyg+0.75) node[right] {$c_2 = v_2 \oplus v_3 \oplus v_6 \oplus v_7$} -- (g3_x10);
    \draw[-,>=latex] (\startx+\stasep+\stasep+\xorsep+\xorsep+\xorsep+\xorsep+\sed, \startyg+0.25) node[right] {$c_3 = v_3 \oplus v_7$} -- (g3_x11);
    \draw[-,>=latex] (\startx+\stasep+\stasep+\xorsep+\xorsep+\xorsep+\xorsep+\sed, \startyg-0.25) node[right] {$c_4 = v_4 \oplus v_5 \oplus v_6 \oplus v_7$} -- (g3_e8);
    \draw[-,>=latex] (\startx+\stasep+\stasep+\xorsep+\xorsep+\xorsep+\xorsep+\sed, \startyg-0.75) node[right] {$c_5 = v_5 \oplus v_7$} -- (g3_e9);
    \draw[-,>=latex] (\startx+\stasep+\stasep+\xorsep+\xorsep+\xorsep+\xorsep+\sed, \startyg-1.25) node[right] {$c_6 = v_6 \oplus v_7$} -- (g3_e10);
    \draw[-,>=latex] (\startx+\stasep+\stasep+\xorsep+\xorsep+\xorsep+\xorsep+\sed, \startyg-1.75) node[right] {$c_7 = v_7 = u_3$} -- (g3_e11);
    }

    \only<4>{
    \draw[-,>=latex] (\startx+\stasep+\stasep+\xorsep+\xorsep+\xorsep+\xorsep+\sed, \startyg+1.75) -- (g3_x8);
    \draw[-,>=latex] (\startx+\stasep+\stasep+\xorsep+\xorsep+\xorsep+\xorsep+\sed, \startyg+1.25) -- (g3_x9);
    \draw[-,>=latex] (\startx+\stasep+\stasep+\xorsep+\xorsep+\xorsep+\xorsep+\sed, \startyg+0.75) -- (g3_x10);
    \draw[-,>=latex] (\startx+\stasep+\stasep+\xorsep+\xorsep+\xorsep+\xorsep+\sed, \startyg+0.25) -- (g3_x11);
    \draw[-,>=latex] (\startx+\stasep+\stasep+\xorsep+\xorsep+\xorsep+\xorsep+\sed, \startyg-0.25) -- (g3_e8);
    \draw[-,>=latex] (\startx+\stasep+\stasep+\xorsep+\xorsep+\xorsep+\xorsep+\sed, \startyg-0.75) -- (g3_e9);
    \draw[-,>=latex] (\startx+\stasep+\stasep+\xorsep+\xorsep+\xorsep+\xorsep+\sed, \startyg-1.25) -- (g3_e10);
    \draw[-,>=latex] (\startx+\stasep+\stasep+\xorsep+\xorsep+\xorsep+\xorsep+\sed, \startyg-1.75) -- (g3_e11);

    \node[draw=Paired-1, scale=0.725, rounded corners=2pt, dashed, fit=(g3_x0) (g3_e0)] (g3_l3_1) {};
    \node[draw=Paired-1, scale=0.725, rounded corners=2pt, dashed, fit=(g3_x1) (g3_e1)] (g2_l3_2) {};
    \node[draw=Paired-1, scale=0.725, rounded corners=2pt, dashed, fit=(g3_x4) (g3_e4)] (g3_l3_3) {};
    \node[draw=Paired-1, scale=0.725, rounded corners=2pt, dashed, fit=(g3_x5) (g3_e5)] (g2_l3_4) {};

    \node[draw=Paired-3, scale=0.925, minimum width=1.45cm, rounded corners=2pt, dashed, fit=(g3_x0) (g3_x2) (g3_x3) (g3_e2) (g3_e3)] (g3_l1_1) {};
    \node[draw=Paired-3, scale=0.925, minimum width=1.45cm, rounded corners=2pt, dashed, fit=(g3_x4) (g3_x6) (g3_x7) (g3_e6) (g3_e7)] (g3_l1_2) {};

    \node[draw=Paired-5, scale=1.040, minimum width=2.2cm, rounded corners=2pt, dashed, fit=(g3_x0) (g3_e5) (g3_x8) (g3_x11) (g3_e8) (g3_e11)] (g3_l1_1) {};

    \renewcommand\startx{1.75}

    \node[draw=black, fill=white, text=black, circle, minimum width=0.5cm, scale=0.8] (g3_n0) at (\startx, \startyg+1.75) {$v_0$};
    \node[draw=black, fill=white, text=black, circle, minimum width=0.3cm, scale=0.8] (g3_n1) at (\startx, \startyg+1.25) {$v_1$};
    \node[draw=black, fill=white, text=black, circle, minimum width=0.3cm, scale=0.8] (g3_n2) at (\startx, \startyg+0.75) {$v_2$};
    \node[draw=black, fill=black, text=white, circle, minimum width=0.3cm, scale=0.8] (g3_n3) at (\startx, \startyg+0.25) {$v_3$};
    \node[draw=black, fill=white, text=black, circle, minimum width=0.3cm, scale=0.8] (g3_n4) at (\startx, \startyg-0.25) {$v_4$};
    \node[draw=black, fill=black, text=white, circle, minimum width=0.3cm, scale=0.8] (g3_n5) at (\startx, \startyg-0.75) {$v_5$};
    \node[draw=black, fill=black, text=white, circle, minimum width=0.3cm, scale=0.8] (g3_n6) at (\startx, \startyg-1.25) {$v_6$};
    \node[draw=black, fill=black, text=white, circle, minimum width=0.3cm, scale=0.8] (g3_n7) at (\startx, \startyg-1.75) {$v_7$};

    \node[draw=Paired-1, fill=Paired-1!10, circle, minimum width=0.5cm, scale=0.8] (g3_n8) at (\startx+\stasep, \startyg+1.5) {$n_0^2$};
    \node[draw=Paired-1, fill=Paired-1!10, circle, minimum width=0.5cm, scale=0.8] (g3_n9) at (\startx+\stasep, \startyg+0.5) {$n_1^2$};
    \node[draw=Paired-1, fill=Paired-1!10, circle, minimum width=0.5cm, scale=0.8] (g3_n10) at (\startx+\stasep, \startyg-0.5) {$n_2^2$};
    \node[draw=Paired-1, fill=Paired-1!10, circle, minimum width=0.5cm, scale=0.8] (g3_n11) at (\startx+\stasep, \startyg-1.5) {$n_3^2$};

    \node[draw=Paired-3, fill=Paired-3!10, circle, minimum width=0.5cm, scale=0.8] (g3_n12) at (\startx+\stasep+\stasep, \startyg+1.00) {$n_0^1$};
    \node[draw=Paired-3, fill=Paired-3!10, circle, minimum width=0.5cm, scale=0.8] (g3_n13) at (\startx+\stasep+\stasep, \startyg-1.00) {$n_1^1$};

    \node[draw=Paired-5, fill=Paired-5!10, circle, minimum width=0.5cm, scale=0.8] (g3_n14) at (\startx+\stasep+\stasep+\stasep, \startyg+0.00) {$n_0^0$};

    \draw [-,>=latex] (g3_n14) -- (g3_n13);
    \draw [-,>=latex] (g3_n14) -- (g3_n12);
    \draw [-,>=latex] (g3_n13) -- (g3_n11);
    \draw [-,>=latex] (g3_n13) -- (g3_n10);
    \draw [-,>=latex] (g3_n12) -- (g3_n9);
    \draw [-,>=latex] (g3_n12) -- (g3_n8);
    \draw [-,>=latex] (g3_n11) -- (g3_n7);
    \draw [-,>=latex] (g3_n11) -- (g3_n6);
    \draw [-,>=latex] (g3_n10) -- (g3_n5);
    \draw [-,>=latex] (g3_n10) -- (g3_n4);
    \draw [-,>=latex] (g3_n9)  -- (g3_n3);
    \draw [-,>=latex] (g3_n9)  -- (g3_n2);
    \draw [-,>=latex] (g3_n8)  -- (g3_n1);
    \draw [-,>=latex] (g3_n8)  -- (g3_n0);
    }
  \end{tikzpicture}
  \end{figure}

  \begin{itemize}
    \item<1-> $\bm{u}$ is the initial message of $K$ bits and $\bm{c}$ is the codeword of $N$ bits ($K < N$)
    % \item<2-> Generator matrix $\bm{\mathcal{G}} = \bm{\mathcal{K}}^{\otimes m}; \bm{\mathcal{K}} =
    %   \begin{bmatrix}
    %   1 & 0 \\
    %   1 & 1
    %   \end{bmatrix}; N = 2^{m}$
    \item<3-> \textcolor{Paired-1}{Frozen bits} $= N - K$ bits set to \textcolor{Paired-1}{zero}
    \item<4-> Binary tree structure
  \end{itemize}
\end{frame}

% \begin{frame}{Decoders}
%   \begin{figure}[!h]
%   \centering
%   \begin{tikzpicture}[scale=0.65, every node/.style={transform shape}]
%     % \path[use as bounding box] (-2.2, -1.2) rectangle (11.8, 5.2);
%     % \draw (-2.2, -1.2) rectangle (11.8, 5.2);

%     \tikzset{ txt/.style     ={draw=Paired-1, rounded corners=0pt, minimum height=1cm, minimum width=2.5cm, text=Paired-1, align=center} }
%     \tikzset{ rxt/.style     ={draw=Paired-3, rounded corners=0pt, minimum height=1cm, minimum width=2.5cm, text=Paired-3, align=center} }
%     \tikzset{ txtplain/.style={draw=Paired-1, fill=Paired-1!50, rounded corners=0pt, minimum height=1cm, minimum width=2.5cm, text=white, align=center} }
%     \tikzset{ rxtplain/.style={draw=Paired-3, fill=Paired-3!50, rounded corners=0pt, minimum height=1cm, minimum width=2.5cm, text=white, align=center} }

%     \node[style=txtplain,         ] (enc) at ( 4,   4) {Polar\\Encoder};
%     \node[style=txt               ] (mod) at ( 8,   4) {BPSK\\Modulator};
%     \Cloud{(11,2)}{}{0.65};
%     \node[text=black, align=center] (chn) at (10.7, 2) {AWGN\\Channel};
%     \node[style=rxt               ] (dem) at ( 8,   0) {BPSK\\Demodulator};
%     \node[style=rxtplain, thick   ] (dec) at ( 4,   0) {Polar\\Decoder};

%     \node[draw=Paired-1, rounded corners=2pt, label={[Paired-1]below:Transmitter}, minimum height=1.5cm, dashed, fit=(enc) (mod)] (tx) {};
%     \node[draw=Paired-3, rounded corners=2pt, label={[Paired-3]above:Receiver   }, minimum height=1.5cm, dashed, fit=(dem) (dec)] (rx) {};

%     \draw[<-,>=latex] (enc.west) --++ (-1.0,0) node [midway,  above           ] {$\bm{u}$};
%     \draw[->,>=latex] (enc)      --   (mod)    node [midway,  above           ] {$\bm{c}$};
%     \draw[->,>=latex] (tx.east)  --   (10,  4) node [midway,  above           ] {$\bm{x}$};
%     \draw[-         ] ( 9.9,4)   --   (10.5,4) node [antenna, above, scale=0.5] {};
%     \draw[-         ] (10  ,0)   --   (10.5,0) node [antenna, above, scale=0.5] {};
%     \draw[<-,>=latex] (rx.east)  --   (10,  0) node [midway,  above           ] {$\bm{y}$};
%     \draw[->,>=latex] (dem)      --   (dec)    node [midway,  above           ] {$\bm{l}$};
%     \draw[->,>=latex] (dec.west) --++ (-1.0,0) node [midway,  above           ] {$\bm{\hat{u}}$};
%     \draw[- ,>=latex] (mod.east) --   (10.5,4);
%     \draw[- ,>=latex] (dem.east) --   ( 9.6,0);
%   \end{tikzpicture}
%   \end{figure}

%   \begin{itemize}
%     \item<1-> Compute intensive processing
%     \item<2-> $\bm{l}$ is an input vector $N$ Log Likelihood Ratios (LLRs)
%     \item<2-> $\bm{\hat{u}}$ is an output vector of $K$ bits
%     \item<3-> Many algorithms with various complexities and decoding performances
%   \end{itemize}
% \end{frame}

\subsection[Successive Cancellation Algorithm]{Successive Cancellation Algorithm}

\begin{frame}{Successive Cancellation (SC) Algorithm}
  \begin{columns}[T] % align columns
  \begin{column}{.5\textwidth}
  \vspace{-0.3cm}
  \begin{figure}[!h]
  \centering
  \begin{tikzpicture}[scale=0.65, every node/.style={transform shape}]
    \tikzset{ txtplain/.style={draw=Paired-3, fill=Paired-3!50, rounded corners=0pt, minimum height=1cm, minimum width=2.5cm, text=white, align=center} }
    \tikzset{ zoom/.style={draw=Paired-3, dash dot} }
    \tikzset{ f/.style ={draw=Paired-5, line width=0.75pt, text=Paired-5} }
    \tikzset{ g/.style ={draw=Paired-1, line width=0.75pt, text=Paired-1} }
    \tikzset{ h/.style ={draw=Paired-3, line width=0.75pt, text=Paired-3} }

    \node[style=txtplain] (dec) at (2.6,8.9) {Polar\\Decoder};

    \draw[<-,>=latex] (dec) --++ ( 2.5,0) node [midway, above] {$\bm{l}$};
    \draw[->,>=latex] (dec) --++ (-2.5,0) node [midway, above] {$\bm{\hat{u}}$};

    \draw[zoom,-,>=latex] (dec.south west) --++ (-3.55, -1.0);
    \draw[zoom,-,>=latex] (dec.south east) --++ (+3.55, -1.0);

    \only<1>{
    \draw[zoom, fill=Paired-3!20] (-2.2,7.4) rectangle (7.4,-1.0);
    }
    \only<2->{
    \draw[zoom                  ] (-2.2,7.4) rectangle (7.4,-1.0);
    }

    \only<2->{
    \draw[fill=Gray!20, draw=none] ( 1.015, 6.96) rectangle (4.215,  6.445);
    \draw[fill=Gray!20, draw=none] (-0.200, 0.01) rectangle (0.195, -0.505);
    \draw[fill=Gray!20, draw=none] ( 0.550, 0.01) rectangle (0.945, -0.505);
    \draw[fill=Gray!20, draw=none] ( 1.300, 0.01) rectangle (1.700, -0.505);
    \draw[fill=Gray!20, draw=none] ( 2.800, 0.01) rectangle (3.200, -0.505);
    }

    \only<3->{
    \draw[fill=Gray!20, draw=none] (0.325, 4.81) rectangle (1.925, 4.295);
    }

    \only<3>{
    \draw[fill=Paired-7!20, draw=none] (1.015, 6.96) rectangle (2.615, 6.445);
    \draw[fill=Paired-9!20, draw=none] (2.615, 6.96) rectangle (4.215, 6.445);
    \draw[fill=Paired-5!20, draw=none] (0.325, 4.81) rectangle (1.925, 4.295);
    }

    \only<4->{
    \draw[fill=Gray!20, draw=none] (-0.025, 2.66) rectangle (0.775, 2.145);
    }

    \only<4>{
    \draw[fill=Paired-7!20, draw=none] (0.325, 4.81) rectangle (1.125, 4.295);
    \draw[fill=Paired-9!20, draw=none] (1.125, 4.81) rectangle (1.925, 4.295);
    \draw[fill=Paired-5!20, draw=none] (-0.025, 2.66) rectangle (0.775, 2.145);
    }

    \only<5->{
    \draw[fill=Gray!20, draw=none] (-0.2, 0.51) rectangle (0.195, -0.005);
    }

    \only<5>{
    \draw[fill=Paired-7!20, draw=none] (-0.025, 2.66) rectangle (0.375, 2.145);
    \draw[fill=Paired-9!20, draw=none] (0.375, 2.66) rectangle (0.775, 2.145);
    \draw[fill=Paired-5!20, draw=none] (-0.2, 0.51) rectangle (0.195, -0.005);
    }

    \only<6->{
    \draw[fill=gray!20, draw=none] ( 0.550, 0.51) rectangle (0.945, 0.005);
    }

    \only<6>{
    \draw[fill=Paired-7!20, draw=none] (-0.025, 2.66) rectangle (0.375, 2.145);
    \draw[fill=Paired-9!20, draw=none] (0.375, 2.66) rectangle (0.775, 2.145);
    \draw[fill=Paired-11!20, draw=none] (-0.200, 0.01) rectangle (0.195, -0.505);
    \draw[fill=Paired-1!20, draw=none] ( 0.550, 0.51) rectangle (0.945, 0.005);
    }

    \only<7->{
    \draw[fill=gray!20, draw=none] (-0.025, 2.16) rectangle (0.375,  1.645);
    \draw[fill=gray!20, draw=none] ( 0.375, 2.16) rectangle (0.775,  1.645);
    }

    \only<7>{
    \draw[fill=Paired-7!20, draw=none] (-0.200, 0.01) rectangle (0.195, -0.505);
    \draw[fill=Paired-9!20, draw=none] ( 0.550, 0.01) rectangle (0.945, -0.505);
    \draw[fill=Paired-3!20, draw=none] (-0.025, 2.16) rectangle (0.375,  1.645);
    \draw[fill=Paired-3!20, draw=none] ( 0.375, 2.16) rectangle (0.775,  1.645);
    }

    \only<8->{
    \draw[fill=gray!20, draw=none] (1.475, 2.66) rectangle (2.275, 2.145);
    }

    \only<8>{
    \draw[fill=Paired-7!20, draw=none] (0.325, 4.81) rectangle (1.125, 4.295);
    \draw[fill=Paired-9!20, draw=none] (1.125, 4.81) rectangle (1.925, 4.295);
    \draw[fill=Paired-11!20, draw=none] (-0.025, 2.16) rectangle (0.775,  1.645);
    \draw[fill=Paired-1!20, draw=none] (1.475, 2.66) rectangle (2.275, 2.145);
    }

    \only<9->{
    \draw[fill=gray!20, draw=none] (1.300, 0.51) rectangle (1.700, 0.005);
    }

    \only<9>{
    \draw[fill=Paired-7!20, draw=none] (1.475, 2.66) rectangle (1.875, 2.145);
    \draw[fill=Paired-9!20, draw=none] (1.875, 2.66) rectangle (2.275, 2.145);
    \draw[fill=Paired-5!20, draw=none] (1.300, 0.51) rectangle (1.700, 0.005);
    }

    \only<10->{
    \draw[fill=gray!20, draw=none] (2.050, 0.51) rectangle (2.450, 0.005);
    }

    \only<10>{
    \draw[fill=Paired-7!20, draw=none] (1.475, 2.66) rectangle (1.875, 2.145);
    \draw[fill=Paired-9!20, draw=none] (1.875, 2.66) rectangle (2.275, 2.145);
    \draw[fill=Paired-11!20, draw=none] ( 1.300, 0.01) rectangle (1.700, -0.505);
    \draw[fill=Paired-1!20, draw=none] (2.050, 0.51) rectangle (2.450, 0.005);
    }

    \only<11->{
    \draw[fill=gray!20, draw=none] (2.050, 0.01) rectangle (2.450,-0.495);
    }

    \only<11>{
    \draw[fill=Paired-7!20, draw=none] (2.050, 0.51) rectangle (2.450, 0.005);
    \draw[fill=gray!20, draw=none] (2.050, 0.01) rectangle (2.450,-0.495);
    }

    \only<12->{
    \draw[fill=gray!20, draw=none] (1.475, 2.16) rectangle (2.275, 1.645);
    }

    \only<12>{
    \draw[fill=Paired-7!20, draw=none] ( 1.300, 0.01) rectangle (1.700, -0.505);
    \draw[fill=Paired-9!20, draw=none] (2.050, 0.01) rectangle (2.450,-0.495);
    \draw[fill=Paired-3!20, draw=none] (1.475, 2.16) rectangle (1.875, 1.645);
    \draw[fill=Paired-3!20, draw=none] (1.875, 2.16) rectangle (2.275, 1.645);
    }

    \only<13->{
    \draw[fill=gray!20, draw=none] (0.325, 4.31) rectangle (1.925, 3.795);
    }

    \only<13>{
    \draw[fill=Paired-7!20, draw=none] (-0.025, 2.16) rectangle (0.775,  1.645);
    \draw[fill=Paired-9!20, draw=none] (1.475, 2.16) rectangle (2.275, 1.645);
    \draw[fill=Paired-3!20, draw=none] (0.325, 4.31) rectangle (1.125, 3.795);
    \draw[fill=Paired-3!20, draw=none] (1.125, 4.31) rectangle (1.925, 3.795);
    }

    \only<14>{
    \draw[fill=Paired-7!20, draw=none] (1.015, 6.96) rectangle (2.615, 6.445);
    \draw[fill=Paired-9!20, draw=none] (2.615, 6.96) rectangle (4.215, 6.445);
    \draw[fill=Paired-11!20, draw=none] (0.325, 4.31) rectangle (1.925, 3.795);
    \draw[fill=Paired-1!20, draw=none] (0.325+3, 4.81) rectangle (1.925+3, 4.295);
    }

    \only<2->{
    \newcommand\La{4} % number of layers in the tree
    \newcommand\VS{2.15} % vertical space between the tree layer

    \pgfmathtruncatemacro\Ll{\La -1}

    \foreach \l in {\Ll,...,0} {

      \pgfmathtruncatemacro\ll{2^\l-1}
      \pgfmathtruncatemacro\rl{\La-\l}
      \pgfmathtruncatemacro\rlmo{\rl -1}
      \pgfmathtruncatemacro\ee{2^\rlmo}
      \pgfmathtruncatemacro\eemo{\ee-1}

      ["\ifthenelse{\l=0}{}
      {
        \draw[dashed] (-2.00, \rlmo*\VS+\VS/2) -- (7.25, \rlmo*\VS+\VS/2);
      }", sloped]

      \node[text width=1.5cm, text centered] at (6.4, \rlmo*\VS) {\l};

      \foreach \g in {0,...,\ll} {

        \foreach \e in {0,...,\eemo} {
          \only<15->{
          \node[draw=black, minimum height=1.0cm, minimum width=0.4cm, text=black, fill=gray!20] (l\l_g\g_e\e) at (0.2*\eemo/2*1.75 + \ee*\g/2*1.5 + \e*0.4, \rlmo*\VS) {};
          }
          \draw[-, line width=0.25pt] (0.2*\eemo/2*1.75 + \ee*\g/2*1.5 + \e*0.4 -0.20, \rlmo*\VS) -- (0.2*\eemo/2*1.75 + \ee*\g/2*1.5 + \e*0.4 +0.20, \rlmo*\VS);
          \ifthenelse{\e=0 \AND \g=0}
          {
            \node[draw=black, minimum height=1.0cm, minimum width=0.4cm, text=black, label={[black]left:\small{$\ee$ ($\lambda$, $\hat{s}$)}}] (l\l_g\g_e\e) at (0.2*\eemo/2*1.75 + \ee*\g/2*1.5 + \e*0.4, \rlmo*\VS) {};
          }
          {
            \node[draw=black, minimum height=1.0cm, minimum width=0.4cm, text=black] (l\l_g\g_e\e) at (0.2*\eemo/2*1.75 + \ee*\g/2*1.5 + \e*0.4, \rlmo*\VS) {};
          }
        }

        \ifthenelse{\l=\Ll}{}
        {
          \draw (0.2*\eemo/2*1.75 + \ee*\g/2*1.5 + \eemo/2*0.4, \rlmo*\VS) node{\tiny{\textbullet}};
        }

        \only<15->{
        \ifthenelse{\l=\Ll}{}
        {
          \pgfmathtruncatemacro\ln{\l+1}
          \pgfmathtruncatemacro\gn{\g*2}
          \pgfmathtruncatemacro\gnpo{\g*2+1}
          \pgfmathtruncatemacro\eee{\eemo/2}
          \pgfmathtruncatemacro\eeemo{\eee-1}
          \pgfmathtruncatemacro\eemt{\eemo-1}
          \ifthenelse{\l<2}
          {
            \draw[->,>=latex, f] (l\l_g\g_e1.south)          -- (l\ln_g\gn_e1.north)           node [midway, text=Paired-5, fill=white] {\small{$f$}};
            \draw[->,>=latex, g] (l\l_g\g_e\eemt.south)      -- (l\ln_g\gnpo_e\eeemo.north)    node [midway, text=Paired-1, fill=white] {\small{$g$}};
            \draw[<-,>=latex, h] (l\l_g\g_e0.south west)     -- (l\ln_g\gn_e0.north west)      node [midway, fill=white] {\small{$h$}};
            \draw[<-,>=latex, h] (l\l_g\g_e\eemo.south east) -- (l\ln_g\gnpo_e\eee.north east) node [midway, fill=white] {\small{$h$}};
          }
          {
            \draw[->,>=latex, f] (l\l_g\g_e0.south)          --                (l\ln_g\gn_e0.north)           node [midway, text=Paired-5, fill=white] {\small{$f$}};
            \draw[->,>=latex, g] (l\l_g\g_e\eemo.south)      --                (l\ln_g\gnpo_e\eee.north)      node [midway, text=Paired-1, fill=white] {\small{$g$}};
            \draw[<-,>=latex, h] (l\l_g\g_e0.south west)     to[bend right=30] (l\ln_g\gn_e0.north west)      node [] {};
            \draw[<-,>=latex, h] (l\l_g\g_e\eemo.south east) to[bend left=30]  (l\ln_g\gnpo_e\eee.north east) node [] {};
          }
        }
        }
      }
    }

    \only<3->{
    \draw[->,>=latex,f] (l0_g0_e1.south) -- (l1_g0_e1.north) node [midway, text=Paired-5, fill=white] {\small{$f$}};
    }
    \only<4->{
    \draw[->,>=latex,f] (l1_g0_e1.south) -- (l2_g0_e1.north) node [midway, text=Paired-5, fill=white] {\small{$f$}};
    }
    \only<5->{
    \draw[->,>=latex,f] (l2_g0_e0.south) -- (l3_g0_e0.north) node [midway, text=Paired-5, fill=white] {\small{$f$}};
    }
    \only<6->{
    \draw[->,>=latex,g] (l2_g0_e1.south) -- (l3_g1_e0.north) node [midway, text=Paired-1, fill=white] {\small{$g$}};
    }
    \only<7->{
    \draw[<-,>=latex,h] (l2_g0_e0.south west) to[bend right=30] (l3_g0_e0.north west) node [midway, text=Paired-3, fill=white, yshift=+1.1cm, xshift=-0.55cm] {\small{$h$}};
    \draw[<-,>=latex,h] (l2_g0_e1.south east) to[bend left=30] (l3_g1_e0.north east) node [] {};
    }
    \only<8->{
    \draw[->,>=latex,g] (l1_g0_e2.south) -- (l2_g1_e0.north) node [midway, text=Paired-1, fill=white] {\small{$g$}};
    }
    \only<9->{
    \draw[->,>=latex,f] (l2_g1_e0.south) -- (l3_g2_e0.north) node [midway, text=Paired-5, fill=white] {\small{$f$}};
    }
    \only<10->{
    \draw[->,>=latex,g] (l2_g1_e1.south) -- (l3_g3_e0.north) node [midway, text=Paired-1, fill=white] {\small{$g$}};
    }
    \only<12->{
    \draw[<-,>=latex,h] (l2_g1_e0.south west) to[bend right=30] (l3_g2_e0.north west) node [] {};
    \draw[<-,>=latex,h] (l2_g1_e1.south east) to[bend left=30] (l3_g3_e0.north east) node [] {};
    }
    \only<13->{
    \draw[<-,>=latex,h] (l1_g0_e0.south west) -- (l2_g0_e0.north west) node [midway, fill=white] {\small{$h$}};
    \draw[<-,>=latex,h] (l1_g0_e3.south east) -- (l2_g1_e1.north east) node [midway, fill=white] {\small{$h$}};
    }
    \only<14->{
    \draw[->,>=latex,g] (l0_g0_e6.south) -- (l1_g1_e2.north) node [midway, text=Paired-1, fill=white] {\small{$g$}};
    }

    \pgfmathtruncatemacro\H{\VS*\La}
    \pgfmathtruncatemacro\HH{7.9}
    \draw[->,>=latex] (7.00, \HH) -- (7.00, -0.55) node [midway, text=black, fill=white,rotate=90] {\small{Layer (tree depth)}};

    \node[text width=1.5cm, text centered] at (1.225, 6.7) {$l_0$};
    \node[text width=1.5cm, text centered] at (1.625, 6.7) {$l_1$};
    \node[text width=1.5cm, text centered] at (2.025, 6.7) {$l_2$};
    \node[text width=1.5cm, text centered] at (2.425, 6.7) {$l_3$};
    \node[text width=1.5cm, text centered] at (2.825, 6.7) {$l_4$};
    \node[text width=1.5cm, text centered] at (3.225, 6.7) {$l_5$};
    \node[text width=1.5cm, text centered] at (3.625, 6.7) {$l_6$};
    \node[text width=1.5cm, text centered] at (4.025, 6.7) {$l_7$};

    \node[text width=1.5cm, text centered] at (0.00, -0.225) {$\textcolor{Paired-1}{0}$};
    \node[text width=1.5cm, text centered] at (0.75, -0.225) {$\textcolor{Paired-1}{0}$};
    \node[text width=1.5cm, text centered] at (1.50, -0.225) {$\textcolor{Paired-1}{0}$};
    \node[text width=1.5cm, text centered] at (2.25, -0.225) {$\hat{u}_0$};
    \node[text width=1.5cm, text centered] at (3.00, -0.225) {$\textcolor{Paired-1}{0}$};
    \node[text width=1.5cm, text centered] at (3.75, -0.225) {$\hat{u}_1$};
    \node[text width=1.5cm, text centered] at (4.50, -0.225) {$\hat{u}_2$};
    \node[text width=1.5cm, text centered] at (5.25, -0.225) {$\hat{u}_3$};
    }
  \end{tikzpicture}
  \end{figure}
  \end{column}
  \begin{column}{.50\textwidth}
  \vspace{-0.4cm}
  \begin{itemize}
    \item<1-> Algorithm proposed by \Arikan~\cite{Arikan2009}
    \item<2-> Traversal of a binary tree
    \item<2-> Computational complexity: $\mathcal{O}(N \times \log(N))$
  \end{itemize}

  \only<3-5>{
  \vspace{0.2cm}
  ~~~~\textbf{Polar functions:}

  \vspace{-0.4cm}
  {\small
  \begin{eqnarray*}
  \setlength\arraycolsep{1.5pt}
  ~~\left\{\begin{array}{l c l}
  \highlight[Paired-5!20]{f(\lambda_a,\lambda_b)} &\approx& \sign(\highlight[Paired-7!20]{\lambda_a}.\highlight[Paired-9!20]{\lambda_b}).\min(|\highlight[Paired-7!20]{\lambda_a}|,|\highlight[Paired-9!20]{\lambda_b}|)\\
  \textcolor{white}{g(\lambda_a,\lambda_b,\hat{s})} &\textcolor{white}{=}& \textcolor{white}{(1-2\hat{s})\lambda_a+\lambda_b}\\
  &&
  \end{array}\right.
  \label{eq:ctx_polar_f_g_h}
  \end{eqnarray*}
  }
  }

  \only<6>{
  \vspace{0.2cm}
  ~~~~\textbf{Polar functions:}

  \vspace{-0.4cm}
  {\small
  \begin{eqnarray*}
  \setlength\arraycolsep{1.5pt}
  ~~\left\{\begin{array}{l c l}
  f(\lambda_a,\lambda_b)         &\approx& \sign(\lambda_a.\lambda_b).\min(|\lambda_a|,|\lambda_b|)\\
  \highlight[Paired-1!20]{g(\lambda_a,\lambda_b,\hat{s})} &=&       (1-2\highlight[Paired-11!20]{\hat{s}})\highlight[Paired-7!20]{\lambda_a}+\highlight[Paired-9!20]{\lambda_b}\\
  &&
  \end{array}\right.
  \label{eq:ctx_polar_f_g_h}
  \end{eqnarray*}
  }
  }

  \only<7>{
  \vspace{0.2cm}
  ~~~~\textbf{Polar functions:}

  \vspace{-0.4cm}
  {\small
  \begin{eqnarray*}
  \setlength\arraycolsep{1.5pt}
  ~~\left\{\begin{array}{l c l}
  f(\lambda_a,\lambda_b)         &\approx& \sign(\lambda_a.\lambda_b).\min(|\lambda_a|,|\lambda_b|)\\
  g(\lambda_a,\lambda_b,\hat{s}) &=&       (1-2\hat{s})\lambda_a+\lambda_b\\
  \highlight[Paired-3!20]{h(\hat{s}_{a}, \hat{s}_{b})}    &=&       (\highlight[Paired-7!20]{\hat{s}_{a}} \oplus \highlight[Paired-9!20]{\hat{s}_{b}}, \highlight[Paired-9!20]{\hat{s}_{b}}).
  \end{array}\right.
  \label{eq:ctx_polar_f_g_h}
  \end{eqnarray*}
  }
  }

  \only<8>{
  \vspace{0.2cm}
  ~~~~\textbf{Polar functions:}

  \vspace{-0.4cm}
  {\small
  \begin{eqnarray*}
  \setlength\arraycolsep{1.5pt}
  ~~\left\{\begin{array}{l c l}
  f(\lambda_a,\lambda_b)         &\approx& \sign(\lambda_a.\lambda_b).\min(|\lambda_a|,|\lambda_b|)\\
  \highlight[Paired-1!20]{g(\lambda_a,\lambda_b,\hat{s})} &=&       (1-2\highlight[Paired-11!20]{\hat{s}})\highlight[Paired-7!20]{\lambda_a}+\highlight[Paired-9!20]{\lambda_b}\\
  h(\hat{s}_{a}, \hat{s}_{b})    &=&       (\hat{s}_{a} \oplus \hat{s}_{b}, \hat{s}_{b}).
  \end{array}\right.
  \label{eq:ctx_polar_f_g_h}
  \end{eqnarray*}
  }
  }

  \only<9>{
  \vspace{0.2cm}
  ~~~~\textbf{Polar functions:}

  \vspace{-0.4cm}
  {\small
  \begin{eqnarray*}
  \setlength\arraycolsep{1.5pt}
  ~~\left\{\begin{array}{l c l}
  \highlight[Paired-5!20]{f(\lambda_a,\lambda_b)} &\approx& \sign(\highlight[Paired-7!20]{\lambda_a}.\highlight[Paired-9!20]{\lambda_b}).\min(|\highlight[Paired-7!20]{\lambda_a}|,|\highlight[Paired-9!20]{\lambda_b}|)\\
  g(\lambda_a,\lambda_b,\hat{s}) &=&       (1-2\hat{s})\lambda_a+\lambda_b\\
  h(\hat{s}_{a}, \hat{s}_{b})    &=&       (\hat{s}_{a} \oplus \hat{s}_{b}, \hat{s}_{b}).
  \end{array}\right.
  \label{eq:ctx_polar_f_g_h}
  \end{eqnarray*}
  }
  }

  \only<10>{
  \vspace{0.2cm}
  ~~~~\textbf{Polar functions:}

  \vspace{-0.4cm}
  {\small
  \begin{eqnarray*}
  \setlength\arraycolsep{1.5pt}
  ~~\left\{\begin{array}{l c l}
  f(\lambda_a,\lambda_b)         &\approx& \sign(\lambda_a.\lambda_b).\min(|\lambda_a|,|\lambda_b|)\\
  \highlight[Paired-1!20]{g(\lambda_a,\lambda_b,\hat{s})} &=&       (1-2\highlight[Paired-11!20]{\hat{s}})\highlight[Paired-7!20]{\lambda_a}+\highlight[Paired-9!20]{\lambda_b}\\
  h(\hat{s}_{a}, \hat{s}_{b})    &=&       (\hat{s}_{a} \oplus \hat{s}_{b}, \hat{s}_{b}).
  \end{array}\right.
  \label{eq:ctx_polar_f_g_h}
  \end{eqnarray*}
  }
  }

  \only<11>{
  \vspace{0.2cm}
  ~~~~\textbf{Polar functions:}

  \vspace{-0.4cm}
  {\small
  \begin{eqnarray*}
  \setlength\arraycolsep{1.5pt}
  ~~\left\{\begin{array}{l c l}
  f(\lambda_a,\lambda_b)         &\approx& \sign(\lambda_a.\lambda_b).\min(|\lambda_a|,|\lambda_b|)\\
  g(\lambda_a,\lambda_b,\hat{s}) &=&       (1-2\hat{s})\lambda_a+\lambda_b\\
  h(\hat{s}_{a}, \hat{s}_{b})    &=&       (\hat{s}_{a} \oplus \hat{s}_{b}, \hat{s}_{b}).
  \end{array}\right.
  \label{eq:ctx_polar_f_g_h}
  \end{eqnarray*}
  }
  }

  \only<12-13>{
  \vspace{0.2cm}
  ~~~~\textbf{Polar functions:}

  \vspace{-0.4cm}
  {\small
  \begin{eqnarray*}
  \setlength\arraycolsep{1.5pt}
  ~~\left\{\begin{array}{l c l}
  f(\lambda_a,\lambda_b)         &\approx& \sign(\lambda_a.\lambda_b).\min(|\lambda_a|,|\lambda_b|)\\
  g(\lambda_a,\lambda_b,\hat{s}) &=&       (1-2\hat{s})\lambda_a+\lambda_b\\
  \highlight[Paired-3!20]{h(\hat{s}_{a}, \hat{s}_{b})}    &=&       (\highlight[Paired-7!20]{\hat{s}_{a}} \oplus \highlight[Paired-9!20]{\hat{s}_{b}}, \highlight[Paired-9!20]{\hat{s}_{b}}).
  \end{array}\right.
  \label{eq:ctx_polar_f_g_h}
  \end{eqnarray*}
  }
  }

  \only<14>{
  \vspace{0.2cm}
  ~~~~\textbf{Polar functions:}

  \vspace{-0.4cm}
  {\small
  \begin{eqnarray*}
  \setlength\arraycolsep{1.5pt}
  ~~\left\{\begin{array}{l c l}
  f(\lambda_a,\lambda_b)         &\approx& \sign(\lambda_a.\lambda_b).\min(|\lambda_a|,|\lambda_b|)\\
  \highlight[Paired-1!20]{g(\lambda_a,\lambda_b,\hat{s})} &=&       (1-2\highlight[Paired-11!20]{\hat{s}})\highlight[Paired-7!20]{\lambda_a}+\highlight[Paired-9!20]{\lambda_b}\\
  h(\hat{s}_{a}, \hat{s}_{b})    &=&       (\hat{s}_{a} \oplus \hat{s}_{b}, \hat{s}_{b}).
  \end{array}\right.
  \label{eq:ctx_polar_f_g_h}
  \end{eqnarray*}
  }
  }

  \only<15->{
  \vspace{0.2cm}
  ~~~~\textbf{Polar functions:}

  \vspace{-0.4cm}
  {\small
  \begin{eqnarray*}
  \setlength\arraycolsep{1.5pt}
  ~~\left\{\begin{array}{l c l}
  f(\lambda_a,\lambda_b)         &\approx& \sign(\lambda_a.\lambda_b).\min(|\lambda_a|,|\lambda_b|)\\
  g(\lambda_a,\lambda_b,\hat{s}) &=&       (1-2\hat{s})\lambda_a+\lambda_b\\
  h(\hat{s}_{a}, \hat{s}_{b})    &=&       (\hat{s}_{a} \oplus \hat{s}_{b}, \hat{s}_{b}).
  \end{array}\right.
  \label{eq:ctx_polar_f_g_h}
  \end{eqnarray*}
  }
  }

  \centering
  \only<15->{
  \scalebox{.6}{
  \setcounter{algocf}{0}
  \begin{algorithm}[H]
    \caption{Decoder SC.}\label{alg:ctx_polar_scl_decoder}

    \SetKwProg{Fn}{Function}{}{}

    \Fn{$\DecoderSC()$}
    {
      \uIf{$\text{not a leaf}$}
      {
        $f()$ \Comment*[r]{multiple elements}

        $\DecoderSC()$ \Comment*[r]{left call}

        $g()$ \Comment*[r]{multiple elements}

        $\DecoderSC()$ \Comment*[r]{right call}

        $h()$ \Comment*[r]{multiple elements}
      }
      \Else(\Comment*[f]{a leaf node})
      {
        $\hardDecide() \text{~or \textbf{frozen bit}}~;$
      }
    }
  \end{algorithm}
  }
  }

  \end{column}
  \end{columns}
\end{frame}

\begin{frame}{SC Intrinsic Parallelism}
  \begin{columns}[T] % align columns
  \begin{column}{.5\textwidth}
  \vspace{-0.3cm}
  \begin{figure}[!h]
  \centering
  \begin{tikzpicture}[scale=0.65, every node/.style={transform shape}]
    \tikzset{ txtplain/.style={draw=Paired-3, fill=Paired-3!50, rounded corners=0pt, minimum height=1cm, minimum width=2.5cm, text=white, align=center} }
    \tikzset{ zoom/.style={draw=Paired-3, dash dot} }
    \tikzset{ f/.style ={draw=Paired-5, line width=0.75pt, text=Paired-5} }
    \tikzset{ g/.style ={draw=Paired-1, line width=0.75pt, text=Paired-1} }
    \tikzset{ h/.style ={draw=Paired-3, line width=0.75pt, text=Paired-3} }

    \node[style=txtplain] (dec) at (2.6,8.9) {Polar\\Decoder};

    \draw[<-,>=latex] (dec) --++ ( 2.5,0) node [midway, above] {$\bm{l}$};
    \draw[->,>=latex] (dec) --++ (-2.5,0) node [midway, above] {$\bm{\hat{u}}$};

    \draw[zoom,-,>=latex] (dec.south west) --++ (-3.55, -1.0);
    \draw[zoom,-,>=latex] (dec.south east) --++ (+3.55, -1.0);

    \draw[zoom                  ] (-2.2,7.4) rectangle (7.4,-1.0);

    \draw[fill=Gray!20, draw=none] ( 1.015, 6.96) rectangle (4.215,  5.945);

    \only<2->{
    \draw[fill=Gray!20, draw=none] (0.325, 4.81) rectangle (1.925, 3.795);
    }

    \only<2>{
    \draw[fill=Paired-7!20, draw=none] (1.015, 6.96) rectangle (2.615, 5.945);
    \draw[fill=Paired-9!20, draw=none] (2.615, 6.96) rectangle (4.215, 5.945);
    \draw[fill=Paired-5!20, draw=none] (0.325, 4.81) rectangle (1.925, 3.795);
    }

    \only<3->{
    \draw[fill=Gray!20, draw=none] (-0.025, 2.66) rectangle (0.775, 1.645);
    }

    \only<3>{
    \draw[fill=Paired-7!20, draw=none] (0.325, 4.81) rectangle (1.125, 3.795);
    \draw[fill=Paired-9!20, draw=none] (1.125, 4.81) rectangle (1.925, 3.795);
    \draw[fill=Paired-5!20, draw=none] (-0.025, 2.66) rectangle (0.775, 1.645);
    }

    \only<4->{
    \draw[fill=Gray!20, draw=none] (-0.2, 0.51) rectangle (0.195, -0.505);
    }

    \only<4>{
    \draw[fill=Paired-7!20, draw=none] (-0.025, 2.66) rectangle (0.375, 1.645);
    \draw[fill=Paired-9!20, draw=none] (0.375, 2.66) rectangle (0.775, 1.645);
    \draw[fill=Paired-5!20, draw=none] (-0.2, 0.51) rectangle (0.195, -0.505);
    }

    \newcommand\La{4} % number of layers in the tree
    \newcommand\VS{2.15} % vertical space between the tree layer

    \pgfmathtruncatemacro\Ll{\La -1}

    \foreach \l in {\Ll,...,0} {

      \pgfmathtruncatemacro\ll{2^\l-1}
      \pgfmathtruncatemacro\rl{\La-\l}
      \pgfmathtruncatemacro\rlmo{\rl -1}
      \pgfmathtruncatemacro\ee{2^\rlmo}
      \pgfmathtruncatemacro\eemo{\ee-1}

      ["\ifthenelse{\l=0}{}
      {
        \draw[dashed] (-2.00, \rlmo*\VS+\VS/2) -- (7.25, \rlmo*\VS+\VS/2);
      }", sloped]

      \node[text width=1.5cm, text centered] at (6.4, \rlmo*\VS) {\l};

      \foreach \g in {0,...,\ll} {

        \foreach \e in {0,...,\eemo} {
          \ifthenelse{\e=0 \AND \g=0}
          {
            \node[draw=black, minimum height=1.0cm, minimum width=0.4cm, text=black, label={[black]left:\small{$\ee$ ($\lambda$)}}] (l\l_g\g_e\e) at (0.2*\eemo/2*1.75 + \ee*\g/2*1.5 + \e*0.4, \rlmo*\VS) {};
          }
          {
            \node[draw=black, minimum height=1.0cm, minimum width=0.4cm, text=black] (l\l_g\g_e\e) at (0.2*\eemo/2*1.75 + \ee*\g/2*1.5 + \e*0.4, \rlmo*\VS) {};
          }
        }

        \ifthenelse{\l=\Ll}{}
        {
          \draw (0.2*\eemo/2*1.75 + \ee*\g/2*1.5 + \eemo/2*0.4, \rlmo*\VS) node{\tiny{\textbullet}};
        }
      }
    }

    \only<2->{
    \draw[-,>=latex,f] (l0_g0_e0.south) --++ (0,-0.10) -| (l0_g0_e4.south);
    \draw[-,>=latex,f] (l0_g0_e1.south) --++ (0,-0.20) -| (l0_g0_e5.south);
    \draw[-,>=latex,f] (l0_g0_e2.south) --++ (0,-0.30) -| (l0_g0_e6.south);
    \draw[-,>=latex,f] (l0_g0_e3.south) --++ (0,-0.40) -| (l0_g0_e7.south);
    \draw[->,>=latex,f] (1.4, 5.845) -- (l1_g0_e0.north);
    \draw[->,>=latex,f] (1.8, 5.745) -- (l1_g0_e1.north);
    \draw[->,>=latex,f] (2.2, 5.645) -- (l1_g0_e2.north);
    \draw[->,>=latex,f] (2.6, 5.545) -- (l1_g0_e3.north);
    }
    \only<3->{
    \draw[-,>=latex,f] (l1_g0_e0.south) --++ (0,-0.10) -| (l1_g0_e2.south);
    \draw[-,>=latex,f] (l1_g0_e1.south) --++ (0,-0.20) -| (l1_g0_e3.south);
    \draw[->,>=latex,f] (1.4-0.675, 5.845-2.15) -- (l2_g0_e0.north);
    \draw[->,>=latex,f] (1.8-0.675, 5.745-2.15) -- (l2_g0_e1.north);
    }
    \only<4->{
    \draw[-,>=latex,f] (l2_g0_e0.south) --++ (0,-0.10) -| (l2_g0_e1.south);
    \draw[->,>=latex,f] (1.4-1.025, 5.845-2.15-2.15) -- (l3_g0_e0.north);
    }

    \pgfmathtruncatemacro\H{\VS*\La}
    \pgfmathtruncatemacro\HH{7.9}
    \draw[->,>=latex] (7.00, \HH) -- (7.00, -0.55) node [midway, text=black, fill=white,rotate=90] {\small{Layer (tree depth)}};

    \node[text width=1.5cm, text centered] at (1.225, 6.45) {$l_0$};
    \node[text width=1.5cm, text centered] at (1.625, 6.45) {$l_1$};
    \node[text width=1.5cm, text centered] at (2.025, 6.45) {$l_2$};
    \node[text width=1.5cm, text centered] at (2.425, 6.45) {$l_3$};
    \node[text width=1.5cm, text centered] at (2.825, 6.45) {$l_4$};
    \node[text width=1.5cm, text centered] at (3.225, 6.45) {$l_5$};
    \node[text width=1.5cm, text centered] at (3.625, 6.45) {$l_6$};
    \node[text width=1.5cm, text centered] at (4.025, 6.45) {$l_7$};
  \end{tikzpicture}
  \end{figure}
  \end{column}
  \begin{column}{.50\textwidth}
  \vspace{0.5cm}
  \begin{itemize}
    \item SC intrinsic parallelism decreases with the tree depth
    \begin{itemize}
      \item Intra-frame parallelism
    \end{itemize}
    \vspace{0.3cm}
    \item<5-> Regular memory access pattern
      \begin{itemize}
        \item Good for the SIMD paradigm
      \end{itemize}
    \vspace{0.3cm}
    \item<6-> \textbf{Problem:} limited SIMD efficiency near the leaves
    \vspace{0.3cm}
    \item<7-> \textbf{Solution 1:} reduce the number of nodes that have a limited amount of parallelism
  \end{itemize}
  \end{column}
  \end{columns}
\end{frame}

\begin{frame}{Simplified Successive Cancellation (SSC) Algorithm}
  \vspace{-0.2cm}
  \begin{figure}[!h]
  \centering
  \begin{tikzpicture}[scale=0.65, every node/.style={transform shape}]
    \path[use as bounding box] (-0.5,5.25) rectangle (14.0,-0.5);

    \tikzset{ any/.style ={draw=gray,     circle, minimum height=0.6cm, text=black, fill=gray!40                                                                              } }
    \tikzset{ frzn/.style={draw=black,    circle, minimum height=0.6cm, text=black                                                                                            } }
    \tikzset{ info/.style={draw=black,    circle, minimum height=0.6cm, text=black, fill=black                                                                                } }
    \tikzset{ r0/.style  ={draw=Paired-1, circle, minimum height=0.6cm, text=black, preaction={fill=Paired-1!40}, pattern=north west lines, pattern color=black!80!Paired-1!70} }
    \tikzset{ r1/.style  ={draw=Paired-3, circle, minimum height=0.6cm, text=black, preaction={fill=Paired-3!40}, pattern=north east lines, pattern color=black!80!Paired-3!70} }
    \tikzset{ rep/.style ={draw=Paired-7, circle, minimum height=0.6cm, text=black, preaction={fill=Paired-7!40}, pattern=crosshatch dots,  pattern color=black!80!Paired-7!70} }
    \tikzset{ spc4/.style={draw=Paired-5, circle, minimum height=0.6cm, text=black, preaction={fill=Paired-5!40}, pattern=horizontal lines, pattern color=black!80!Paired-5!70} }
    \tikzset{ spc/.style ={draw=Paired-9, circle, minimum height=0.6cm, text=black, preaction={fill=Paired-9!40}, pattern=grid,             pattern color=black!80!Paired-9!70} }

    \tikzset{ f/.style ={draw=black, text=black} }
    \tikzset{ g/.style ={draw=black, text=black} }
    \tikzset{ h/.style ={draw=black, text=black} }

    \node[frzn                                           ] (l3_g0) at (0.0, 0.0) {};
    \node[frzn                                           ] (l3_g1) at (1.0, 0.0) {};
    \node[frzn                                           ] (l3_g2) at (2.0, 0.0) {};
    \node[info                                           ] (l3_g3) at (3.0, 0.0) {};
    \node[frzn                                           ] (l3_g4) at (4.0, 0.0) {};
    \node[info                                           ] (l3_g5) at (5.0, 0.0) {};
    \node[info                                           ] (l3_g6) at (6.0, 0.0) {};
    \node[info                                           ] (l3_g7) at (7.0, 0.0) {};
    \only<1>{
    \node[any, label={[black]above:\small{\texttt{?}}}   ] (l2_g0) at (0.5, 1.5) {};
    }
    \only<2->{
    \node[r0,  label={[black]above:\small{\texttt{R0}}}  ] (l2_g0) at (0.5, 1.5) {};
    }
    \only<1-2>{
    \node[any, label={[black]above:\small{\texttt{?}}}   ] (l2_g1) at (2.5, 1.5) {};
    }
    \only<3->{
    \node[rep, label={[black]above:\small{\texttt{REP}}} ] (l2_g1) at (2.5, 1.5) {};
    }
    \only<1-3>{
    \node[any, label={[black]above:\small{\texttt{?}}}   ] (l2_g2) at (4.5, 1.5) {};
    }
    \only<4->{
    \node[rep, label={[black]above:\small{\texttt{REP}}} ] (l2_g2) at (4.5, 1.5) {};
    }
    \only<1-4>{
    \node[any, label={[black]above:\small{\texttt{?}}}   ] (l2_g3) at (6.5, 1.5) {};
    }
    \only<5->{
    \node[r1,  label={[black]above:\small{\texttt{R1}}}  ] (l2_g3) at (6.5, 1.5) {};
    }
    \only<1-5>{
    \node[any, label={[black]above:\small{\texttt{?}}}   ] (l1_g0) at (1.5, 3.0) {};
    }
    \only<6->{
    \node[rep, label={[black]above:\small{\texttt{REP}}} ] (l1_g0) at (1.5, 3.0) {};
    }
    \only<1-6>{
    \node[any, label={[black]above:\small{\texttt{?}}}   ] (l1_g1) at (5.5, 3.0) {};
    }
    \only<7->{
    \node[spc4, label={[black]above:\small{\texttt{SPC}}}] (l1_g1) at (5.5, 3.0) {};
    }
    \node[any,  label={[black]above:\small{\texttt{?}}}  ] (l0_g0) at (3.5, 4.5) {};

    \only<1>{
    \draw[f,->,>=latex] (l2_g0) -- (l3_g0) node [midway, above, text=black, xshift=-0.1cm, yshift=-0.1cm] {$f$};
    \draw[g,->,>=latex] (l2_g0) -- (l3_g1) node [midway, above, text=black, xshift=+0.1cm, yshift=-0.1cm] {$g$};
    \draw[h,->,>=latex] (l2_g0) to [out=180,in=120,looseness=8] (l2_g0) node [right, text=black, xshift=-1.25cm, yshift=0.4cm] {$h$};
    }
    \only<2->{
    \draw[->,>=latex] (l2_g0) -- (l3_g0) node [midway, text=black, sloped] {|};
    \draw[->,>=latex] (l2_g0) -- (l3_g1) node [midway, text=black, sloped] {|};
    }
    \only<1-2>{
    \draw[f,->,>=latex] (l2_g1) -- (l3_g2) node [midway, above, text=black, xshift=-0.1cm, yshift=-0.1cm] {$f$};
    \draw[g,->,>=latex] (l2_g1) -- (l3_g3) node [midway, above, text=black, xshift=+0.1cm, yshift=-0.1cm] {$g$};
    \draw[h,->,>=latex] (l2_g1) to [out=  0,in= 60,looseness=8] (l2_g1) node [right, text=black, xshift= 0.75cm, yshift=0.4cm] {$h$};
    }
    \only<3->{
    \draw[->,>=latex] (l2_g1) -- (l3_g2) node [midway, text=black, sloped] {|};
    \draw[->,>=latex] (l2_g1) -- (l3_g3) node [midway, text=black, sloped] {|};
    }
    \only<1-3>{
    \draw[f,->,>=latex] (l2_g2) -- (l3_g4) node [midway, above, text=black, xshift=-0.1cm, yshift=-0.1cm] {$f$};
    \draw[g,->,>=latex] (l2_g2) -- (l3_g5) node [midway, above, text=black, xshift=+0.1cm, yshift=-0.1cm] {$g$};
    \draw[h,->,>=latex] (l2_g2) to [out=180,in=120,looseness=8] (l2_g2) node [right, text=black, xshift=-1.25cm, yshift=0.4cm] {$h$};
    }
    \only<4->{
    \draw[->,>=latex] (l2_g2) -- (l3_g4) node [midway, text=black, sloped] {|};
    \draw[->,>=latex] (l2_g2) -- (l3_g5) node [midway, text=black, sloped] {|};
    }
    \only<1-4>{
    \draw[f,->,>=latex] (l2_g3) -- (l3_g6) node [midway, above, text=black, xshift=-0.1cm, yshift=-0.1cm] {$f$};
    \draw[g,->,>=latex] (l2_g3) -- (l3_g7) node [midway, above, text=black, xshift=+0.1cm, yshift=-0.1cm] {$g$};
    \draw[h,->,>=latex] (l2_g3) to [out=  0,in= 60,looseness=8] (l2_g3) node [right, text=black, xshift= 0.75cm, yshift=0.4cm] {$h$};
    }
    \only<5->{
    \draw[->,>=latex] (l2_g3) -- (l3_g6) node [midway, text=black, sloped] {|};
    \draw[->,>=latex] (l2_g3) -- (l3_g7) node [midway, text=black, sloped] {|};
    }
    \only<1-5>{
    \draw[f,->,>=latex] (l1_g0) -- (l2_g0) node [midway, above, text=black] {$f$};
    \draw[g,->,>=latex] (l1_g0) -- (l2_g1) node [midway, above, text=black] {$g$};
    \draw[h,->,>=latex] (l1_g0) to [out=180,in=120,looseness=8] (l1_g0) node [right, text=black, xshift=-1.25cm, yshift=0.4cm] {$h$};
    }
    \only<6->{
    \draw[->,>=latex] (l1_g0) -- (l2_g0) node [midway, text=black, sloped] {|};
    \draw[->,>=latex] (l1_g0) -- (l2_g1) node [midway, text=black, sloped] {|};
    }
    \only<1-6>{
    \draw[f,->,>=latex] (l1_g1) -- (l2_g2) node [midway, above, text=black] {$f$};
    \draw[g,->,>=latex] (l1_g1) -- (l2_g3) node [midway, above, text=black] {$g$};
    \draw[h,->,>=latex] (l1_g1) to [out=  0,in= 60,looseness=8] (l1_g1) node [right, text=black, xshift= 0.75cm, yshift=0.4cm] {$h$};
    }
    \only<7->{
    \draw[->,>=latex] (l1_g1) -- (l2_g2) node [midway, text=black, sloped] {|};
    \draw[->,>=latex] (l1_g1) -- (l2_g3) node [midway, text=black, sloped] {|};
    }

    \draw[f,->,>=latex] (l0_g0) -- (l1_g0) node [midway, above, text=black] {$f$};
    \draw[g,->,>=latex] (l0_g0) -- (l1_g1) node [midway, above, text=black] {$g$};

    \draw[h,->,>=latex] (l0_g0) to [out=0,in=60,looseness=8] (l0_g0) node [right, text=black, xshift=0.75cm, yshift=0.4cm] {$h$};

    \only<8->{
    \draw[->,>=latex, black, line width=1.0pt] (7.5, 2.25) -> (8.5, 2.25);

    \node[rep,  label={[black]above:\small{\texttt{REP}}}] (l1_g0_cut) at ( 9.5, 1.5) {};
    \node[spc4, label={[black]above:\small{\texttt{SPC}}}] (l1_g1_cut) at (13.5, 1.5) {};
    \node[any,  label={[black]above:\small{\texttt{?}}}  ] (l0_g0_cut) at (11.5, 3.0) {};

    \draw[f,->,>=latex] (l0_g0_cut) -- (l1_g0_cut) node [midway, above, text=black] {$f$};
    \draw[g,->,>=latex] (l0_g0_cut) -- (l1_g1_cut) node [midway, above, text=black] {$g$};
    \draw[h,->,>=latex] (l0_g0_cut) to [out=0,in=60,looseness=8] (l0_g0_cut) node [right, text=black, xshift=0.75cm, yshift=0.4cm] {$h$};

    \draw[->,>=latex] (l1_g0_cut) to [out=-120,in=-60,looseness=8] (l1_g0_cut) node [left, text=black, xshift=0.4cm, yshift=-1.15cm] {$rep$};
    \draw[->,>=latex] (l1_g1_cut) to [out=-120,in=-60,looseness=8] (l1_g1_cut) node [left, text=black, xshift=0.4cm, yshift=-1.15cm] {$spc$};
    }
  \end{tikzpicture}
  \end{figure}

  \begin{itemize}
    \item SC simplifications have been proposed in the literature~\cite{Alamdar-Yazdi2011,Sarkis2014a}
    \begin{itemize}
      \item Introducing new type of nodes => tree pruning
      % \item<8-> Drastically reduce the number of nodes with a low level of parallelism
      \item<8-> Drastically reduce the number of nodes: \textbf{from 2047 nodes to 291 nodes} ($N = 1024$ and $R = 1/2$)
    \end{itemize}
  \end{itemize}
  \vfill
  \enumcite{Alamdar-Yazdi2011}
\end{frame}

\begin{frame}{SC SIMDization Strategies}
  \vfill
  \begin{columns}[t]
    \begin{column}[T]{5.0cm}
      \textbf{Intra-frame SIMD strategy:}
      \begin{figure}[!h]
      \centering
      \scalebox{.65}{
      \begin{tikzpicture}
        \tikzset{ la/.style ={draw=black, minimum width=0.7cm, minimum height=0.7cm, text=black, fill=Paired-7!20         } }
        \tikzset{ lb/.style ={draw=black, minimum width=0.7cm, minimum height=0.7cm, text=black, fill=Paired-9!20         } }
        \tikzset{ lc/.style ={draw=black, minimum width=0.7cm, minimum height=0.7cm, text=black, fill=Paired-5!20         } }
        \tikzset{ lh/.style ={draw=white, minimum width=0.7cm, minimum height=0.7cm, text=black, fill=white               } }

        \tikzstyle{instr}=[->,>=stealth,rounded corners=3pt, fill=white, draw=Paired-5, text=Paired-5]
        \tikzstyle{lnk}=[->,>=stealth,rounded corners=3pt]

        \newcommand\vs{3.0}
        \newcommand\lft{0.0}
        \newcommand\ctr{4.9}
        \newcommand\rth{9.8}

        \node[lh] (hack) at (\lft+0.0, -\vs+0.6) {};

        \node[la] (r0_e0) at (\lft+0.0, -\vs) {\footnotesize{$\lambda_{a}^0$}};
        \node[la] (r0_e1) at (\lft+0.7, -\vs) {\footnotesize{$\lambda_{a}^1$}};
        \node[la] (r0_e2) at (\lft+1.4, -\vs) {\footnotesize{$\lambda_{a}^2$}};
        \node[la] (r0_e3) at (\lft+2.1, -\vs) {\footnotesize{$\lambda_{a}^3$}};
        \node[lb] (r0_e4) at (\lft+2.8, -\vs) {\footnotesize{$\lambda_{b}^0$}};
        \node[lb] (r0_e5) at (\lft+3.5, -\vs) {\footnotesize{$\lambda_{b}^1$}};
        \node[lb] (r0_e6) at (\lft+4.2, -\vs) {\footnotesize{$\lambda_{b}^2$}};
        \node[lb] (r0_e7) at (\lft+4.9, -\vs) {\footnotesize{$\lambda_{a}^3$}};

        \node[instr] (instr0) at (\lft+2.45, -\vs-\vs+1.5) {\footnotesize{$\lambda_{c}^i = f(\lambda_{a}^i, \lambda_{b}^i)$}};

        \draw[lnk] (r0_e1.south east) |- (instr0.west);
        \draw[lnk] (r0_e6.south west) |- (instr0.east);
        \draw[lnk] (instr0.south) -- (2.45,-\vs-\vs+0.35);

        \node[lc] (r0_e0) at (\lft+1.4, -\vs-\vs) {\footnotesize{$\lambda_{c}^0$}};
        \node[lc] (r0_e1) at (\lft+2.1, -\vs-\vs) {\footnotesize{$\lambda_{c}^1$}};
        \node[lc] (r0_e2) at (\lft+2.8, -\vs-\vs) {\footnotesize{$\lambda_{c}^2$}};
        \node[lc] (r0_e3) at (\lft+3.5, -\vs-\vs) {\footnotesize{$\lambda_{c}^3$}};
      \end{tikzpicture}
      }
      \end{figure}
    \end{column}
    \begin{column}[T]{5.0cm}
      \textbf{Inter-frame SIMD strategy:}
      \begin{figure}[!h]
      \centering
      \scalebox{.65}{
      \begin{tikzpicture}
        \tikzset{ e/.style  ={draw=black, minimum width=0.7cm, minimum height=0.7cm, text=black, fill=gray!40             } }
        \tikzset{ en/.style ={draw=black, minimum width=0.7cm, minimum height=0.7cm, text=black, fill=Paired-5!40         } }
        \tikzset{ ens/.style={draw=black, minimum width=0.7cm, minimum height=0.7cm, text=black, fill=Paired-6!40         } }
        \tikzset{ enn/.style={draw=black, minimum width=0.7cm, minimum height=0.7cm, text=black, fill=gray!40!Paired-5!40 } }
        \tikzset{ ep/.style ={draw=black, minimum width=0.7cm, minimum height=0.7cm, text=black, fill=Paired-7!40         } }
        \tikzset{ eps/.style={draw=black, minimum width=0.7cm, minimum height=0.7cm, text=black, fill=Paired-8!40         } }
        \tikzset{ ept/.style={            minimum width=0.7cm, minimum height=0.7cm] } }

        \tikzset{ la/.style ={draw=black, minimum width=0.7cm, minimum height=0.7cm, text=black, fill=Paired-7!20         } }
        \tikzset{ lb/.style ={draw=black, minimum width=0.7cm, minimum height=0.7cm, text=black, fill=Paired-9!20         } }
        \tikzset{ lc/.style ={draw=black, minimum width=0.7cm, minimum height=0.7cm, text=black, fill=Paired-5!20         } }

        \tikzset{ lah/.style ={draw=black, minimum width=0.7cm, minimum height=0.7cm, text=black, fill=white         } }
        \tikzset{ lbh/.style ={draw=black, minimum width=0.7cm, minimum height=0.7cm, text=black, fill=white         } }
        \tikzset{ lch/.style ={draw=black, minimum width=0.7cm, minimum height=0.7cm, text=black, fill=white         } }

        \tikzstyle{lnk}=[->,>=stealth,rounded corners=3pt]
        \tikzstyle{shf}=[black!20]
        \tikzstyle{instr}=[->,>=stealth,rounded corners=3pt, fill=white, draw=Paired-5, text=Paired-5]

        \newcommand\vs{3.0}
        \newcommand\lft{0.0}
        \newcommand\ctr{4.9}
        \newcommand\rth{9.8}

        \newcommand\shfx{0.45}
        \newcommand\shfy{0.20}

        \node[la ] (r3_e0) at (\lft+0.0+\shfx+\shfx+\shfx, -\vs+\shfy+\shfy+\shfy) {};
        \node[lah] (r3_e1) at (\lft+0.7+\shfx+\shfx+\shfx, -\vs+\shfy+\shfy+\shfy) {};
        \node[lah] (r3_e2) at (\lft+1.4+\shfx+\shfx+\shfx, -\vs+\shfy+\shfy+\shfy) {};
        \node[lah] (r3_e3) at (\lft+2.1+\shfx+\shfx+\shfx, -\vs+\shfy+\shfy+\shfy) {};
        \node[lb ] (r3_e4) at (\lft+2.8+\shfx+\shfx+\shfx, -\vs+\shfy+\shfy+\shfy) {};
        \node[lbh] (r3_e5) at (\lft+3.5+\shfx+\shfx+\shfx, -\vs+\shfy+\shfy+\shfy) {};
        \node[lbh] (r3_e6) at (\lft+4.2+\shfx+\shfx+\shfx, -\vs+\shfy+\shfy+\shfy) {};
        \node[lbh] (r3_e7) at (\lft+4.9+\shfx+\shfx+\shfx, -\vs+\shfy+\shfy+\shfy) {};

        \node[la ] (r2_e0) at (\lft+0.0+\shfx+\shfx, -\vs+\shfy+\shfy) {};
        \node[lah] (r2_e1) at (\lft+0.7+\shfx+\shfx, -\vs+\shfy+\shfy) {};
        \node[lah] (r2_e2) at (\lft+1.4+\shfx+\shfx, -\vs+\shfy+\shfy) {};
        \node[lah] (r2_e3) at (\lft+2.1+\shfx+\shfx, -\vs+\shfy+\shfy) {};
        \node[lb ] (r2_e4) at (\lft+2.8+\shfx+\shfx, -\vs+\shfy+\shfy) {};
        \node[lbh] (r2_e5) at (\lft+3.5+\shfx+\shfx, -\vs+\shfy+\shfy) {};
        \node[lbh] (r2_e6) at (\lft+4.2+\shfx+\shfx, -\vs+\shfy+\shfy) {};
        \node[lbh] (r2_e7) at (\lft+4.9+\shfx+\shfx, -\vs+\shfy+\shfy) {};

        \node[la ] (r1_e0) at (\lft+0.0+\shfx, -\vs+\shfy) {};
        \node[lah] (r1_e1) at (\lft+0.7+\shfx, -\vs+\shfy) {};
        \node[lah] (r1_e2) at (\lft+1.4+\shfx, -\vs+\shfy) {};
        \node[lah] (r1_e3) at (\lft+2.1+\shfx, -\vs+\shfy) {};
        \node[lb ] (r1_e4) at (\lft+2.8+\shfx, -\vs+\shfy) {};
        \node[lbh] (r1_e5) at (\lft+3.5+\shfx, -\vs+\shfy) {};
        \node[lbh] (r1_e6) at (\lft+4.2+\shfx, -\vs+\shfy) {};
        \node[lbh] (r1_e7) at (\lft+4.9+\shfx, -\vs+\shfy) {};

        \node[la ] (r0_e0) at (\lft+0.0, -\vs) {\footnotesize{$\lambda_{a}^0$}};
        \node[lah] (r0_e1) at (\lft+0.7, -\vs) {\footnotesize{$\lambda_{a}^1$}};
        \node[lah] (r0_e2) at (\lft+1.4, -\vs) {\footnotesize{$\lambda_{a}^2$}};
        \node[lah] (r0_e3) at (\lft+2.1, -\vs) {\footnotesize{$\lambda_{a}^3$}};
        \node[lb ] (r0_e4) at (\lft+2.8, -\vs) {\footnotesize{$\lambda_{b}^0$}};
        \node[lbh] (r0_e5) at (\lft+3.5, -\vs) {\footnotesize{$\lambda_{b}^1$}};
        \node[lbh] (r0_e6) at (\lft+4.2, -\vs) {\footnotesize{$\lambda_{b}^2$}};
        \node[lbh] (r0_e7) at (\lft+4.9, -\vs) {\footnotesize{$\lambda_{a}^3$}};

        \node[instr] (instr0) at (\lft+1.4, -\vs-\vs+1.5) {\footnotesize{$\lambda_{c}^0 = f(\lambda_{a}^0, \lambda_{b}^0)$}};

        \draw[lnk] (r0_e0.south) |- (instr0.west);
        \draw[lnk] (r0_e4.south) |- (instr0.east);
        \draw[lnk] (instr0.south) -- (\lft+1.4,-\vs-\vs+0.35);

        \draw[<->,>=stealth,rounded corners=3pt] (\lft+4.9+0.6, -\vs-0.35) -- (\lft+4.9+\shfx+\shfx+\shfx+0.60, -\vs+\shfy+\shfy+\shfy-0.35) node [midway, text=black, fill=none, sloped, yshift=-0.2cm] {\footnotesize{n frames}};

        \node[lc ] (r3_e0) at (\lft+1.4+\shfx+\shfx+\shfx, -\vs-\vs+\shfy+\shfy+\shfy) {};
        \node[lch] (r3_e1) at (\lft+2.1+\shfx+\shfx+\shfx, -\vs-\vs+\shfy+\shfy+\shfy) {};
        \node[lch] (r3_e2) at (\lft+2.8+\shfx+\shfx+\shfx, -\vs-\vs+\shfy+\shfy+\shfy) {};
        \node[lch] (r3_e3) at (\lft+3.5+\shfx+\shfx+\shfx, -\vs-\vs+\shfy+\shfy+\shfy) {};

        \node[lc ] (r2_e0) at (\lft+1.4+\shfx+\shfx, -\vs-\vs+\shfy+\shfy) {};
        \node[lch] (r2_e1) at (\lft+2.1+\shfx+\shfx, -\vs-\vs+\shfy+\shfy) {};
        \node[lch] (r2_e2) at (\lft+2.8+\shfx+\shfx, -\vs-\vs+\shfy+\shfy) {};
        \node[lch] (r2_e3) at (\lft+3.5+\shfx+\shfx, -\vs-\vs+\shfy+\shfy) {};

        \node[lc ] (r1_e0) at (\lft+1.4+\shfx, -\vs-\vs+\shfy) {};
        \node[lch] (r1_e1) at (\lft+2.1+\shfx, -\vs-\vs+\shfy) {};
        \node[lch] (r1_e2) at (\lft+2.8+\shfx, -\vs-\vs+\shfy) {};
        \node[lch] (r1_e3) at (\lft+3.5+\shfx, -\vs-\vs+\shfy) {};

        \node[lc ] (r0_e0) at (\lft+1.4, -\vs-\vs) {\footnotesize{$\lambda_{c}^0$}};
        \node[lch] (r0_e1) at (\lft+2.1, -\vs-\vs) {\footnotesize{$\lambda_{c}^1$}};
        \node[lch] (r0_e2) at (\lft+2.8, -\vs-\vs) {\footnotesize{$\lambda_{c}^2$}};
        \node[lch] (r0_e3) at (\lft+3.5, -\vs-\vs) {\footnotesize{$\lambda_{c}^3$}};
      \end{tikzpicture}
      }
      \end{figure}
    \end{column}
  \end{columns}
  \vfill
  \pause
  \begin{columns}[t]
    \begin{column}[T]{5.5cm}
      \begin{itemize}
        \item $+$ low latency
        \item $-$ leaves cannot be vectorized
      \end{itemize}
    \end{column}
    \begin{column}[T]{5.5cm}
      \begin{itemize}
        \item $+$ high throughput
        \item $-$ high latency
      \end{itemize}
    \end{column}
  \end{columns}
  \vfill
\end{frame}

% \begin{frame}[containsverbatim,fragile]{Polar Application Programming Interface (Polar API)}
%   \vfill
%   \begin{itemize}
%     \item Polar functions ($f$, $g$ and $h$) implemented with \textbf{a portable SIMD library}
%     \vspace{0.2cm}
%     \item Portability
%     \begin{itemize}
%       \item Same description for NEON, SSE, AVX and AVX-512
%       \item Independant from the vectorization strategy (intra-/inter-frame)
%     \end{itemize}
%     \vspace{0.2cm}
%     \pause
%     \item Flexibility
%     \begin{itemize}
%       \item Same description for real floating-point and fixed-point representation
%       \item Compatible with: \verb|double|, \verb|float|, \verb|int8_t| and \verb|int16_t| types
%     \end{itemize}
%     \vspace{0.2cm}
%     \pause
%     \item Efficiency
%     \begin{itemize}
%       \item High througput (intra-/inter-frame)
%       \item Low latency (intra-frame)
%     \end{itemize}
%   \end{itemize}
%   \vfill
% %   \begin{minted}[linenos]{cpp}
% % template <typename R>
% % mipp::<R> f(const mipp::<R> la, const mipp::<R> lb) {
% %   return mipp::copysign(mipp::min(mipp::abs(la), mipp::abs(lb)),
% %                         mipp::sign(la, lb));
% % } // f(la, lb) = sign(la.lb).min(|la|, |lb|)

% % template <typename B, typename R>
% % mipp::<R> g(const mipp::<R> la, const mipp::<R> lb, const mipp::<B> s) {
% %   return mipp::copysign(la, s) + lb;
% % } // g(la, lb, sa) = (1-2s)la + lb

% % template <typename B>
% % mipp::<B> h(const mipp::<B> sa, const mipp::<B> sb) {
% %   return sa ^ sb;
% % } // h(sa, sb) = sa XOR sb
% %   \end{minted}
% \end{frame}

\begin{frame}{SC Dynamic Versus Generated Implementation}
  \vspace{-0.4cm}
  \begin{columns}[t]
    \begin{column}[T]{6.0cm}
      \centering
      \scalebox{.65}{
      \setcounter{algocf}{0}
      \begin{algorithm}[H]
        \caption{Decoder SC Dynamic.}

        \SetKwProg{Fn}{Function}{}{}

        \Fn{$\DecoderSC()$}
        {
          \uIf{$\text{not a leaf}$}
          {
            $f()$ \Comment*[r]{multiple elements}

            $\DecoderSC()$ \Comment*[r]{left call}

            $g()$ \Comment*[r]{multiple elements}

            $\DecoderSC()$ \Comment*[r]{right call}

            $h()$ \Comment*[r]{multiple elements}
          }
          \Else(\Comment*[f]{a leaf node})
          {
            $\hardDecide() \text{~or \textbf{frozen bit}}~;$
          }
        }
      \end{algorithm}
      }

      \only<1-2>{
      \begin{figure}[!h]
      \centering
      \begin{tikzpicture}[scale=0.65, every node/.style={transform shape}]
        % \path[use as bounding box] (-0.5,5.25) rectangle (14.0,-0.5);

        \tikzset{ any/.style ={draw=gray,     circle, minimum height=0.6cm, text=black, fill=gray!40                                                                              } }
        \tikzset{ frzn/.style={draw=black,    circle, minimum height=0.6cm, text=black                                                                                            } }
        \tikzset{ info/.style={draw=black,    circle, minimum height=0.6cm, text=white, fill=black                                                                                } }
        \tikzset{ r0/.style  ={draw=Paired-1, circle, minimum height=0.6cm, text=black, preaction={fill=Paired-1!40}, pattern=north west lines, pattern color=black!80!Paired-1!70} }
        \tikzset{ r1/.style  ={draw=Paired-3, circle, minimum height=0.6cm, text=black, preaction={fill=Paired-3!40}, pattern=north east lines, pattern color=black!80!Paired-3!70} }
        \tikzset{ rep/.style ={draw=Paired-7, circle, minimum height=0.6cm, text=black, preaction={fill=Paired-7!40}, pattern=crosshatch dots,  pattern color=black!80!Paired-7!70} }
        \tikzset{ spc4/.style={draw=Paired-5, circle, minimum height=0.6cm, text=black, preaction={fill=Paired-5!40}, pattern=horizontal lines, pattern color=black!80!Paired-5!70} }
        \tikzset{ spc/.style ={draw=Paired-9, circle, minimum height=0.6cm, text=black, preaction={fill=Paired-9!40}, pattern=grid,             pattern color=black!80!Paired-9!70} }

        \tikzset{ f/.style ={draw=black, text=black} }
        \tikzset{ g/.style ={draw=black, text=black} }
        \tikzset{ h/.style ={draw=black, text=black} }

        \only<1>{
        \node[frzn] (l3_g0) at (0.0, 0.0) {};
        \node[frzn] (l3_g1) at (1.0, 0.0) {};
        \node[frzn] (l3_g2) at (2.0, 0.0) {};
        \node[info] (l3_g3) at (3.0, 0.0) {};
        \node[frzn] (l3_g4) at (4.0, 0.0) {};
        \node[info] (l3_g5) at (5.0, 0.0) {};
        \node[info] (l3_g6) at (6.0, 0.0) {};
        \node[info] (l3_g7) at (7.0, 0.0) {};
        \node[any ] (l2_g0) at (0.5, 1.5) {};
        \node[any ] (l2_g1) at (2.5, 1.5) {};
        \node[any ] (l2_g2) at (4.5, 1.5) {};
        \node[any ] (l2_g3) at (6.5, 1.5) {};
        \node[any ] (l1_g0) at (1.5, 3.0) {};
        \node[any ] (l1_g1) at (5.5, 3.0) {};
        \node[any ] (l0_g0) at (3.5, 4.5) {};
        }

        \only<2->{
        \node[frzn] (l3_g0) at (0.0, 0.0) {\small{4}};
        \node[frzn] (l3_g1) at (1.0, 0.0) {\small{5}};
        \node[frzn] (l3_g2) at (2.0, 0.0) {\small{7}};
        \node[info] (l3_g3) at (3.0, 0.0) {\small{8}};
        \node[frzn] (l3_g4) at (4.0, 0.0) {};
        \node[info] (l3_g5) at (5.0, 0.0) {};
        \node[info] (l3_g6) at (6.0, 0.0) {};
        \node[info] (l3_g7) at (7.0, 0.0) {};
        \node[any ] (l2_g0) at (0.5, 1.5) {\small{3}};
        \node[any ] (l2_g1) at (2.5, 1.5) {\small{6}};
        \node[any ] (l2_g2) at (4.5, 1.5) {};
        \node[any ] (l2_g3) at (6.5, 1.5) {};
        \node[any ] (l1_g0) at (1.5, 3.0) {\small{2}};
        \node[any ] (l1_g1) at (5.5, 3.0) {\small{9}};
        \node[any ] (l0_g0) at (3.5, 4.5) {\small{1}};
        }

        \draw[f,->,>=latex] (l2_g0) -- (l3_g0) node [midway, above, text=black, xshift=-0.1cm, yshift=-0.1cm] {$f$};
        \draw[g,->,>=latex] (l2_g0) -- (l3_g1) node [midway, above, text=black, xshift=+0.1cm, yshift=-0.1cm] {$g$};
        \draw[h,->,>=latex] (l2_g0) to [out=180,in=120,looseness=8] (l2_g0) node [right, text=black, xshift=-1.25cm, yshift=0.4cm] {$h$};
        \draw[f,->,>=latex] (l2_g1) -- (l3_g2) node [midway, above, text=black, xshift=-0.1cm, yshift=-0.1cm] {$f$};
        \draw[g,->,>=latex] (l2_g1) -- (l3_g3) node [midway, above, text=black, xshift=+0.1cm, yshift=-0.1cm] {$g$};
        \draw[h,->,>=latex] (l2_g1) to [out=  0,in= 60,looseness=8] (l2_g1) node [right, text=black, xshift= 0.75cm, yshift=0.4cm] {$h$};
        \draw[f,->,>=latex] (l2_g2) -- (l3_g4) node [midway, above, text=black, xshift=-0.1cm, yshift=-0.1cm] {$f$};
        \draw[g,->,>=latex] (l2_g2) -- (l3_g5) node [midway, above, text=black, xshift=+0.1cm, yshift=-0.1cm] {$g$};
        \draw[h,->,>=latex] (l2_g2) to [out=180,in=120,looseness=8] (l2_g2) node [right, text=black, xshift=-1.25cm, yshift=0.4cm] {$h$};
        \draw[f,->,>=latex] (l2_g3) -- (l3_g6) node [midway, above, text=black, xshift=-0.1cm, yshift=-0.1cm] {$f$};
        \draw[g,->,>=latex] (l2_g3) -- (l3_g7) node [midway, above, text=black, xshift=+0.1cm, yshift=-0.1cm] {$g$};
        \draw[h,->,>=latex] (l2_g3) to [out=  0,in= 60,looseness=8] (l2_g3) node [right, text=black, xshift= 0.75cm, yshift=0.4cm] {$h$};
        \draw[f,->,>=latex] (l1_g0) -- (l2_g0) node [midway, above, text=black] {$f$};
        \draw[g,->,>=latex] (l1_g0) -- (l2_g1) node [midway, above, text=black] {$g$};
        \draw[h,->,>=latex] (l1_g0) to [out=180,in=120,looseness=8] (l1_g0) node [right, text=black, xshift=-1.25cm, yshift=0.4cm] {$h$};
        \draw[f,->,>=latex] (l1_g1) -- (l2_g2) node [midway, above, text=black] {$f$};
        \draw[g,->,>=latex] (l1_g1) -- (l2_g3) node [midway, above, text=black] {$g$};
        \draw[h,->,>=latex] (l1_g1) to [out=  0,in= 60,looseness=8] (l1_g1) node [right, text=black, xshift= 0.75cm, yshift=0.4cm] {$h$};
        \draw[f,->,>=latex] (l0_g0) -- (l1_g0) node [midway, above, text=black] {$f$};
        \draw[g,->,>=latex] (l0_g0) -- (l1_g1) node [midway, above, text=black] {$g$};
        \draw[h,->,>=latex] (l0_g0) to [out=0,in=60,looseness=8] (l0_g0) node [right, text=black, xshift=0.75cm, yshift=0.4cm] {$h$};
      \end{tikzpicture}
      \end{figure}
      }
    \end{column}
    \begin{column}[T]{6.0cm}
      \pause
      \centering
      \scalebox{.65}{
      \setcounter{algocf}{1}
      \begin{algorithm}[H]
        \caption{Decoder SC Generated.}

        \SetKwProg{Fn}{Function}{}{}

        \Fn{$\DecoderSC()$}
        {
          $f()$ \Comment*[r]{1 -> 2 [4 elmts]}

          $f()$ \Comment*[r]{2 -> 3 [2 elmts]}

          $f()$ \Comment*[r]{3 -> 4 [1 elmts]}

          $g()$ \Comment*[r]{3 -> 5 [1 elmts]}

          $h()$ \Comment*[r]{(4, 5) -> 3 [2 elmts]}

          $g()$ \Comment*[r]{2 -> 6 [2 elmts]}

          $f()$ \Comment*[r]{6 -> 7 [1 elmts]}

          $g()$ \Comment*[r]{6 -> 8 [1 elmts]}

          $\hardDecide()$ \Comment*[r]{8 -> 8 [1 elmts]}

          $h()$ \Comment*[r]{(7, 8) -> 6 [2 elmts]}

          $h()$ \Comment*[r]{(3, 6) -> 2 [4 elmts]}

          $g()$ \Comment*[r]{1 -> 9 [4 elmts]}

          $...$
        }
      \end{algorithm}
      }
    \end{column}
  \end{columns}
  \only<3->{
  \vspace{1cm}
  \begin{columns}
    \begin{column}[T]{6.0cm}
      \begin{itemize}
        \item $+$ High flexibility
        % \begin{itemize}
        %   \item Adapt to many code rate $R = K/N$
        %   \item Adapt to many frozen bits configurations
        % \end{itemize}
        \item $-$ Recursive calls and \textit{if} overhead
      \end{itemize}
    \end{column}
    \begin{column}[T]{6.0cm}
      \only<4->{
      \begin{itemize}
        \item $-$ Reduced flexibility
        % \begin{itemize}
        %   \item Fixed code rate $R = K/N$
        %   \item Fixed frozen bits positions
        % \end{itemize}
        \item $+$ Improved throughput and latency
      \end{itemize}
      }
    \end{column}
  \end{columns}
  }
\end{frame}

\subsection[Source Code Generation]{Source Code Generation}

\begin{frame}{Limitation of the Source Code Generation Strategy}
  \vfill
  \begin{figure}[!h]
    \centering
    \scalebox{.5}{
    \begin{tikzpicture}%[scale=0.50, every node/.style={transform shape}]
    \begin{axis}[/pgfplots/table/ignore chars={ }, %footnotesize,
                 width=1.0\linewidth, height=0.7\linewidth,
                 ymode = log,
                 log basis y={2},
                 xticklabel style={black!70}, yticklabel style={black!70},
                 xlabel=Codeword size ($N$), ylabel=Decoder binary size (KB), grid=both, grid style={gray!30},
                 xmin=6, xmax=16,
                 xticklabels={$2^5$, $2^6$, $2^7$, $2^8$, $2^9$, $2^{10}$, $2^{11}$, $2^{12}$, $2^{13}$, $2^{14}$, $2^{15}$, $2^{16}$},
                 % yticklabels={2,4,8,16,32,64,128,256,512,1024,2048,4096},
                 yticklabels={2,8,32,128,512,2048},
                 grid style={dashed, gray!30},
                 %ymin=-5, ymax=102,
                 % tick align=outside, tickpos=left,
                 %label style={font=\large},
                 % tick label style={font=\large},
                 legend pos=south east, legend columns=1]
        \addplot[mark=o,        Paired-5, semithick    ] table [x index=0, y index=1] {../main/chapter4/fig/polar/sc_gen_l1i_size/dat/samples_generated_decoders_sizes.dat}; \label{plot:lllline1}
        \addplot[mark=square,   Paired-1, semithick    ] table [x index=0, y index=3] {../main/chapter4/fig/polar/sc_gen_l1i_size/dat/samples_generated_decoders_sizes.dat}; \label{plot:lllline2}
        \addplot[mark=none,     Paired-7, dashed, thick] coordinates {(6,32) (16,32)}; \label{plot:lllline3}
    \end{axis}

    \matrix [draw,
             matrix of nodes,
             anchor=north,
             inner sep=2.3pt,
             fill=white,
             column 2/.style={anchor=base east},
             ampersand replacement=\&] at (2.3,6.5)
    {
        \ref{plot:lllline2} \& intra-frame \\
        \ref{plot:lllline1} \& inter-frame \\
        \ref{plot:lllline3} \& L1I size    \\
    };
  \end{tikzpicture}
  }
  \pause
  \end{figure}
  \begin{itemize}
    \item \textbf{Problem:} Codewords$~> N = 2^8$ exceeds the CPU instruction cache size
  %   \pause
  %   \vspace{0.2cm}
  %   \item \textbf{Solution 1:} Do not store the memory addresses offsets in the generated code
  %   \pause
  %   \vspace{0.2cm}
  %   \item \textbf{Solution 2:} Implement a sub-tree folding algorithm to reduce the binary size
  \end{itemize}
  \vfill
\end{frame}

\begin{frame}{Sub-tree Folding Algorithm}
  \begin{figure}[!h]
    \centering
    \scalebox{.4}{
    \begin{tikzpicture}[baseline]
      \tikzset{ any/.style ={draw=gray,     circle, minimum height=0.6cm, text=black, fill=gray!40                                                                              } }
      \tikzset{ frzn/.style={draw=black,    circle, minimum height=0.6cm, text=black                                                                                            } }
      \tikzset{ info/.style={draw=black,    circle, minimum height=0.6cm, text=black, fill=black                                                                                } }
      \tikzset{ r0/.style  ={draw=Paired-1, circle, minimum height=0.6cm, text=black, preaction={fill=Paired-1!40}, pattern=north west lines, pattern color=black!80!Paired-1!70} }
      \tikzset{ r1/.style  ={draw=Paired-3, circle, minimum height=0.6cm, text=black, preaction={fill=Paired-3!40}, pattern=north east lines, pattern color=black!80!Paired-3!70} }
      \tikzset{ rep/.style ={draw=Paired-7, circle, minimum height=0.6cm, text=black, preaction={fill=Paired-7!40}, pattern=crosshatch dots,  pattern color=black!80!Paired-7!70} }
      \tikzset{ spc4/.style={draw=Paired-5, circle, minimum height=0.6cm, text=black, preaction={fill=Paired-5!40}, pattern=horizontal lines, pattern color=black!80!Paired-5!70} }
      \tikzset{ spc/.style ={draw=Paired-9, circle, minimum height=0.6cm, text=black, preaction={fill=Paired-9!40}, pattern=grid,             pattern color=black!80!Paired-9!70} }

      \path[use as bounding box] (-0.5, +1.25) rectangle (14.5, -1.3);

      \node[any,  label={[black,align=center]above:\small{Any}}                              ] (any)  at ( 0.0, 0.0) {};
      \node[frzn, label={[black,align=center]above:\small{Frozen bit}\\\small{(leaf)}}       ] (frz)  at ( 2.0, 0.0) {};
      \node[info, label={[black,align=center]above:\small{Info. bit}\\\small{(leaf)}}        ] (ufrz) at ( 4.0, 0.0) {};
      \node[r0,   label={[black,align=center]above:\small{\texttt{R0}}}                      ] (r0)   at ( 6.0, 0.0) {};
      \node[r1,   label={[black,align=center]above:\small{\texttt{R1}}}                      ] (r1)   at ( 8.0, 0.0) {};
      \node[rep,  label={[black,align=center]above:\small{\texttt{REP}}}                     ] (rep)  at (10.0, 0.0) {};
      \node[spc4, label={[black,align=center]above:\small{\texttt{SPC}$_\text{\texttt{4}}$}} ] (spc4) at (12.0, 0.0) {};
      \node[spc,  label={[black,align=center]above:\small{\texttt{SPC}$_\text{\texttt{4+}}$}}] (spc)  at (14.0, 0.0) {};

      \draw[<-,>=latex        ] (2.5, -1.25) -- ( 5.5, -1.25) node [midway, above, text=black] {\small{Left edge ($f$)}};
      \draw[->,>=latex, dashed] (8.5, -1.25) -- (11.5, -1.25) node [midway, above, text=black] {\small{Right edge ($g$)}};
    \end{tikzpicture}
    }
  \end{figure}

  \begin{columns}
  \begin{column}[T]{5.0cm}
  \begin{figure}[!h]
    \centering
    \scalebox{.4}{
    \begin{tikzpicture}[baseline]
      \tikzset{ any/.style ={draw=gray,     circle, minimum height=0.6cm, text=black, fill=gray!40                                                                              } }
      \tikzset{ frzn/.style={draw=black,    circle, minimum height=0.6cm, text=black                                                                                            } }
      \tikzset{ info/.style={draw=black,    circle, minimum height=0.6cm, text=black, fill=black                                                                                } }
      \tikzset{ r0/.style  ={draw=Paired-1, circle, minimum height=0.6cm, text=black, preaction={fill=Paired-1!40}, pattern=north west lines, pattern color=black!80!Paired-1!70} }
      \tikzset{ r1/.style  ={draw=Paired-3, circle, minimum height=0.6cm, text=black, preaction={fill=Paired-3!40}, pattern=north east lines, pattern color=black!80!Paired-3!70} }
      \tikzset{ rep/.style ={draw=Paired-7, circle, minimum height=0.6cm, text=black, preaction={fill=Paired-7!40}, pattern=crosshatch dots,  pattern color=black!80!Paired-7!70} }
      \tikzset{ spc4/.style={draw=Paired-5, circle, minimum height=0.6cm, text=black, preaction={fill=Paired-5!40}, pattern=horizontal lines, pattern color=black!80!Paired-5!70} }
      \tikzset{ spc/.style ={draw=Paired-9, circle, minimum height=0.6cm, text=black, preaction={fill=Paired-9!40}, pattern=grid,             pattern color=black!80!Paired-9!70} }

      \node[any] (l0_n0)  at ( 6.5, 6.0) {};

      \node[any] (l1_n0)  at ( 4.5, 5.0) {};
      \node[any] (l1_n1)  at ( 8.5, 5.0) {};

      \draw[->,>=latex        ] (l0_n0) -- (l1_n0);
      \draw[->,>=latex, dashed] (l0_n0) -- (l1_n1);

      \node[any] (l2_n0)  at ( 1.5, 4.0) {};
      \node[any] (l2_n1)  at ( 4.5, 4.0) {};
      \node[any] (l2_n2)  at ( 8.5, 4.0) {};
      \node[any] (l2_n3)  at (11.5, 4.0) {};

      \draw[->,>=latex        ] (l1_n0) -- (l2_n0);
      \draw[->,>=latex, dashed] (l1_n0) -- (l2_n1);
      \draw[->,>=latex        ] (l1_n1) -- (l2_n2);
      \draw[->,>=latex, dashed] (l1_n1) -- (l2_n3);

      \node[r0 ] (l3_n0)  at ( 0.5, 3.0) {};
      \node[any] (l3_n1)  at ( 1.5, 3.0) {};
      \node[any] (l3_n2)  at ( 4.0, 3.0) {};
      \node[any] (l3_n3)  at ( 5.0, 3.0) {};
      \node[any] (l3_n4)  at ( 8.0, 3.0) {};
      \node[any] (l3_n5)  at ( 9.0, 3.0) {};
      \node[any] (l3_n6)  at (11.5, 3.0) {};
      \node[r1 ] (l3_n7)  at (12.5, 3.0) {};

      \draw[->,>=latex        ] (l2_n0) -- (l3_n0);
      \draw[->,>=latex, dashed] (l2_n0) -- (l3_n1);
      \draw[->,>=latex        ] (l2_n1) -- (l3_n2);
      \draw[->,>=latex, dashed] (l2_n1) -- (l3_n3);
      \draw[->,>=latex        ] (l2_n2) -- (l3_n4);
      \draw[->,>=latex, dashed] (l2_n2) -- (l3_n5);
      \draw[->,>=latex        ] (l2_n3) -- (l3_n6);
      \draw[->,>=latex, dashed] (l2_n3) -- (l3_n7);

      \node[r0 ] (l4_n0)  at ( 0.5, 2.0) {};
      \node[any] (l4_n1)  at ( 1.5, 2.0) {};
      \node[r0 ] (l4_n2)  at ( 3.0, 2.0) {};
      \node[any] (l4_n3)  at ( 4.0, 2.0) {};
      \node[any] (l4_n4)  at ( 5.0, 2.0) {};
      \node[spc] (l4_n5)  at ( 6.0, 2.0) {};
      \node[rep] (l4_n6)  at ( 7.0, 2.0) {};
      \node[any] (l4_n7)  at ( 8.0, 2.0) {};
      \node[any] (l4_n8)  at ( 9.0, 2.0) {};
      \node[r1 ] (l4_n9)  at (10.0, 2.0) {};
      \node[any] (l4_n10) at (11.5, 2.0) {};
      \node[r1 ] (l4_n11) at (12.5, 2.0) {};

      \draw[->,>=latex        ] (l3_n1) -- (l4_n0);
      \draw[->,>=latex, dashed] (l3_n1) -- (l4_n1);
      \draw[->,>=latex        ] (l3_n2) -- (l4_n2);
      \draw[->,>=latex, dashed] (l3_n2) -- (l4_n3);
      \draw[->,>=latex        ] (l3_n3) -- (l4_n4);
      \draw[->,>=latex, dashed] (l3_n3) -- (l4_n5);
      \draw[->,>=latex        ] (l3_n4) -- (l4_n6);
      \draw[->,>=latex, dashed] (l3_n4) -- (l4_n7);
      \draw[->,>=latex        ] (l3_n5) -- (l4_n8);
      \draw[->,>=latex, dashed] (l3_n5) -- (l4_n9);
      \draw[->,>=latex        ] (l3_n6) -- (l4_n10);
      \draw[->,>=latex, dashed] (l3_n6) -- (l4_n11);

      \node[r0  ] (l5_n0)  at ( 0.5, 1.0) {};
      \node[any ] (l5_n1)  at ( 1.5, 1.0) {};
      \node[rep ] (l5_n2)  at ( 3.0, 1.0) {};
      \node[spc4] (l5_n3)  at ( 4.0, 1.0) {};
      \node[rep ] (l5_n4)  at ( 5.0, 1.0) {};
      \node[spc4] (l5_n5)  at ( 6.0, 1.0) {};
      \node[rep ] (l5_n6)  at ( 7.0, 1.0) {};
      \node[spc4] (l5_n7)  at ( 8.0, 1.0) {};
      \node[rep ] (l5_n8)  at ( 9.0, 1.0) {};
      \node[spc4] (l5_n9)  at (10.0, 1.0) {};
      \node[any ] (l5_n10) at (11.5, 1.0) {};
      \node[r1  ] (l5_n11) at (12.5, 1.0) {};

      \draw[->,>=latex        ] (l4_n1)  -- (l5_n0);
      \draw[->,>=latex, dashed] (l4_n1)  -- (l5_n1);
      \draw[->,>=latex        ] (l4_n3)  -- (l5_n2);
      \draw[->,>=latex, dashed] (l4_n3)  -- (l5_n3);
      \draw[->,>=latex        ] (l4_n4)  -- (l5_n4);
      \draw[->,>=latex, dashed] (l4_n4)  -- (l5_n5);
      \draw[->,>=latex        ] (l4_n7)  -- (l5_n6);
      \draw[->,>=latex, dashed] (l4_n7)  -- (l5_n7);
      \draw[->,>=latex        ] (l4_n8)  -- (l5_n8);
      \draw[->,>=latex, dashed] (l4_n8)  -- (l5_n9);
      \draw[->,>=latex        ] (l4_n10) -- (l5_n10);
      \draw[->,>=latex, dashed] (l4_n10) -- (l5_n11);

      \node[r0] (l6_n0) at ( 1.0, 0.0) {};
      \node[r1] (l6_n1) at ( 2.0, 0.0) {};
      \node[r0] (l6_n2) at (11.0, 0.0) {};
      \node[r1] (l6_n3) at (12.0, 0.0) {};

      \draw[->,>=latex        ] (l5_n1)  -- (l6_n0);
      \draw[->,>=latex, dashed] (l5_n1)  -- (l6_n1);
      \draw[->,>=latex        ] (l5_n10) -- (l6_n2);
      \draw[->,>=latex, dashed] (l5_n10) -- (l6_n3);

      \node[draw=Paired-5, rounded corners=2pt, dashed, fit=(l5_n1) (l6_n0) (l6_n1)] {};
      \node[draw=Paired-5, rounded corners=2pt, dashed, fit=(l5_n10) (l6_n2) (l6_n3)] {};

      \node[draw=Paired-3, rounded corners=2pt, dashed, fit=(l4_n3) (l5_n2) (l5_n3)] {};
      \node[draw=Paired-3, rounded corners=2pt, dashed, fit=(l4_n4) (l5_n4) (l5_n5)] {};
      \node[draw=Paired-3, rounded corners=2pt, dashed, fit=(l4_n7) (l5_n6) (l5_n7)] {};
      \node[draw=Paired-3, rounded corners=2pt, dashed, fit=(l4_n8) (l5_n8) (l5_n9)] {};
    \end{tikzpicture}
    }
  \end{figure}
  \end{column}
  \begin{column}[T]{1.0cm}
    \vspace{2cm}
    \centering
    {\color{black}\Large\MVRightarrow}
  \end{column}
  \begin{column}[T]{5.0cm}
    \begin{figure}[!h]
    \centering
    \scalebox{.4}{
    \begin{tikzpicture}[baseline]
      \tikzset{ any/.style ={draw=gray,     circle, minimum height=0.6cm, text=black, fill=gray!40                                                                              } }
      \tikzset{ frzn/.style={draw=black,    circle, minimum height=0.6cm, text=black                                                                                            } }
      \tikzset{ info/.style={draw=black,    circle, minimum height=0.6cm, text=black, fill=black                                                                                } }
      \tikzset{ r0/.style  ={draw=Paired-1, circle, minimum height=0.6cm, text=black, preaction={fill=Paired-1!40}, pattern=north west lines, pattern color=black!80!Paired-1!70} }
      \tikzset{ r1/.style  ={draw=Paired-3, circle, minimum height=0.6cm, text=black, preaction={fill=Paired-3!40}, pattern=north east lines, pattern color=black!80!Paired-3!70} }
      \tikzset{ rep/.style ={draw=Paired-7, circle, minimum height=0.6cm, text=black, preaction={fill=Paired-7!40}, pattern=crosshatch dots,  pattern color=black!80!Paired-7!70} }
      \tikzset{ spc4/.style={draw=Paired-5, circle, minimum height=0.6cm, text=black, preaction={fill=Paired-5!40}, pattern=horizontal lines, pattern color=black!80!Paired-5!70} }
      \tikzset{ spc/.style ={draw=Paired-9, circle, minimum height=0.6cm, text=black, preaction={fill=Paired-9!40}, pattern=grid,             pattern color=black!80!Paired-9!70} }

      \node[any] (l0_n0)  at ( 6.5, 6.0) {};

      \node[any] (l1_n0)  at ( 4.5, 5.0) {};
      \node[any] (l1_n1)  at ( 8.5, 5.0) {};

      \draw[->,>=latex        ] (l0_n0) -- (l1_n0);
      \draw[->,>=latex, dashed] (l0_n0) -- (l1_n1);

      \node[any] (l2_n0)  at ( 1.5, 4.0) {};
      \node[any] (l2_n1)  at ( 4.5, 4.0) {};
      \node[any] (l2_n2)  at ( 8.5, 4.0) {};
      \node[any] (l2_n3)  at (11.5, 4.0) {};

      \draw[->,>=latex        ] (l1_n0) -- (l2_n0);
      \draw[->,>=latex, dashed] (l1_n0) -- (l2_n1);
      \draw[->,>=latex        ] (l1_n1) -- (l2_n2);
      \draw[->,>=latex, dashed] (l1_n1) -- (l2_n3);

      \node[r0 ] (l3_n0)  at ( 0.5, 3.0) {};
      \node[any] (l3_n1)  at ( 1.5, 3.0) {};
      \node[any] (l3_n2)  at ( 4.0, 3.0) {};
      \node[any] (l3_n3)  at ( 5.0, 3.0) {};
      \node[any] (l3_n4)  at ( 8.0, 3.0) {};
      \node[any] (l3_n5)  at ( 9.0, 3.0) {};
      \node[any] (l3_n6)  at (11.5, 3.0) {};
      \node[r1 ] (l3_n7)  at (12.5, 3.0) {};

      \draw[->,>=latex        ] (l2_n0) -- (l3_n0);
      \draw[->,>=latex, dashed] (l2_n0) -- (l3_n1);
      \draw[->,>=latex        ] (l2_n1) -- (l3_n2);
      \draw[->,>=latex, dashed] (l2_n1) -- (l3_n3);
      \draw[->,>=latex        ] (l2_n2) -- (l3_n4);
      \draw[->,>=latex, dashed] (l2_n2) -- (l3_n5);
      \draw[->,>=latex        ] (l2_n3) -- (l3_n6);
      \draw[->,>=latex, dashed] (l2_n3) -- (l3_n7);

      \node[any] (l4_n0) at ( 1.50, 2.0) {};
      \node[r0 ] (l4_n1) at ( 2.75, 2.0) {};
      \node[spc] (l4_n2) at ( 5.00, 2.0) {};
      \node[any] (l4_n3) at ( 6.50, 2.0) {};
      \node[rep] (l4_n4) at ( 8.00, 2.0) {};
      \node[r1 ] (l4_n5) at (10.25, 2.0) {};
      \node[any] (l4_n6) at (11.50, 2.0) {};

      \draw[->,>=latex, dashed] (l3_n1) -- (l4_n0);
      \draw[->,>=latex        ] (l3_n1) -- (l4_n1);
      \draw[->,>=latex        ] (l3_n2) -- (l4_n1);
      \draw[->,>=latex, dashed] (l3_n2) -- (l4_n3);
      \draw[->,>=latex, dashed] (l3_n3) -- (l4_n2);
      \draw[->,>=latex        ] (l3_n3) -- (l4_n3);
      \draw[->,>=latex, dashed] (l3_n4) -- (l4_n3);
      \draw[->,>=latex        ] (l3_n4) -- (l4_n4);
      \draw[->,>=latex        ] (l3_n5) -- (l4_n3);
      \draw[->,>=latex, dashed] (l3_n5) -- (l4_n5);
      \draw[->,>=latex, dashed] (l3_n6) -- (l4_n5);
      \draw[->,>=latex        ] (l3_n6) -- (l4_n6);

      \node[r0  ] (l5_n0)  at ( 0.50, 1.0) {};
      \node[any ] (l5_n1)  at ( 1.50, 1.0) {};
      \node[rep ] (l5_n2)  at ( 6.00, 1.0) {};
      \node[spc4] (l5_n3)  at ( 7.00, 1.0) {};
      \node[r1  ] (l5_n4)  at (12.50, 1.0) {};

      \draw[->,>=latex        ] (l4_n0) -- (l5_n0);
      \draw[->,>=latex, dashed] (l4_n0) -- (l5_n1);
      \draw[->,>=latex        ] (l4_n3) -- (l5_n2);
      \draw[->,>=latex, dashed] (l4_n3) -- (l5_n3);
      % \draw[->,>=latex        ] (l4_n6) -- (l5_n1);
      \draw[->,>=latex        ] (l4_n6) to[out=230,in=0] (l5_n1);
      \draw[->,>=latex, dashed] (l4_n6) -- (l5_n4);

      \node[r0] (l6_n0) at ( 1.0, 0.0) {};
      \node[r1] (l6_n1) at ( 2.0, 0.0) {};

      \draw[->,>=latex        ] (l5_n1) -- (l6_n0);
      \draw[->,>=latex, dashed] (l5_n1) -- (l6_n1);

      \node[draw=Paired-5, rounded corners=2pt, dashed, fit=(l5_n1) (l6_n0) (l6_n1)] {};
      \node[draw=Paired-3, rounded corners=2pt, dashed, fit=(l4_n3) (l5_n2) (l5_n3)] {};
    \end{tikzpicture}
    }
  \end{figure}
  \end{column}
  \end{columns}
  \vfill
  \begin{itemize}
    \item Find similar sub-patterns and create functions for them
    \item Compromise between the number of functions calls and the binary size
    \item Threshold to avoid functions that contain a too small number of Polar functions
  \end{itemize}
\end{frame}

\begin{frame}{Impact of the Compression Techniques on the Decoder Binary Size}
  % \vspace{-0.5cm}
  \begin{figure}[!h]
    \centering
    \scalebox{.5}{
    \begin{tikzpicture}%[scale=0.50, every node/.style={transform shape}]
    \begin{axis}[/pgfplots/table/ignore chars={ }, %footnotesize,
                 width=1.0\linewidth, height=0.7\linewidth,
                 ymode = log,
                 log basis y={2},
                 xticklabel style={black!70}, yticklabel style={black!70},
                 xlabel=Codeword size ($N$), ylabel=Decoder binary size (KB), grid=both, grid style={gray!30},
                 xmin=6, xmax=16,
                 ymin=0, ymax=4096,
                 xticklabels={$2^5$, $2^6$, $2^7$, $2^8$, $2^9$, $2^{10}$, $2^{11}$, $2^{12}$, $2^{13}$, $2^{14}$, $2^{15}$, $2^{16}$},
                 % yticklabels={2,4,8,16,32,64,128,256,512,1024,2048,4096},
                 yticklabels={0,2,8,32,128,512,2048},
                 grid style={dashed, gray!30},
                 %ymin=-5, ymax=102,
                 % tick align=outside, tickpos=left,
                 %label style={font=\large},
                 % tick label style={font=\large},
                 legend pos=south east, legend columns=1]
        \addplot[mark=o,      Paired-5,  semithick                              ] table [x index=0, y index=1] {../main/chapter4/fig/polar/sc_gen_l1i_size/dat/samples_generated_decoders_sizes.dat}; \label{plot:llllline1}
        \addplot[mark=square, Paired-1,  semithick                              ] table [x index=0, y index=3] {../main/chapter4/fig/polar/sc_gen_l1i_size/dat/samples_generated_decoders_sizes.dat}; \label{plot:llllline2}
        \addplot[mark=none,   draw=none, semithick, dashed                      ] table [x index=0, y index=1] {../main/chapter4/fig/polar/sc_gen_l1i_size/dat/samples_generated_decoders_sizes_after_compression.dat}; \label{plot:llllline6}
        \addplot[mark=none,   draw=none, semithick, dashed                      ] table [x index=0, y index=3] {../main/chapter4/fig/polar/sc_gen_l1i_size/dat/samples_generated_decoders_sizes_after_compression.dat}; \label{plot:llllline7}
        % hack for legend
        \addplot[mark=o,      Paired-5,  semithick, dashed, mark options={solid}] coordinates {(5,32) (17,32)}; \label{plot:llllline3}
        \addplot[mark=square, Paired-1,  semithick, dashed, mark options={solid}] coordinates {(5,32) (17,32)}; \label{plot:llllline4}
        \addplot[mark=none,   Paired-7,             dashed, thick               ] coordinates {(6,32) (16,32)}; \label{plot:llllline5}
        \only<2->{
        \addplot[mark=o,      Paired-5,  semithick, dashed, mark options={solid}] table [x index=0, y index=1] {../main/chapter4/fig/polar/sc_gen_l1i_size/dat/samples_generated_decoders_sizes_after_compression.dat};
        \addplot[mark=square, Paired-1,  semithick, dashed, mark options={solid}] table [x index=0, y index=3] {../main/chapter4/fig/polar/sc_gen_l1i_size/dat/samples_generated_decoders_sizes_after_compression.dat};
        }
        \only<3->{
        \node[anchor=west] (source) at (axis cs:11.8,2.5){\textbullet\ Enable sub-tree folding};
        \draw[->,thick] (axis cs:12, 3) -- (axis cs:12, 6.5);
        }
    \end{axis}
    \only<1>{
    \matrix [draw,
             matrix of nodes,
             anchor=north,
             inner sep=2.3pt,
             fill=white,
             column 1/.style={anchor=base west},
             column 2/.style={anchor=base west},
             column 3/.style={anchor=base west},
             ampersand replacement=\&] at (2.3,6.5)
    {
                    \& w/o c.               \& w. c. \\
        intra-frame \& \ref{plot:llllline2} \&       \\
        inter-frame \& \ref{plot:llllline1} \&       \\
        L1I size    \& \ref{plot:llllline5} \&       \\
    };
    }
    \only<2->{
    \matrix [draw,
             matrix of nodes,
             anchor=north,
             inner sep=2.3pt,
             fill=white,
             column 1/.style={anchor=base west},
             column 2/.style={anchor=base west},
             column 3/.style={anchor=base west},
             ampersand replacement=\&] at (2.3,6.5)
    {
                    \& w/o c.               \& w. c.                \\
        intra-frame \& \ref{plot:llllline2} \& \ref{plot:llllline4} \\
        inter-frame \& \ref{plot:llllline1} \& \ref{plot:llllline3} \\
        L1I size    \& \ref{plot:llllline5} \&                      \\
    };
    }
    \end{tikzpicture}
    }
  \end{figure}
  \vfill
  \begin{itemize}
    \item \textbf{Problem:} Codewords$~> N = 2^8$ exceeds the CPU instruction cache size
    \vspace{0.2cm}
    \item<2-> \textbf{Solution 1:} Do not store the memory addresses offsets in the generated code
    \vspace{0.2cm}
    \item<3-> \textbf{Solution 2:} Implement a sub-tree folding algorithm to reduce the binary size
  \end{itemize}
\end{frame}

\subsection[Outcome]{Outcome}

\begin{frame}{High Performance Implementations}
  \vfill
  \begin{enumerate}
    \item High performance implementation of the \textbf{dynamic SC decoder}
    \begin{itemize}
      \item Fine tuned with template meta-programming techniques
      \item Focus on \textbf{maximizing the flexibility}
    \end{itemize}
    \pause
    \vspace{0.3cm}
    \item Polar ECC Decoder \textbf{Generation Environment}
    \begin{itemize}
      \item Source code generation strategy
      \item Focus on achieving \textbf{highest possible performances}
    \end{itemize}
  \end{enumerate}
  \vfill
  \pause

  \vspace*{.5em}
  ~~~~~{\color{bleuUni}\Large\MVRightarrow} Support intra-/inter-frame SIMDization %thanks to the Polar API

  \vspace*{.5em}
  ~~~~~{\color{bleuUni}\Large\MVRightarrow} Compatible with tree pruning techniques

  \vfill
\end{frame}

% \begin{frame}{Polar Decoder: Successive Cancellation List Algorithm}
%   \begin{itemize}
%     \item Based on the Successive Cancellation decoder
%     \item Maintains a finite list of $L$ possible decoding trees
%     \item Chooses the most likely tree according to a metric
%     \item Decoding performance are better for short to medium size codewords
%     \item Can be combined with a CRC code to discriminate the metric
%       \\\vspace*{.5em}
%       {\color{bleuUni}\Large\MVRightarrow} Decoding performance is, again, improved
%   \end{itemize}
% \end{frame}

% \subsection[Evaluations and Comparisons]{Evaluations and Comparisons}

\begin{frame}[fragile]{Experimentation Protocol}

  \begin{itemize}
    \item Compiler
    \begin{itemize}
      \item GCC version 4.8
      \item Optimization flags: \verb|-O3 -funroll-loops|
    \end{itemize}
    \vspace{0.2cm}
    \item Focus on the \textbf{SC generated decoders}
    \begin{itemize}
      \item Intra-frame SIMD: 32-bit floating-point representation + \verb|AVX| instructions
      \item Inter-frame SIMD: 8-bit fixed-point representation + \verb|SSE4.2| instructions
    \end{itemize}
    \vspace{0.2cm}
    \item \textbf{Single core performance}
  \end{itemize}
  \vfill
  \begin{table}[htp]
    \centering
    \begin{tabular}{c | c }
    \multirow{1}{*}{\textbf{CPU}} & Intel\R Xeon\TM E3-1225 \\
    \hline
    \textbf{Cores/Freq.}          & 4 cores, 3.1-3.4 Ghz    \\
    \textbf{Arch.}                & \emph{Sandy Bridge}     \\
    \textbf{Process}              & 32 nm                   \\
    \multirow{1}{*}{\textbf{LLC}} & L3 6 MB                 \\
    \end{tabular}
  \end{table}

\end{frame}

\begin{frame}{Impact of the Tree Pruning on the Decoding Speed}
  \begin{columns}
  \begin{column}[T]{6cm}
    \begin{figure}[!h]
    \centering
    \scalebox{.5}{
    \begin{tikzpicture}[every axis/.style={
                        /pgfplots/table/ignore chars={|}, %footnotesize,
                        width=2.0\linewidth, height=1.40\linewidth,
                        xticklabel style={black!70}, yticklabel style={black!70},
                        % tick align=outside, tickpos=left,
                        ybar stacked, bar width=14pt,
                        legend style={at={(0.39,0.95)}, anchor=north}, legend columns=-1,
                        ylabel={Coded throughput (Mb/s)}, xlabel={Code rate ($R = K / N$)},
                        ymajorgrids, grid style={gray!30}, grid style={dashed, gray!50}, xmin=0.05, xmax=0.45, xtick=data, ymin=0, ymax=550.0,
                        x tick label style={rotate=-45}, xticklabels={1/5, 1/2, 5/6, 9/10}}]

      \begin{axis}[]
        \addplot+[draw=black,    fill=black!50                                                                                 ] table [x=R, y expr=\thisrowno{4}              ] {../main/chapter2/fig/polar/sc_tree_cut/dat/E31225_samples_intra_32b_opti_spc4+.dat};
        \only<2->{
        \addplot+[draw=Paired-1, fill=Paired-1!40, postaction={pattern color = black!80!Paired-1!70, pattern=north west  lines}] table [x=R, y expr=\thisrowno{5}-\thisrowno{4}] {../main/chapter2/fig/polar/sc_tree_cut/dat/E31225_samples_intra_32b_opti_spc4+.dat};
        }
        \only<3->{
        \addplot+[draw=Paired-3, fill=Paired-3!40, postaction={pattern color = black!80!Paired-3!70, pattern=north east  lines}] table [x=R, y expr=\thisrowno{6}-\thisrowno{5}] {../main/chapter2/fig/polar/sc_tree_cut/dat/E31225_samples_intra_32b_opti_spc4+.dat};
        }
        \only<4->{
        \addplot+[draw=Paired-7, fill=Paired-7!40, postaction={pattern color = black!80!Paired-7!70, pattern=crosshatch  dots} ] table [x=R, y expr=\thisrowno{7}-\thisrowno{6}] {../main/chapter2/fig/polar/sc_tree_cut/dat/E31225_samples_intra_32b_opti_spc4+.dat};
        }
        \only<6->{
        \addplot+[draw=Paired-5, fill=Paired-5!40, postaction={pattern color = black!80!Paired-5!70, pattern=horizontal lines} ] table [x=R, y expr=\thisrowno{8}-\thisrowno{7}] {../main/chapter2/fig/polar/sc_tree_cut/dat/E31225_samples_intra_32b_opti_spc4+.dat};
        }
        \only<5->{
        \addplot+[draw=Paired-9, fill=Paired-9!40, postaction={pattern color = black!80!Paired-9!70, pattern=grid}             ] table [x=R, y expr=\thisrowno{9}-\thisrowno{8}] {../main/chapter2/fig/polar/sc_tree_cut/dat/E31225_samples_intra_32b_opti_spc4+.dat};
        }
        \only<1-5>{
        \legend{\texttt{ref}\text{ }, \texttt{R0}\text{ }, \texttt{R1}\text{ }, \texttt{REP}\text{ }, $\texttt{SPC}_\text{\texttt{4+}}$}
        }
        \only<6>{
        \legend{\texttt{ref}\text{ }, \texttt{R0}\text{ }, \texttt{R1}\text{ }, \texttt{REP}\text{ }, $\texttt{SPC}_\text{\texttt{4}}$\text{ } , $\texttt{SPC}_\text{\texttt{4+}}$}
        }
      \end{axis}

      \only<6->{
      \begin{axis}[bar shift=-20pt, hide axis]
        \addplot+[draw=black,    fill=black!50                                                                                 ] table [x=R, y expr=\thisrowno{4}              ] {../main/chapter2/fig/polar/sc_tree_cut/dat/E31225_samples_intra_32b_opti_spc4.dat};
        \addplot+[draw=Paired-1, fill=Paired-1!40, postaction={pattern color = black!80!Paired-1!70, pattern=north west  lines}] table [x=R, y expr=\thisrowno{5}-\thisrowno{4}] {../main/chapter2/fig/polar/sc_tree_cut/dat/E31225_samples_intra_32b_opti_spc4.dat};
        \addplot+[draw=Paired-3, fill=Paired-3!40, postaction={pattern color = black!80!Paired-3!70, pattern=north east  lines}] table [x=R, y expr=\thisrowno{6}-\thisrowno{5}] {../main/chapter2/fig/polar/sc_tree_cut/dat/E31225_samples_intra_32b_opti_spc4.dat};
        \addplot+[draw=Paired-7, fill=Paired-7!40, postaction={pattern color = black!80!Paired-7!70, pattern=crosshatch  dots} ] table [x=R, y expr=\thisrowno{7}-\thisrowno{6}] {../main/chapter2/fig/polar/sc_tree_cut/dat/E31225_samples_intra_32b_opti_spc4.dat};
        \addplot+[draw=Paired-5, fill=Paired-5!40, postaction={pattern color = black!80!Paired-5!70, pattern=horizontal lines} ] table [x=R, y expr=\thisrowno{8}-\thisrowno{7}] {../main/chapter2/fig/polar/sc_tree_cut/dat/E31225_samples_intra_32b_opti_spc4.dat};
      \end{axis}
      }

      \only<7->{
      \begin{axis}[bar shift=+20pt,hide axis]
        \addplot+[draw=black,    fill=black!50                                                                                 ] table [x=R, y expr=\thisrowno{4}               ] {../main/chapter2/fig/polar/sc_tree_cut/dat/E31225_samples_intra_32b_opti_spc16-.dat};
        \addplot+[draw=Paired-1, fill=Paired-1!40, postaction={pattern color = black!80!Paired-1!70, pattern=north west  lines}] table [x=R, y expr=\thisrowno{5}-\thisrowno{4} ] {../main/chapter2/fig/polar/sc_tree_cut/dat/E31225_samples_intra_32b_opti_spc16-.dat};
        \addplot+[draw=Paired-3, fill=Paired-3!40, postaction={pattern color = black!80!Paired-3!70, pattern=north east  lines}] table [x=R, y expr=\thisrowno{6}-\thisrowno{5} ] {../main/chapter2/fig/polar/sc_tree_cut/dat/E31225_samples_intra_32b_opti_spc16-.dat};
        \addplot+[draw=Paired-7, fill=Paired-7!40, postaction={pattern color = black!80!Paired-7!70, pattern=crosshatch  dots} ] table [x=R, y expr=\thisrowno{7}-\thisrowno{6} ] {../main/chapter2/fig/polar/sc_tree_cut/dat/E31225_samples_intra_32b_opti_spc16-.dat};
        \addplot+[draw=Paired-5, fill=Paired-5!40, postaction={pattern color = black!80!Paired-5!70, pattern=horizontal lines} ] table [x=R, y expr=\thisrowno{8}-\thisrowno{7} ] {../main/chapter2/fig/polar/sc_tree_cut/dat/E31225_samples_intra_32b_opti_spc16-.dat};
        \addplot+[draw=Paired-9, fill=Paired-9!40, postaction={pattern color = black!80!Paired-9!70, pattern=grid}             ] table [x=R, y expr=\thisrowno{9}-\thisrowno{8} ] {../main/chapter2/fig/polar/sc_tree_cut/dat/E31225_samples_intra_32b_opti_spc16-.dat};
        \addplot+[draw=Paired-9, fill=Paired-9!10, postaction={pattern color = black!80!Paired-9!70, pattern=grid}             ] table [x=R, y expr=\thisrowno{10}-\thisrowno{9}] {../main/chapter2/fig/polar/sc_tree_cut/dat/E31225_samples_intra_32b_opti_spc16-.dat};
        \legend{\texttt{ref}\text{ }, \texttt{R0}\text{ }, \texttt{R1}\text{ }, \texttt{REP}\text{ }, $\texttt{SPC}_\text{\texttt{4}}$\text{ } , $\texttt{SPC}_\text{\texttt{4+}}$ , $\texttt{SPC}_\text{\texttt{16-}}$}
      \end{axis}
      }
    \end{tikzpicture}
    }
    \caption*{Intra-frame SIMD}
    \end{figure}
  \end{column}
  \begin{column}[T]{6cm}
    \begin{figure}[!h]
    \centering
    \scalebox{.5}{
    \begin{tikzpicture}[every axis/.style={
                        /pgfplots/table/ignore chars={|}, %footnotesize,
                        width=2.0\linewidth, height=1.40\linewidth,
                        xticklabel style={black!70}, yticklabel style={black!70},
                        % tick align=outside, tickpos=left,
                        ybar stacked, bar width=14pt,
                        legend style={at={(0.39,0.95)}, anchor=north}, legend columns=-1,
                        ylabel={Coded throughput (Mb/s)}, xlabel={Code rate ($R = K / N$)},
                        ymajorgrids, grid style={gray!30}, grid style={dashed, gray!50}, xmin=0.05, xmax=0.45, xtick=data, ymin=0, ymax=1900.0,
                        x tick label style={rotate=-45}, xticklabels={1/5, 1/2, 5/6, 9/10}}]

      \begin{axis}
        \addplot+[draw=black,    fill=black!50                                                                                 ] table [x=R, y expr=\thisrowno{4}              ] {../main/chapter2/fig/polar/sc_tree_cut/dat/E31225_samples_inter_8b_opti_spc4+.dat};
        \only<2->{
        \addplot+[draw=Paired-1, fill=Paired-1!40, postaction={pattern color = black!80!Paired-1!70, pattern=north west  lines}] table [x=R, y expr=\thisrowno{5}-\thisrowno{4}] {../main/chapter2/fig/polar/sc_tree_cut/dat/E31225_samples_inter_8b_opti_spc4+.dat};
        }
        \only<3->{
        \addplot+[draw=Paired-3, fill=Paired-3!40, postaction={pattern color = black!80!Paired-3!70, pattern=north east  lines}] table [x=R, y expr=\thisrowno{6}-\thisrowno{5}] {../main/chapter2/fig/polar/sc_tree_cut/dat/E31225_samples_inter_8b_opti_spc4+.dat};
        }
        \only<4->{
        \addplot+[draw=Paired-7, fill=Paired-7!40, postaction={pattern color = black!80!Paired-7!70, pattern=crosshatch  dots} ] table [x=R, y expr=\thisrowno{7}-\thisrowno{6}] {../main/chapter2/fig/polar/sc_tree_cut/dat/E31225_samples_inter_8b_opti_spc4+.dat};
        }
        \only<5->{
        \addplot+[draw=Paired-5, fill=Paired-5!40, postaction={pattern color = black!80!Paired-5!70, pattern=horizontal lines} ] table [x=R, y expr=\thisrowno{8}-\thisrowno{7}] {../main/chapter2/fig/polar/sc_tree_cut/dat/E31225_samples_inter_8b_opti_spc4+.dat};
        \addplot+[draw=Paired-9, fill=Paired-9!40, postaction={pattern color = black!80!Paired-9!70, pattern=grid}             ] table [x=R, y expr=\thisrowno{9}-\thisrowno{8}] {../main/chapter2/fig/polar/sc_tree_cut/dat/E31225_samples_inter_8b_opti_spc4+.dat};
        }
      \end{axis}

      \only<6->{
      \begin{axis}[bar shift=-20pt, hide axis]
        \addplot+[draw=black,    fill=black!50                                                                                 ] table [x=R, y expr=\thisrowno{4}              ] {../main/chapter2/fig/polar/sc_tree_cut/dat/E31225_samples_inter_8b_opti_spc4.dat};
        \addplot+[draw=Paired-1, fill=Paired-1!40, postaction={pattern color = black!80!Paired-1!70, pattern=north west  lines}] table [x=R, y expr=\thisrowno{5}-\thisrowno{4}] {../main/chapter2/fig/polar/sc_tree_cut/dat/E31225_samples_inter_8b_opti_spc4.dat};
        \addplot+[draw=Paired-3, fill=Paired-3!40, postaction={pattern color = black!80!Paired-3!70, pattern=north east  lines}] table [x=R, y expr=\thisrowno{6}-\thisrowno{5}] {../main/chapter2/fig/polar/sc_tree_cut/dat/E31225_samples_inter_8b_opti_spc4.dat};
        \addplot+[draw=Paired-7, fill=Paired-7!40, postaction={pattern color = black!80!Paired-7!70, pattern=crosshatch  dots} ] table [x=R, y expr=\thisrowno{7}-\thisrowno{6}] {../main/chapter2/fig/polar/sc_tree_cut/dat/E31225_samples_inter_8b_opti_spc4.dat};
        \addplot+[draw=Paired-5, fill=Paired-5!40, postaction={pattern color = black!80!Paired-5!70, pattern=horizontal lines} ] table [x=R, y expr=\thisrowno{8}-\thisrowno{7}] {../main/chapter2/fig/polar/sc_tree_cut/dat/E31225_samples_inter_8b_opti_spc4.dat};
      \end{axis}
      }

      \only<7->{
      \begin{axis}[bar shift=+20pt,hide axis]
        \addplot+[draw=black,    fill=black!50                                                                                 ] table [x=R, y expr=\thisrowno{4}               ] {../main/chapter2/fig/polar/sc_tree_cut/dat/E31225_samples_inter_8b_opti_spc16-.dat};
        \addplot+[draw=Paired-1, fill=Paired-1!40, postaction={pattern color = black!80!Paired-1!70, pattern=north west  lines}] table [x=R, y expr=\thisrowno{5}-\thisrowno{4} ] {../main/chapter2/fig/polar/sc_tree_cut/dat/E31225_samples_inter_8b_opti_spc16-.dat};
        \addplot+[draw=Paired-3, fill=Paired-3!40, postaction={pattern color = black!80!Paired-3!70, pattern=north east  lines}] table [x=R, y expr=\thisrowno{6}-\thisrowno{5} ] {../main/chapter2/fig/polar/sc_tree_cut/dat/E31225_samples_inter_8b_opti_spc16-.dat};
        \addplot+[draw=Paired-7, fill=Paired-7!40, postaction={pattern color = black!80!Paired-7!70, pattern=crosshatch  dots} ] table [x=R, y expr=\thisrowno{7}-\thisrowno{6} ] {../main/chapter2/fig/polar/sc_tree_cut/dat/E31225_samples_inter_8b_opti_spc16-.dat};
        \addplot+[draw=Paired-5, fill=Paired-5!40, postaction={pattern color = black!80!Paired-5!70, pattern=horizontal lines} ] table [x=R, y expr=\thisrowno{8}-\thisrowno{7} ] {../main/chapter2/fig/polar/sc_tree_cut/dat/E31225_samples_inter_8b_opti_spc16-.dat};
        \addplot+[draw=Paired-9, fill=Paired-9!40, postaction={pattern color = black!80!Paired-9!70, pattern=grid}             ] table [x=R, y expr=\thisrowno{9}-\thisrowno{8} ] {../main/chapter2/fig/polar/sc_tree_cut/dat/E31225_samples_inter_8b_opti_spc16-.dat};
        \addplot+[draw=Paired-9, fill=Paired-9!10, postaction={pattern color = black!80!Paired-9!70, pattern=grid}             ] table [x=R, y expr=\thisrowno{10}-\thisrowno{9}] {../main/chapter2/fig/polar/sc_tree_cut/dat/E31225_samples_inter_8b_opti_spc16-.dat};
      \end{axis}
      }
    \end{tikzpicture}
    }
    \caption*{Inter-frame SIMD}
    \end{figure}
  \end{column}
  \end{columns}
  \vfill
  \begin{itemize}
    \item Frame size $N = 2048$, compression techniques off
    \item<7-> High level of flexibility: tree pruning rules can be switched on/off
  \end{itemize}
\end{frame}

\begin{frame}{Latency and Throughput Performance Comparisons}
  \begin{columns}
  \begin{column}[T]{6cm}
    \begin{figure}[!h]
    \centering
    \scalebox{.5}{
    \begin{tikzpicture}
      \begin{axis}[/pgfplots/table/ignore chars={ }, %footnotesize,
                   width=2.0\linewidth, height=1.4\linewidth,
                   xticklabel style={black!70}, yticklabel style={black!70},
                   xmode = log,
                   ymode = log,
                   log basis x={2},
                   xlabel=Codeword size ($N$), ylabel=Latency ($\mu$s), grid=both, grid style={gray!30},
                   xmin=4, xmax=65536,
                   % tick align=outside, tickpos=left,
                   %label style={font=\large},
                   % tick label style={font=\large},
                   grid style={dashed, gray!30},
                   legend pos=south east, legend columns=1]
          \addplot[mark=o,      Paired-1,    semithick                              ] table [x index=0, y index=15] {../main/chapter4/fig/polar/sc_gen_thr_intra/dat/E31225_samples_intra_32b_aff3ct_r5_6.dat}; \label{plot:intra_line1}
          \addplot[mark=o,      Paired-1!70, semithick, dashed, mark options={solid}] table [x index=0, y index=15] {../main/chapter4/fig/polar/sc_gen_thr_intra/dat/E31225_samples_intra_32b_aff3ct_r1_2.dat}; \label{plot:intra_line2}
          \addplot[mark=square, Paired-5,    semithick                              ] table [x index=0, y index=15] {../main/chapter4/fig/polar/sc_gen_thr_inter/dat/E31225_samples_inter_8b_aff3ct_r5_6.dat}; \label{plot:inter_line1}
          \addplot[mark=square, Paired-5!70, semithick, dashed, mark options={solid}] table [x index=0, y index=15] {../main/chapter4/fig/polar/sc_gen_thr_inter/dat/E31225_samples_inter_8b_aff3ct_r1_2.dat}; \label{plot:inter_line2}
      \end{axis}

      \matrix [draw,
               matrix of nodes,
               anchor=north,
               inner sep=2.3pt,
               fill=white,
               column 1/.style={anchor=base east},
               ampersand replacement=\&] at (2.7,6.5)
      {
                      \& $R = 5/6$              \& $R = 1/2$              \\
          intra-frame \& \ref{plot:intra_line1} \& \ref{plot:intra_line2} \\
          inter-frame \& \ref{plot:inter_line1} \& \ref{plot:inter_line2} \\
      };
    \end{tikzpicture}
    }
    \caption*{\footnotesize{~~Latency performance (lower is better)}}
    \end{figure}
  \end{column}
  \begin{column}[T]{6cm}
    \begin{figure}[!h]
    \centering
    \scalebox{.5}{
    \begin{tikzpicture}
      \begin{axis}[/pgfplots/table/ignore chars={ }, %footnotesize,
                   width=2.0\linewidth, height=1.4\linewidth,
                   xticklabel style={black!70}, yticklabel style={black!70},
                   xmode = log,
                   log basis x={2},
                   xlabel=Codeword size ($N$), ylabel=Coded throughput (Mb/s), grid=both, grid style={gray!30},
                   xmin=4, xmax=65536,
                   % tick align=outside, tickpos=left,
                   %label style={font=\large},
                   % tick label style={font=\large},
                   grid style={dashed, gray!30},
                   legend pos=south east, legend columns=1]
          \addplot[mark=o,      Paired-1,    semithick                              ] table [x index=0, y index=13] {../main/chapter4/fig/polar/sc_gen_thr_intra/dat/E31225_samples_intra_32b_aff3ct_r5_6.dat}; \label{plot:intra_line1}
          \addplot[mark=o,      Paired-1!70, semithick, dashed, mark options={solid}] table [x index=0, y index=13] {../main/chapter4/fig/polar/sc_gen_thr_intra/dat/E31225_samples_intra_32b_aff3ct_r1_2.dat}; \label{plot:intra_line2}
          \addplot[mark=square, Paired-5,    semithick                              ] table [x index=0, y index=13] {../main/chapter4/fig/polar/sc_gen_thr_inter/dat/E31225_samples_inter_8b_aff3ct_r5_6.dat}; \label{plot:inter_line1}
          \addplot[mark=square, Paired-5!70, semithick, dashed, mark options={solid}] table [x index=0, y index=13] {../main/chapter4/fig/polar/sc_gen_thr_inter/dat/E31225_samples_inter_8b_aff3ct_r1_2.dat}; \label{plot:inter_line2}

      \end{axis}
      % \matrix [draw,
      %          matrix of nodes,
      %          anchor=north,
      %          inner sep=2.3pt,
      %          fill=white,
      %          column 1/.style={anchor=base east},
      %          ampersand replacement=\&] at (4.5,4.0)
      % {
      %                 \& $R = 5/6$              \& $R = 1/2$              \\
      %     intra-frame \& \ref{plot:intra_line1} \& \ref{plot:intra_line2} \\
      %     inter-frame \& \ref{plot:inter_line1} \& \ref{plot:inter_line2} \\
      % };
    \end{tikzpicture}
    }
    \caption*{\footnotesize{~~Throughput performance (higher is better)}}
    \end{figure}
  \end{column}
  \end{columns}
  \vfill
  \begin{itemize}
    \item Intra-frame SIMD strategy: \textbf{Lowest latencies}
    \item Inter-frame SIMD strategy: \textbf{Highest throughputs}
  \end{itemize}
\end{frame}

\begin{frame}{Comparisons with State-of-the-Art Implementations}
  \vspace{-0.5cm}
  \begin{columns}
  \begin{column}[T]{6cm}
  \begin{figure}[!h]
    \centering
    \scalebox{.5}{
    \begin{tikzpicture}
      \begin{axis}[/pgfplots/table/ignore chars={ }, %footnotesize,
                   width=2.0\linewidth, height=1.4\linewidth,
                   xticklabel style={black!70}, yticklabel style={black!70},
                   xmode = log,
                   log basis x={2},
                   xlabel=Codeword size ($N$), ylabel=Coded throughput (Mb/s), grid=both, grid style={gray!30},
                   xmin=4, xmax=65536,
                   % tick align=outside, tickpos=left,
                   %label style={font=\large},
                   % tick label style={font=\large},
                   grid style={dashed, gray!30},
                   legend pos=south east, legend columns=1]
          \addplot[mark=o,    Paired-1,    semithick                                   ] table [x index=0, y index=13] {../main/chapter4/fig/polar/sc_gen_thr_intra/dat/E31225_samples_intra_32b_aff3ct_r5_6.dat}; \label{plot:iintra_line1}
          \addplot[mark=o,    Paired-1!70, semithick, dashed, mark options={solid}     ] table [x index=0, y index=13] {../main/chapter4/fig/polar/sc_gen_thr_intra/dat/E31225_samples_intra_32b_aff3ct_r1_2.dat}; \label{plot:iintra_line3}
          \addplot[mark=x,    Paired-3,    semithick, thick,  only marks, mark size=4.0] table [x index=0, y index=13] {../main/chapter4/fig/polar/sc_gen_thr_intra/dat/E31225_samples_intra_32b_sarkis_r5_6.dat}; \label{plot:iintra_line2}
          \addplot[mark=star, Paired-3!70, semithick, thick,  only marks, mark size=4.0] table [x index=0, y index=13] {../main/chapter4/fig/polar/sc_gen_thr_intra/dat/E31225_samples_intra_32b_sarkis_r1_2.dat}; \label{plot:iintra_line4}
      \end{axis}

      \matrix [draw,
               matrix of nodes,
               anchor=north,
               inner sep=2.3pt,
               fill=white,
               column 1/.style={anchor=base east},
               ampersand replacement=\&] at (7.85,2.0)
      {
                            \& $R = 5/6$               \& $R = 1/2$               \\
          this work         \& \ref{plot:iintra_line1} \& \ref{plot:iintra_line3} \\
          \cite{Sarkis2014} \& \ref{plot:iintra_line2} \& \ref{plot:iintra_line4} \\
      };
    \end{tikzpicture}
    }
  \caption*{\footnotesize{~~Intra-frame SIMD}}
  \end{figure}
  \end{column}
  \begin{column}[T]{6cm}
  \begin{figure}[!h]
    \centering
    \scalebox{.5}{
    \begin{tikzpicture}
      \begin{axis}[/pgfplots/table/ignore chars={ }, %footnotesize,
                   width=2.0\linewidth, height=1.4\linewidth,
                   xticklabel style={black!70}, yticklabel style={black!70},
                   xmode = log,
                   log basis x={2},
                   xlabel=Codeword size ($N$), ylabel=Coded throughput (Mb/s), grid=both, grid style={gray!30},
                   xmin=4, xmax=65536,
                   % tick align=outside, tickpos=left,
                   %label style={font=\large},
                   % tick label style={font=\large},
                   grid style={dashed, gray!30},
                   legend pos=south east, legend columns=1]
          \addplot[mark=square,   Paired-5,    semithick                              ] table [x index=0, y index=13] {../main/chapter4/fig/polar/sc_gen_thr_inter/dat/E31225_samples_inter_8b_aff3ct_r5_6.dat}; \label{plot:iinter_line1}
          \addplot[mark=triangle, Paired-3,    semithick                              ] table [x index=0, y index=13] {../main/chapter4/fig/polar/sc_gen_thr_inter/dat/E31225_samples_inter_8b_handw_r5_6.dat }; \label{plot:iinter_line2}
          \addplot[mark=square,   Paired-5!70, semithick, dashed, mark options={solid}] table [x index=0, y index=13] {../main/chapter4/fig/polar/sc_gen_thr_inter/dat/E31225_samples_inter_8b_aff3ct_r1_2.dat}; \label{plot:iinter_line3}
          \addplot[mark=triangle, Paired-3!70, semithick, dashed, mark options={solid}] table [x index=0, y index=13] {../main/chapter4/fig/polar/sc_gen_thr_inter/dat/E31225_samples_inter_8b_handw_r1_2.dat }; \label{plot:iinter_line4}
      \end{axis}

      \matrix [draw,
               matrix of nodes,
               anchor=north,
               inner sep=2.3pt,
               fill=white,
               column 1/.style={anchor=base east},
               ampersand replacement=\&] at (4.3,2.0)
      {
                            \& $R = 5/6$               \& $R = 1/2$               \\
          this work         \& \ref{plot:iinter_line1} \& \ref{plot:iinter_line3} \\
          \cite{LeGal2015a} \& \ref{plot:iinter_line2} \& \ref{plot:iinter_line4} \\
      };
    \end{tikzpicture}
    }
  \caption*{\footnotesize{~~Inter-frame SIMD}}
  \end{figure}
  \end{column}
  \end{columns}
  \vfill
  \enumcite{Sarkis2014}

  \enumcite{LeGal2015a}
\end{frame}

% \begin{frame}{Latency Performance Comparisons, Lower is Better}
%   \vspace{-0.5cm}
%   \begin{columns}
%   \begin{column}[T]{6cm}
%   \begin{figure}[!h]
%     \centering
%     \scalebox{.5}{
%     \begin{tikzpicture}
%       \begin{axis}[/pgfplots/table/ignore chars={ }, %footnotesize,
%                    width=2.0\linewidth, height=1.4\linewidth,
%                    xticklabel style={black!70}, yticklabel style={black!70},
%                    xmode = log,
%                    ymode = log,
%                    log basis x={2},
%                    xlabel=Codeword size ($N$), ylabel=Latency ($\mu$s), grid=both, grid style={gray!30},
%                    xmin=256, xmax=32768, ymin=0.4, ymax=800,
%                    % tick align=outside, tickpos=left,
%                    %label style={font=\large},
%                    % tick label style={font=\large},
%                    grid style={dashed, gray!30},
%                    legend pos=south east, legend columns=1]
%           \addplot[mark=square, Paired-1,    semithick                                   ] table [x index=0, y index=15] {../main/chapter4/fig/polar/sc_gen_thr_intra/dat/E31225_samples_intra_32b_aff3ct_r5_6.dat}; \label{plot:intra_line1}
%           \addplot[mark=square, Paired-1!70, semithick, dashed, mark options={solid}     ] table [x index=0, y index=15] {../main/chapter4/fig/polar/sc_gen_thr_intra/dat/E31225_samples_intra_32b_aff3ct_r1_2.dat}; \label{plot:intra_line3}
%           \addplot[mark=x,      Paired-3,    semithick, thick,  only marks, mark size=4.0] table [x index=0, y index=15] {../main/chapter4/fig/polar/sc_gen_thr_intra/dat/E31225_samples_intra_32b_sarkis_r5_6.dat}; \label{plot:intra_line2}
%           \addplot[mark=star,   Paired-3!70, semithick, thick,  only marks, mark size=4.0] table [x index=0, y index=15] {../main/chapter4/fig/polar/sc_gen_thr_intra/dat/E31225_samples_intra_32b_sarkis_r1_2.dat}; \label{plot:intra_line4}
%       \end{axis}

%       \matrix [draw,
%                matrix of nodes,
%                anchor=north,
%                inner sep=2.3pt,
%                fill=white,
%                column 1/.style={anchor=base east},
%                ampersand replacement=\&] at (7.85,2.0)
%       {
%                             \& $R = 5/6$              \& $R = 1/2$              \\
%           this work         \& \ref{plot:intra_line1} \& \ref{plot:intra_line3} \\
%           \cite{Sarkis2014} \& \ref{plot:intra_line2} \& \ref{plot:intra_line4} \\
%       };
%     \end{tikzpicture}
%     }
%   \caption*{Intra-frame SIMD}
%   \end{figure}
%   \end{column}
%   \begin{column}[T]{6cm}
%   \begin{figure}[!h]
%     \centering
%     \scalebox{.5}{
%     \begin{tikzpicture}
%       \begin{axis}[/pgfplots/table/ignore chars={ }, %footnotesize,
%                    width=2.0\linewidth, height=1.4\linewidth,
%                    xticklabel style={black!70}, yticklabel style={black!70},
%                    xmode = log,
%                    ymode = log,
%                    log basis x={2},
%                    xlabel=Codeword size ($N$), ylabel=Latency ($\mu$s), grid=both, grid style={gray!30},
%                    xmin=256, xmax=32768, ymin=0.4, ymax=800,
%                    % tick align=outside, tickpos=left,
%                    %label style={font=\large},
%                    % tick label style={font=\large},
%                    grid style={dashed, gray!30},
%                    legend pos=south east, legend columns=1]
%           \addplot[mark=square, Paired-5,    semithick                              ] table [x index=0, y index=15] {../main/chapter4/fig/polar/sc_gen_thr_inter/dat/E31225_samples_inter_8b_aff3ct_r5_6.dat}; \label{plot:inter_line1}
%           \addplot[mark=o,      Paired-3,    semithick                              ] table [x index=0, y index=15] {../main/chapter4/fig/polar/sc_gen_thr_inter/dat/E31225_samples_inter_8b_handw_r5_6.dat }; \label{plot:inter_line2}
%           \addplot[mark=square, Paired-5!70, semithick, dashed, mark options={solid}] table [x index=0, y index=15] {../main/chapter4/fig/polar/sc_gen_thr_inter/dat/E31225_samples_inter_8b_aff3ct_r1_2.dat}; \label{plot:inter_line3}
%           \addplot[mark=o,      Paired-3!70, semithick, dashed, mark options={solid}] table [x index=0, y index=15] {../main/chapter4/fig/polar/sc_gen_thr_inter/dat/E31225_samples_inter_8b_handw_r1_2.dat }; \label{plot:inter_line4}
%       \end{axis}

%       \matrix [draw,
%                matrix of nodes,
%                anchor=north,
%                inner sep=2.3pt,
%                fill=white,
%                column 1/.style={anchor=base east},
%                ampersand replacement=\&] at (7.85,2.0)
%       {
%                             \& $R = 5/6$              \& $R = 1/2$              \\
%           this work         \& \ref{plot:inter_line1} \& \ref{plot:inter_line3} \\
%           \cite{LeGal2015a} \& \ref{plot:inter_line2} \& \ref{plot:inter_line4} \\
%       };
%     \end{tikzpicture}
%     }
%   \caption*{Inter-frame SIMD}
%   \end{figure}
%   \end{column}
%   \end{columns}
%   \vfill
%   \enumcite{Sarkis2014}

%   \enumcite{LeGal2015a}
% \end{frame}
