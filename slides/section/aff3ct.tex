%!TEX root = ../slides.tex

\section[\AFFECT]{\AFFECT: A Fast Forward Error Correction Toolbox}

\subsection[Motivations]{Motivations}

\begin{frame}{A Fast Forward Error Correction Toolbox}
  \vfill
  \textbf{Observations:}

  \vspace{0.1cm}
  \begin{itemize}
    \item Signal processing community: \textbf{not used to share source codes}
    \item Few existing open-source codes:
    \begin{itemize}
      \item Often \textbf{written with high-level languages}
      \item \textbf{Rare high performance implementations}
    \end{itemize}
  \end{itemize}

  \vfill
  \pause
  \textbf{\AFFECT:}

  \vspace{0.1cm}
  \begin{enumerate}
    \item \Cxx \textbf{open-source toolbox}: ease the \textbf{reproduction of scientific results}
    \pause
    \item Focus on high performance implementations: \textbf{high speed simulations}
    \pause
    \item Manage the \textbf{algorithmic heterogeneity} to propose a \textbf{unified library}
  \end{enumerate}
  \vfill
\end{frame}

\subsection[Library of Digital Communication Algorithms]{Library of Digital Communication Algorithms}

\begin{frame}{Features}
  \vfill
  \begin{itemize}
    \item \textbf{Channel codes:} Polar, LDPC, Turbo codes, Product codes, RSC, CRC, BCH, RS, RA, Repetition
    \vspace{0.1cm}
    \item \textbf{Channel models:} BSC, BEC, Optical, AWGN, Rayleigh
    \vspace{0.1cm}
    \item \textbf{Digital modulations:} PSK, QAM, PAM, OOK, CPM, SCMA
  \end{itemize}
  \vfill
  \pause

  \vspace*{.5em}
  ~~~~~{\color{bleuUni}\Large\MVRightarrow} Most digital communication standards are supported

  \vspace*{.5em}
  ~~~~~{\color{bleuUni}\Large\MVRightarrow} About \textbf{40 high performance software decoder implementations}
  \vfill
\end{frame}

\begin{frame}[fragile,containsverbatim]{Typical Digital Communication Chain}
  \begin{figure}[!h]
    \centering
    \scalebox{.60}{
    \begin{tikzpicture}
      \path[use as bounding box] (-1.4, 4.65) rectangle (12.65, -0.65);

      \tikzset{ tsk/.style ={draw=Paired-1, rounded corners=0pt, minimum height=1cm, minimum width=2.5cm, text=Paired-1} }
      \tikzset{ mdl/.style ={draw=Paired-3, rounded corners=2pt} }
      \tikzset{ sin/.style ={draw=Paired-7, circle, minimum width=0.3cm, text=black, preaction={fill=white}, pattern=north east lines, pattern color=Paired-7} }
      \tikzset{ sout/.style={draw=Paired-5, circle, minimum width=0.3cm, text=black, preaction={fill=white}, pattern=crosshatch dots,  pattern color=Paired-5} }

      \draw (-4.4, 4.65) rectangle (-1.6, 2.4);
      \node[draw=Paired-3, rounded corners=2pt, dashed, minimum height=0.3cm, minimum width=0.3cm, text=Paired-1, label={[Paired-3]right:Module}       ] (l2) at (-4.1, 4.3) {};
      \node[draw=Paired-1, rounded corners=0pt,         minimum height=0.3cm, minimum width=0.3cm, text=Paired-1, label={[Paired-1]right:Task}         ] (l1) at (-4.1, 3.8) {};
      \node[sout,                                                                                                 label={[Paired-5]right:Output socket}] (l3) at (-4.1, 3.3) {};
      \node[sin,                                                                                                  label={[Paired-7]right:Input socket} ] (l4) at (-4.1, 2.8) {};

      \only<2->{
      \node[tsk ] (src) at ( 0, 4) {generate};
      \node[sout] (src_so) at (0+1.25, 4) {};
      }
      \only<3->{
      \node[tsk ] (enc) at ( 4, 4) {encode};
      \node[sin ] (enc_si) at (4-1.25, 4) {};
      \node[sout] (enc_so) at (4+1.25, 4) {};
      }
      \only<4->{
      \node[tsk ] (mod) at ( 8, 4) {modulate};
      \node[sin ] (mod_si) at (8-1.25, 4) {};
      \node[sout] (mod_so) at (8+1.25, 4) {};
      }
      \only<5->{
      \node[tsk ] (chn) at (11.25, 2) {add\_noise};
      \node[sin ] (chn_si) at (11.25, 2.5) {};
      \node[sout] (chn_so) at (11.25, 1.5) {};
      }
      \only<4->{
      \node[tsk ] (dem) at ( 8, 0) {demodulate};
      \node[sout] (dem_so) at (8-1.25, 0) {};
      \node[sin ] (dem_si) at (8+1.25, 0) {};
      }
      \only<6->{
      \node[tsk ] (dec) at ( 4, 0) {decode};
      \node[sout] (dec_so) at (4-1.25, 0) {};
      \node[sin ] (dec_si) at (4+1.25, 0) {};
      }
      \only<7->{
      \node[tsk ] (mnt) at ( 0, 0) {check\_errors};
      \node[sin ] (mnt_si1) at (0+1.25, +0.2) {};
      \node[sin ] (mnt_si2) at (0+1.25, -0.2) {};
      }

      \only<2->{
      \node[mdl, label={[Paired-3]below:Source (Src)},  dashed, fit=(src) (src_so)] {};
      }
      \only<3->{
      \node[mdl, label={[Paired-3]below:Encoder (Enc)}, dashed, fit=(enc) (enc_si) (enc_so)] {};
      }
      \only<4->{
      \node[mdl, label={[Paired-3]center:Modem (Mdm)},  dashed, fit=(mod) (dem) (mod_si) (mod_so) (dem_si) (dem_so)] {};
      }
      \only<5->{
      \node[mdl, label={[Paired-3]below:Channel (Chn)}, dashed, fit=(chn) (chn_si) (chn_so)] {};
      }
      \only<6->{
      \node[mdl, label={[Paired-3]above:Decoder (Dec)}, dashed, fit=(dec) (dec_si) (dec_so)] {};
      }
      \only<7->{
      \node[mdl, label={[Paired-3]above:Monitor (Mnt)}, dashed, fit=(mnt) (mnt_si1) (mnt_si2)] {};
      }

      \only<8->{
      \draw[->,>=latex] (src_so)               -- (enc_si)       node [midway, above] {$\bm{u}$};
      }
      \only<9->{
      \draw[->,>=latex] (enc_so)               -- (mod_si)       node [midway, above] {$\bm{c}$};
      }
      \only<10->{
      \draw[->,>=latex] (mod_so)               -| (chn_si)       node [midway, above] {$\bm{x}$};
      }
      \only<11->{
      \draw[->,>=latex] (chn_so)               |- (dem_si)       node [midway, below] {$\bm{y}$};
      }
      \only<12->{
      \draw[->,>=latex] (dem_so)               -- (dec_si)       node [midway, below] {$\bm{l}$};
      }
      \only<13->{
      \draw[->,>=latex] (dec_so)      -- (2,0) |- (mnt_si2)      node [midway, below] {$\bm{\hat{u}}$};
      }
      \only<14->{
      \draw[->,>=latex] (src_so.east) -- (2,4) |- (mnt_si1.east) node [midway, above] {};
      }
    \end{tikzpicture}
    }
  \end{figure}

  \begin{overprint}
  \onslide<2>
  \begin{minted}[linenos, highlightlines={2}]{cpp}
// allocate the module objects
aff3ct::module::Source_random<>      Src(K   );
  \end{minted}
  \onslide<3>
  \begin{minted}[linenos, highlightlines={3}]{cpp}
// allocate the module objects
aff3ct::module::Source_random<>      Src(K   );
aff3ct::module::Encoder_repetition<> Enc(K, N);
  \end{minted}
  \onslide<4>
  \begin{minted}[linenos, highlightlines={4}]{cpp}
// allocate the module objects
aff3ct::module::Source_random<>      Src(K   );
aff3ct::module::Encoder_repetition<> Enc(K, N);
aff3ct::module::Modem_BPSK<>         Mdm(N   );
  \end{minted}
  \onslide<5>
  \begin{minted}[linenos, highlightlines={5}]{cpp}
// allocate the module objects
aff3ct::module::Source_random<>      Src(K   );
aff3ct::module::Encoder_repetition<> Enc(K, N);
aff3ct::module::Modem_BPSK<>         Mdm(N   );
aff3ct::module::Channel_AWGN<>       Chn(N   );
  \end{minted}
  \onslide<6>
  \begin{minted}[linenos, highlightlines={6}]{cpp}
// allocate the module objects
aff3ct::module::Source_random<>      Src(K   );
aff3ct::module::Encoder_repetition<> Enc(K, N);
aff3ct::module::Modem_BPSK<>         Mdm(N   );
aff3ct::module::Channel_AWGN<>       Chn(N   );
aff3ct::module::Decoder_repetiton<>  Dec(K, N);
  \end{minted}
  \onslide<7>
  \begin{minted}[linenos, highlightlines={7}]{cpp}
// allocate the module objects
aff3ct::module::Source_random<>      Src(K   );
aff3ct::module::Encoder_repetition<> Enc(K, N);
aff3ct::module::Modem_BPSK<>         Mdm(N   );
aff3ct::module::Channel_AWGN<>       Chn(N   );
aff3ct::module::Decoder_repetiton<>  Dec(K, N);
aff3ct::module::Monitor_BFER<>       Mnt(K, E);
  \end{minted}
  \onslide<8>
  \begin{minted}[linenos, highlightlines={2}]{cpp}
// bind the sockets over the tasks
Enc[sck::encode      ::u ].bind(Src[sck::generate   ::u]);
  \end{minted}
  \onslide<9>
  \begin{minted}[linenos, highlightlines={3}]{cpp}
// bind the sockets over the tasks
Enc[sck::encode      ::u ].bind(Src[sck::generate   ::u]);
Mdm[sck::modulate    ::c ].bind(Enc[sck::encode     ::c]);
  \end{minted}
  \onslide<10>
  \begin{minted}[linenos, highlightlines={4}]{cpp}
// bind the sockets over the tasks
Enc[sck::encode      ::u ].bind(Src[sck::generate   ::u]);
Mdm[sck::modulate    ::c ].bind(Enc[sck::encode     ::c]);
Chn[sck::add_noise   ::x ].bind(Mdm[sck::modulate   ::x]);
  \end{minted}
  \onslide<11>
  \begin{minted}[linenos, highlightlines={5}]{cpp}
// bind the sockets over the tasks
Enc[sck::encode      ::u ].bind(Src[sck::generate   ::u]);
Mdm[sck::modulate    ::c ].bind(Enc[sck::encode     ::c]);
Chn[sck::add_noise   ::x ].bind(Mdm[sck::modulate   ::x]);
Mdm[sck::demodulate  ::y ].bind(Chn[sck::add_noise  ::y]);
  \end{minted}
  \onslide<12>
  \begin{minted}[linenos, highlightlines={6}]{cpp}
// bind the sockets over the tasks
Enc[sck::encode      ::u ].bind(Src[sck::generate   ::u]);
Mdm[sck::modulate    ::c ].bind(Enc[sck::encode     ::c]);
Chn[sck::add_noise   ::x ].bind(Mdm[sck::modulate   ::x]);
Mdm[sck::demodulate  ::y ].bind(Chn[sck::add_noise  ::y]);
Dec[sck::decode      ::l ].bind(Mdm[sck::demodulate ::l]);
  \end{minted}
  \onslide<13>
  \begin{minted}[linenos, highlightlines={7}]{cpp}
// bind the sockets over the tasks
Enc[sck::encode      ::u ].bind(Src[sck::generate   ::u]);
Mdm[sck::modulate    ::c ].bind(Enc[sck::encode     ::c]);
Chn[sck::add_noise   ::x ].bind(Mdm[sck::modulate   ::x]);
Mdm[sck::demodulate  ::y ].bind(Chn[sck::add_noise  ::y]);
Dec[sck::decode      ::l ].bind(Mdm[sck::demodulate ::l]);
Mnt[sck::check_errors::u2].bind(Dec[sck::decode     ::u]);
  \end{minted}
  \onslide<14>
  \begin{minted}[linenos, highlightlines={8}]{cpp}
// bind the sockets over the tasks
Enc[sck::encode      ::u ].bind(Src[sck::generate   ::u]);
Mdm[sck::modulate    ::c ].bind(Enc[sck::encode     ::c]);
Chn[sck::add_noise   ::x ].bind(Mdm[sck::modulate   ::x]);
Mdm[sck::demodulate  ::y ].bind(Chn[sck::add_noise  ::y]);
Dec[sck::decode      ::l ].bind(Mdm[sck::demodulate ::l]);
Mnt[sck::check_errors::u2].bind(Dec[sck::decode     ::u]);
Mnt[sck::check_errors::u1].bind(Src[sck::generate   ::u]);
  \end{minted}
  \onslide<15>
  \begin{minted}[linenos]{cpp}
// build a sequence of tasks (= a chain) from the previous binding
aff3ct::tools::Sequence comm_chain(Src[tsk::generate    ],  // first task
                                   Mnt[tsk::check_errors]); // last task

// execute the communication chain
comm_chain.exec([&Mnt]() {
  return Mnt.is_done(); // stops when 'is_done()' returns 'true'
});
  \end{minted}
  \end{overprint}
\end{frame}

\subsection[Functionnal Simulator]{Simulation of Digital Communication Algorithms}

\begin{frame}{Simulation of a 5G Receiver}
  \vspace{-0.3cm}
  \begin{figure}[!h]
    \centering
    \scalebox{.50}{
    \begin{tikzpicture}
      \node[draw=Paired-5, dashed, rounded corners=0pt, minimum height=1cm, minimum width=2.5cm, text=Paired-5, align=center] (azcw) at ( 0, 0) {All-zero\\Codeword};

      \node[draw=Paired-1, rounded corners=0pt, minimum height=1cm, minimum width=2.5cm, text=Paired-1, align=center] (chn) at ( 3.5, 0) {AWGN\\Channel};
      \node[draw=Paired-3, rounded corners=0pt, minimum height=1cm, minimum width=2.5cm, text=Paired-3, align=center] (dem) at ( 7.0, 0) {BPSK\\Demodulator};
      \node[draw=Paired-3, rounded corners=0pt, minimum height=1cm, minimum width=2.5cm, text=Paired-3, align=center] (qnt) at (10.5, 0) {32 to 8-bit\\Quantizer};
      \node[draw=Paired-3, rounded corners=0pt, minimum height=1cm, minimum width=2.5cm, text=Paired-3, align=center] (dec) at (14.0, 0) {FA-SSCL\\Decoder};
      \node[draw=Paired-3, rounded corners=0pt, minimum height=1cm, minimum width=2.5cm, text=Paired-3, align=center] (crc) at (17.5, 0) {CRC\\Extractor};
      \node[draw=Paired-3, rounded corners=0pt, minimum height=1cm, minimum width=2.5cm, text=Paired-3              ] (mnt) at (21.0, 0) {Monitor};

      \node[draw=Paired-3, rounded corners=2pt, label={[Paired-3]above:Receiver}, minimum height=2cm, dashed, fit=(dem) (dec) (mnt)] {};

      \draw[->,>=latex, dashed] (azcw) -- (chn) node [midway, below] {$N$};
      \draw[->,>=latex        ] (chn)  -- (dem) node [midway, below] {$N$};
      \draw[->,>=latex        ] (dem)  -- (qnt) node [midway, below] {$N$};
      \draw[->,>=latex        ] (qnt)  -- (dec) node [midway, below] {$N$};
      \draw[->,>=latex        ] (dec)  -- (crc) node [midway, below] {$K_1$};
      \draw[->,>=latex        ] (crc)  -- (mnt) node [midway, below] {$K_2$};
    \end{tikzpicture}
    }
  \end{figure}
  \pause
  \begin{columns}
  \begin{column}[T]{8cm}
  \vspace{-0.3cm}
  \begin{figure}[!h]
    \centering
    \scalebox{.65}{
    \begin{tikzpicture}
      \begin{groupplot}[/pgfplots/table/ignore chars={|},
                        height=1.125\textwidth,  width=0.75\textwidth,
                        xticklabel style={black!70}, yticklabel style={black!70},
                        group style={group name=ccsds, group size= 2 by 1, horizontal sep=2cm, vertical sep=2.2cm},
                        ymode = log, xmin=2.5, xmax=4.5, xtick={0,0.5,...,4.5},
                        xlabel=$E_b/N_0 \text{(dB)}$, grid=both, grid style={gray!30},
                        legend pos=south west]
        \nextgroupplot[ylabel=Frame Error Rate]
        \addplot[mark=o,      Paired-5, semithick] table [x=Eb/N0, y=FER] {../main/chapter3/fig/polar/scl_adaptive/dat/polar_2048_1723_1.0_4.5_FASCL8_CRC32_SPC4_i5-6600K.txt};  \label{plot:line1}
        \addplot[mark=square, Paired-1, semithick] table [x=Eb/N0, y=FER] {../main/chapter3/fig/polar/scl_adaptive/dat/polar_2048_1723_1.0_4.5_FASCL32_CRC32_SPC4_i5-6600K.txt}; \label{plot:line2}

        \coordinate (legend) at (axis description cs:0.61,0.03);

        \nextgroupplot[ylabel=Information throughput (Mb/s), log basis y={2}, log ticks with fixed point, grid style={dashed, gray!30}]
        \addplot[mark=o,      Paired-5, semithick] table [x=Eb/N0, y=ITHR] {../main/chapter3/fig/polar/scl_adaptive/dat/polar_2048_1723_1.0_4.5_FASCL8_CRC32_SPC4_i5-6600K.txt};
        \addplot[mark=square, Paired-1, semithick] table [x=Eb/N0, y=ITHR] {../main/chapter3/fig/polar/scl_adaptive/dat/polar_2048_1723_1.0_4.5_FASCL32_CRC32_SPC4_i5-6600K.txt};
      \end{groupplot}
      \only<1-2>{
      \draw[white, fill=white] (5, -0.95) rectangle (11.05, 7.45);
      }

      \matrix [
          draw,
          matrix of nodes,
          anchor=south east,
          fill=white,
          column 1/.style={anchor=base west},
          ampersand replacement=\&
      ] at (legend) {
          $L = 8$  \& \ref{plot:line1} \\
          $L = 32$ \& \ref{plot:line2} \\
      };
    \end{tikzpicture}
    }
  \end{figure}
  \end{column}
  \begin{column}[T]{4cm}
    \vspace{0.2cm}
    \begin{itemize}
      \item $N = 2048$
      \item $K_1 = 1755$
      \item $K_2 = 1723$
      \vspace{0.3cm}
      \item Adaptive SSCL decoders
      \item Based on SSC decoder
      \item $L$ decoding trees
      \item<3-> Early termination criteria with a CRC
    \end{itemize}
  \end{column}
  \end{columns}
\end{frame}

\begin{frame}{Evaluation on Various Multi-core Architectures}
  \begin{table}[htp]
    \vspace{-0.7cm}
    \centering
    {\resizebox{\linewidth}{!}{
    \begin{tabular}{l  l | l | l | c | c | c | c | c}
    \multicolumn{2}{c |}{\multirow{2}{*}{\textbf{CPU}}} & \multicolumn{2}{c |}{\textbf{SIMD instr.}} & \multirow{2}{*}{\textbf{\# Proc.}} & \textbf{\# Cores}  & \textbf{Freq.} & \multirow{2}{*}{\textbf{SMT}} & \textbf{Turbo} \\ \cline{3-4}
            &                                           & \textbf{Name}      & \textbf{Size}         &                                    & \textbf{per Proc.} & (GHz)          &                               & \textbf{Boost} \\
    \hline
    ARM\R   & \textcolor{Paired-5}{ThunderX2\R CN9975}  & \texttt{NEON}      & 128-bit               & 2                                  &  28                & 2.00           & 4                             & \xmark         \\
    Intel\R & \textcolor{Paired-3}{Xeon Phi\TM 7230}    & \texttt{AVX-512F}  & 512-bit               & 1                                  &  64                & 1.30           & 4                             & \cmark         \\
    Intel\R & \textcolor{Paired-7}{Xeon\TM E5-2680 v3}  & \texttt{AVX2}      & 256-bit               & 2                                  &  12                & 2.50           & 1                             & \xmark         \\
    Intel\R & \textcolor{Paired-1}{Xeon\TM Gold 6140}   & \texttt{AVX-512BW} & 512-bit               & 2                                  &  18                & 2.30           & 2                             & \cmark         \\
  % Intel\R & Xeon\TM Gold 6142                         & \texttt{AVX-512BW} & 512-bit               & 2                                  &  16                & 2.60           & 1                             & \xmark         \\
    Intel\R & \textcolor{Paired-1!60}{Xeon\TM Gold 6240}& \texttt{AVX-512BW} & 512-bit               & 2                                  &  18                & 2.60           & 1                             & \xmark         \\
  % AMD\R   & EPYC\TM 7452                              & \texttt{AVX2}      & 256-bit               & 2                                  &  32                & 2.35           & 1                             & \xmark         \\
    AMD\R   & \textcolor{Paired-9}{EPYC\TM 7702}        & \texttt{AVX2}      & 256-bit               & 2                                  &  64                & 2.00           & 1                             & \xmark         \\
    \end{tabular}
    }}
  \end{table}

  \begin{columns}
    \begin{column}[T]{6cm}
      \vspace{0.79cm}
      \begin{figure}
      \centering
      \scalebox{.50}{
      \begin{tikzpicture}

        \begin{axis}[/pgfplots/table/ignore chars={ },
                     width=2.0\linewidth, height=1.4\linewidth,
                     xticklabel style={black!70}, yticklabel style={black!70},
                     xlabel=Number of threads, ylabel=Speedup, grid=both, grid style={gray!30},
                     xmin=1, xmax=256, xtick={1,32,64,...,256}, ytick={1,16,32,...,96},
                     ymin=-5, ymax=102,
                     grid style={dashed, gray!30},
                     legend pos=south east, legend columns=1]
            \addplot[mark=pentagon, Paired-5,    semithick] table [x=t, y expr=\thisrow{Mbps}/ 28.219] {../main/chapter4/fig/simu/speedup/dat/Cavium_ThunderX2_CN9975.txt}; \label{plot:lline1}
            \addplot[mark=triangle, Paired-3,    semithick] table [x=t, y expr=\thisrow{Mbps}/ 26.892] {../main/chapter4/fig/simu/speedup/dat/Intel_Xeon_Phi_7230.txt    }; \label{plot:lline2}
            \addplot[mark=o,        Paired-1,    semithick] table [x=t, y expr=\thisrow{Mbps}/164.946] {../main/chapter4/fig/simu/speedup/dat/Intel_Xeon_Gold_6140.txt   }; \label{plot:lline3}
            \addplot[mark=otimes,   Paired-1!60, semithick] table [x=t, y expr=\thisrow{Mbps}/126.409] {../main/chapter4/fig/simu/speedup/dat/Intel_Xeon_Gold_6240.txt   }; \label{plot:lline4}
            \addplot[mark=square,   Paired-7,    semithick] table [x=t, y expr=\thisrow{Mbps}/ 81.759] {../main/chapter4/fig/simu/speedup/dat/Intel_Xeon_E5-2680v3.txt   }; \label{plot:lline5}
            \addplot[mark=+,        Paired-9,    semithick] table [x=t, y expr=\thisrow{Mbps}/113.040] {../main/chapter4/fig/simu/speedup/dat/AMD_EPYC_7702.txt          }; \label{plot:lline6}
            \addplot[mark=none,     black!65              ] coordinates {(0,0) (512,512)};

            \path (16.2, 12.4) coordinate (X);
        \end{axis}

        \matrix[draw,
                matrix of nodes,
                anchor=south west,
                inner sep=2.3pt,
                column 2/.style={anchor=base east},
                fill=white,
                ampersand replacement=\&]
                at (5.60,0.18)
        {
            \ref{plot:lline1} \& $2 \times$ThunderX2 CN9975 \\
            \ref{plot:lline2} \& $1 \times$Xeon Phi 7230    \\
            \ref{plot:lline5} \& $2 \times$Xeon E5-2680 v3  \\
            \ref{plot:lline3} \& $2 \times$Xeon Gold 6140   \\
            \ref{plot:lline4} \& $2 \times$Xeon Gold 6240   \\
            \ref{plot:lline6} \& $2 \times$EPYC 7702        \\
        };
      \end{tikzpicture}
      }
      \end{figure}
    \end{column}
    \begin{column}[T]{6cm}
      \begin{figure}
      \centering
      \scalebox{.50}{
      \begin{tikzpicture}
      \path[use as bounding box] (-1.55, -1.05) rectangle (10.45, 8.4);
      \begin{axis}[ybar,
                   width=2.0\linewidth, height=1.4\linewidth,
                   xticklabel style={black!70}, yticklabel style={black!70},
                   bar width=6pt,
                   legend style={at={(0.985,0.975)}, anchor=north east,legend columns=1},
                   legend cell align=right,
                   area legend,
                   ylabel={Info. throughput (Mb/s)},
                   symbolic x coords={1T1SMT,NT1SMT,NT2SMT,NT3SMT,NT4SMT},
                   xticklabel style={align=center},
                   xticklabels={Single thread\\1-way SMT,
                                All threads\\1-way SMT,
                                All threads\\2-way SMT,
                                All threads\\3-way SMT,
                                All threads\\4-way SMT},
                   xtick=data,
                   ymajorgrids, grid style={dashed, gray!30},
                   ymin=0,
                   ymax=5900,
                   nodes near coords,
                   nodes near coords align={vertical},
                   every node near coord/.append style={rotate=90, anchor=west, font=\footnotesize},
                   restrict y to domain*=0:6500, % Cut values off at 14
                   visualization depends on=rawy\as\rawy, % Save the unclipped values
                   after end axis/.code={ % Draw line indicating break
                      \draw [ultra thick, white, decoration={snake, amplitude=1pt}, decorate] (rel axis cs:0,1.05) -- (rel axis cs:1,1.05);
                   },
                   nodes near coords={%
                    \pgfmathprintnumber{\rawy}% Print unclipped values
                   },
                   axis lines*=left,
                   clip=false
                  ]

          % ThunderX2 CN9975
          \addplot[black!25!Paired-5,fill=Paired-5!40,draw=Paired-5,
                   postaction={pattern color = black!50!Paired-5!70,
                               pattern=crosshatch dots}
                  ]
                  coordinates {(1T1SMT,28) (NT1SMT,1573) (NT2SMT,2148) (NT3SMT,2446) (NT4SMT,2604)};
          \label{plot1}

          % Xeon Phi 7230
          \addplot[black!25!Paired-3,fill=Paired-3!40,draw=Paired-3,
                    postaction={pattern color = black!50!Paired-3!70,
                                pattern=horizontal lines}
                   ]
                   coordinates {(1T1SMT,27) (NT1SMT,1716) (NT2SMT,2196) (NT3SMT,2208) (NT4SMT,2159)};
          \label{plot2}

          % Xeon E5-2680 v3
          \addplot[black!25!Paired-7,fill=Paired-7!40,draw=Paired-7,
                   postaction={pattern color = black!50!Paired-7!70,
                               pattern=grid}
                  ]
                  coordinates {(1T1SMT,82) (NT1SMT,1950)};
          \label{plot4}

          % Xeon Gold 6140
          \addplot[black!25!Paired-1,fill=Paired-1!40,draw=Paired-1,
                   postaction={pattern color = black!50!Paired-1!70,
                               pattern=flexible hatch north west}
                  ]
                  coordinates {(1T1SMT,165) (NT1SMT,3533) (NT2SMT,5436)};
          \label{plot5}

          % Xeon Gold 6240
          \addplot[black!25!Paired-1!60,fill=Paired-1!20,draw=Paired-1!60,
                   postaction={pattern color = black!50!Paired-1!50,
                               pattern=flexible hatch north east}
                  ]
                  coordinates {(1T1SMT,126) (NT1SMT,4294)};
          \label{plot6}

          % EPYC 7702
          \addplot[black!25!Paired-9,fill=Paired-9!40,draw=Paired-9,
                   postaction={pattern color = black!50!Paired-9!70,
                               pattern=none}
                  ]
                  coordinates {(1T1SMT,113) (NT1SMT,11145)};
          \label{plot3}

          \legend{$2 \times$ThunderX2 CN9975,
                  $1 \times$Xeon Phi 7230,
                  $2 \times$Xeon E5-2680 v3,
                  $2 \times$Xeon Gold 6140,
                  $2 \times$Xeon Gold 6240,
                  $2 \times$EPYC 7702,}
      \end{axis}
      \end{tikzpicture}
      }
      \end{figure}
    \end{column}
  \end{columns}

\end{frame}
