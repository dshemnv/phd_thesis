%!TEX root = ../slides.tex

\section[MIPP]{\MIPP: A \Cxx Wrapper for SIMD Instructions}

\subsection[Proposal]{Proposal}

\begin{frame}{\longMIPP (\MIPP)}
  \vfill
  \textbf{Objectives:}

  \vspace{0.1cm}
  \begin{enumerate}
    \item Efficient software implementations of digital communication algorithms
    \item Portable description: adapt to various ISAs
    \item Flexible source code: manage multiple data types
  \end{enumerate}
  \vspace{0.3cm}
  \pause
  \textbf{Proposal:}

  \vspace{0.1cm}
  \begin{itemize}
    \item<2-> Single Instruction Multiple Data (SIMD) paradigm
    \item<3-> SIMD \textbf{\Cxx wrapper}
    \begin{itemize}
      \item<3-> Maximizes code portability
      \item<3-> \textbf{Improves expressiveness} over intrinsics
      \item<3-> \textbf{No dependencies} to other libraries
    \end{itemize}
    \item<4-> Programming model \textbf{close to intrinsics}
    \begin{itemize}
      \item<4-> \textbf{Lightweight approach} compared to existing solutions (Boost.SIMD, Eigen)
      \item<4-> Focus on \textbf{fixed-point instructions}
    \end{itemize}
  \end{itemize}
  \vfill
\end{frame}

\subsection[Programming Interface]{Programming Interface}

\begin{frame}[fragile]{Features}
  \vfill
  \begin{itemize}
    \item \textcolor{Paired-5}{Typed}, portable SIMD \textbf{data register} (\texttt{mipp::Reg<\textcolor{Paired-1}{T}>})
    \vspace{0.2cm}
    \item \textcolor{Paired-5}{Typed}, portable SIMD \textbf{mask register} (\texttt{mipp::Msk<\textcolor{Paired-1}{N}>})
    \vspace{0.2cm}
    \item Supports \textcolor{Paired-5}{\textbf{operator overloading}}
    \begin{itemize}
      \item \texttt{mipp::Reg<\textcolor{Paired-1}{float}> a = 1.f, b = 2.f; \textcolor{Paired-1}{auto} c = a + b;}
    \end{itemize}
    \vspace{0.2cm}
    \item Eases the \textcolor{Paired-5}{\textbf{register initializations}} (broadcasts, sets and loads)
    \begin{itemize}
      \item \texttt{mipp::Reg<\textcolor{Paired-1}{int}> a = 1, b = \{1, 2, 3, 4\}, c = ptr;}
    \end{itemize}
    \vspace{0.2cm}
    \item Defines an \textbf{aligned memory allocator} for the \texttt{std::vector<\textcolor{Paired-1}{T},\textcolor{Paired-1}{A}>} class (\texttt{mipp::vector<\textcolor{Paired-1}{T}>})
  \end{itemize}
  \vfill
\end{frame}

\begin{frame}[fragile]{Hello World!}
  \vspace{-2mm}
  \begin{columns}[t]
    \begin{column}[T]{8.6cm}
      \begin{overprint}
      \onslide<1>
      \begin{minted}[linenos]{cpp}
template <typename T>
void addVector(const std::vector<T> &A,
               const std::vector<T> &B,
                     std::vector<T> &C)
{
	// N elements per SIMD register (static)
	constexpr int N = mipp::N<T>();
	for (int i = 0; i < A.size(); i += N)
	{
		mipp::Reg<T> rA = &A[i];   // SIMD load
		mipp::Reg<T> rB = &B[i];   // SIMD load
		mipp::Reg<T> rC = rA + rB; // SIMD addition
		rC.store(&C[i]);           // SIMD store
	}
}
      \end{minted}
      \onslide<2>
      \begin{minted}[linenos, highlightlines={10-13}]{cpp}
template <typename T>
void addVector(const std::vector<T> &A,
               const std::vector<T> &B,
                     std::vector<T> &C)
{
	// N elements per SIMD register (static)
	constexpr int N = mipp::N<T>();
	for (int i = 0; i < A.size(); i += N)
	{
		mipp::Reg<T> rA = &A[i];   // SIMD load
		mipp::Reg<T> rB = &B[i];   // SIMD load
		mipp::Reg<T> rC = rA + rB; // SIMD addition
		rC.store(&C[i]);           // SIMD store
	}
}
      \end{minted}
      \onslide<3>
      \begin{minted}[linenos, highlightlines={2-4,7,10-12}]{cpp}
template <typename T>
void addVector(const std::vector<T> &A,
               const std::vector<T> &B,
                     std::vector<T> &C)
{
	// N elements per SIMD register (static)
	constexpr int N = mipp::N<T>();
	for (int i = 0; i < A.size(); i += N)
	{
		mipp::Reg<T> rA = &A[i];   // SIMD load
		mipp::Reg<T> rB = &B[i];   // SIMD load
		mipp::Reg<T> rC = rA + rB; // SIMD addition
		rC.store(&C[i]);           // SIMD store
	}
}
      \end{minted}
      \onslide<4>
      \begin{minted}[linenos]{cpp}
template <typename T>
void addVector(const std::vector<T> &A,
               const std::vector<T> &B,
                     std::vector<T> &C)
{
	// N elements per SIMD register (static)
	constexpr int N = mipp::N<T>();
	for (int i = 0; i < A.size(); i += N)
	{
		mipp::Reg<T> rA = &A[i];   // SIMD load
		mipp::Reg<T> rB = &B[i];   // SIMD load
		mipp::Reg<T> rC = rA + rB; // SIMD addition
		rC.store(&C[i]);           // SIMD store
	}
}
      \end{minted}
      \end{overprint}
    \end{column}
  \end{columns}
  \pause
  \begin{itemize}
    \item \MIPP operates at \textbf{register level}
    \pause
    \item Same code can \textbf{work on various data types} and \textbf{instruction sets}
    \pause
    	\\\vspace*{.5em}
      {\color{bleuUni}\Large\MVRightarrow} Good for \textbf{flexibility} and \textbf{portability}!
  \end{itemize}
\end{frame}

\subsection[Related works]{Related works}

\begin{frame}[fragile]{SIMD Wrappers}
  \vfill
  Two main strategies:

  \vspace{0.1cm}
  \begin{enumerate}
    \item \textbf{operator overloading} (used in \MIPP)
    \item \textbf{expression templates}, can automatically match instructions like Fused Multiply–Add (FMA, $d = a \times b + c$)
  \end{enumerate}

  \begin{table}[htp]
  \tabcolsep=6pt
  \centering
  \label{tab:comp}
  {\small\resizebox{1.0\linewidth}{!}{
  \begin{tabular}{|r|r|r||c|c|c|c||c|c|c|c|c|c||c|c|}
  \hline
  \multicolumn{3}{|c||}{\multirow{2}{*}{\textbf{General Information}}}                                    & \multicolumn{4}{c||}{\multirow{2}{*}{\textbf{Instruction Set}}}                                    & \multicolumn{6}{c||}{\multirow{2}{*}{\textbf{Data Type}}}                    & \multicolumn{2}{c|}{\multirow{2}{*}{\textbf{Features}}} \\
  \multicolumn{3}{|c||}{}                                                                                 & \multicolumn{4}{c||}{}                                                                             & \multicolumn{6}{c||}{}                                                       & \multicolumn{2}{c|}{}                                   \\ \hline
  \multicolumn{2}{|r|}{\textbf{Name}} & \multicolumn{1}{r||}{\textbf{Ref.}}                               & \textbf{\texttt{SSE}} & \textbf{\texttt{AVX}} & \textbf{\texttt{AVX-512}} & \textbf{\texttt{NEON}} & \multicolumn{2}{c|}{\textbf{Float}} & \multicolumn{4}{c||}{\textbf{Integer}} & \textbf{Math}  & \textbf{\Cxx}                          \\ \cline{8-13}
  \multicolumn{2}{|r|}{} & \multicolumn{1}{r||}{}                                                         & 128-bit               & 256-bit               & 512-bit                   & 128-bit                & 64      & 32                        & 64     & 32     & 16     & 8           & \textbf{Func.} & \textbf{Technique}                     \\ \hline \hline
  \multirow{8}{*}{\rotatebox[origin=c]{90}{\textbf{Library}}} & \textbf{\MIPP}      & \cite{Cassagne2018} & \cmark                & \cmark                & \cmark                    & \cmark                 & \cmark  & \cmark                    & \cmark & \cmark & \cmark & \cmark      & \cmark         & Op. overload.                          \\ %\cline{2-19}
                                                              & \textbf{\VCL}       & \cite{Fog}          & \cmark                & \cmark                & \cmark                    & \xmark                 & \cmark  & \cmark                    & \cmark & \cmark & \cmark & \cmark      & \cmark         & Op. overload.                          \\ %\cline{2-19}
                                                              & \textbf{\simdpp}    & \cite{Kanapickas}   & \cmark                & \cmark                & \cmark                    & \cmark                 & \cmark  & \cmark                    & \cmark & \cmark & \cmark & \cmark      & \xmark         & Expr. templ.                           \\ %\cline{2-19}
                                                              & \textbf{\TSIMD}     & \cite{Moller2016}   & \cmark                & \cmark                & \xmark                    & \cmark                 & \xmark  & \cmark                    & \xmark & \cmark & \cmark & \cmark      & \xmark         & Op. overload.                          \\ %\cline{2-19}
                                                              & \textbf{\Vc}        & \cite{Kretz2012}    & \cmark                & \cmark                & \xmark                    & \xmark                 & \cmark  & \cmark                    & \cmark & \cmark & \cmark & \xmark      & \cmark         & Op. overload.                          \\ %\cline{2-19}
                                                              & \textbf{\xsimd}     & \cite{Mabille}      & \cmark                & \cmark                & \xmark                    & \xmark                 & \cmark  & \cmark                    & \cmark & \cmark & \xmark & \xmark      & \cmark         & Op. overload.                          \\ %\cline{2-19}
                                                              & \textbf{\BoostSIMD} & \cite{Esterie2012}  & \cmark                & \xmark                & \xmark                    & \xmark                 & \cmark  & \cmark                    & \cmark & \cmark & \cmark & \cmark      & \cmark         & Expr. templ.                           \\ %\cline{2-19}
                                                              & \textbf{\bSIMD}     & \cite{Esterie2012a} & \cmark                & \cmark                & \cmark                    & \cmark                 & \cmark  & \cmark                    & \cmark & \cmark & \cmark & \cmark      & \cmark         & Expr. templ.                           \\ \hline
  \end{tabular}
  }}
  \caption*{Main \Cxx SIMD wrappers/libraries and their characteristics in 2018.}
  \end{table}
  \vfill
\end{frame}

\begin{frame}{Experimentation Protocol}
  \vfill
  \begin{itemize}
    \item Experiments
    \begin{itemize}
    \item Comparison to \textbf{related works} on the Mandelbrot set
    \end{itemize}
    \item Tests are \textbf{mono-threaded}
    \item GNU \Cxx compiler version 5
  \end{itemize}
  \vfill
  \begin{table}[htp]
    \tabcolsep=6pt
    \centering
    \label{tab:targets}
    {\small
    \begin{tabular}{c|c c c}
    \textbf{Instr. set}             & \textbf{\texttt{NEON}} & \textbf{\texttt{SSE / AVX}} & \textbf{\texttt{AVX-512}} \\ \hline
    \textbf{Name}                   & \textbf{Exynos5422}    & \textbf{Core i5-5300U}      & \textbf{Xeon Phi 7230}    \\ %\hline
    \textbf{Year}                   & 2014                   & 2015                        & 2016                      \\ %\hline
    \textbf{Vendor}                 & Samsung                & Intel                       & Intel                     \\ %\hline
    \multirow{2}{*}{\textbf{Arch.}} & ARMv7                  & x86-64                      & Knights                   \\
                                    & Cortex A15             & Broadwell                   & Landing                   \\ %\hline
    \textbf{Cores/Freq.}            & 4/2.0 GHz              & 2/2.3 GHz                   & 64/1.3 GHz                \\ %\hline
    \textbf{LLC}                    & 2 MB L2                & 3 MB L3                     & 32 MB L2                  \\ %\hline
    \textbf{TDP}                    & $\sim$4 W              & 15 W                        & 215 W                     \\
    \end{tabular}
    }
  \end{table}
  \vfill
\end{frame}

\begin{frame}{Comparison on the Mandelbrot Set}
  \begin{itemize}
    \item \textbf{Computational intensive} kernel
    \item Need to manage an \textbf{early termination}
    \item Benchmark \Cxx source code is freely available\footnote{Full code is available online: \url{https://gitlab.inria.fr/acassagn/mandelbrot}}
  \end{itemize}
  \pause
  \begin{figure}[htp]
    \centering
    \begin{tikzpicture}[scale=0.8, every node/.style={transform shape}]
    \begin{axis}[ybar,
                 width=1.0\linewidth, height=0.500\linewidth,
                 bar width=4pt,
                 legend style={at={(0.985,0.975)}, anchor=north east,legend columns=1},
                 legend cell align=right,
                 area legend,
                 ylabel={Speeedup}, ytick={0,2,...,14},
                 symbolic x coords={intr,mippm,vcl,simdpp,tsimd,vc,xsimd,boost},
                 xticklabel style={black!70}, yticklabel style={black!70},
                 xticklabels={Intrinsics,
                              MIPP,
                              VCL,
                              simdpp,
                              T-SIMD,
                              Vc,
                              xsimd,
                              Boost.SIMD},
                 xtick=data,
                 ymajorgrids,
                 grid style={dashed, gray!30},
                 ymin=0, ymax=15,
                 every node near coord/.append style={rotate=90, anchor=west},
                 x tick label style={rotate=-45},
                ]

        % NEON
        \addplot[black!25!Paired-5,fill=Paired-5!40,draw=Paired-5,
                 postaction={pattern color = black!50!Paired-5!70,
                             pattern=crosshatch dots}
                ]
                coordinates {(intr,3.52) (mippm,3.45) (vcl,0) (simdpp,2.44) (tsimd,3.37) (vc,0) (xsimd,0) (boost,0)};
        \label{plot1}

        % SSE
        % \only<3->{
        \addplot[black!25!Paired-3,fill=Paired-3!40,draw=Paired-3,
                  postaction={pattern color = black!50!Paired-3!70,
                              pattern=horizontal lines}
                 ]
                 coordinates {(intr,3.74) (mippm,3.70) (vcl,3.67) (simdpp,3.28) (tsimd,3.39) (vc,1.17) (xsimd,3.40) (boost,1.03)};
        \label{plot2}
        % }

        % AVX
        % \only<4->{
        \addplot[black!25!Paired-7,fill=Paired-7!40,draw=Paired-7,
                 postaction={pattern color = black!50!Paired-7!70,
                             pattern=grid}
                ]
                coordinates {(intr,6.98) (mippm,6.76) (vcl,6.98) (simdpp,5.92) (tsimd,5.48) (vc,6.81) (xsimd,6.99) (boost,0.83)};
        \label{plot4}
        % }

        % AVX-512
        % \only<5->{
        \addplot[black!25!Paired-1,fill=Paired-1!40,draw=Paired-1,
                 postaction={pattern color = black!50!Paired-1!70,
                             pattern=flexible hatch north west}
                ]
                coordinates {(intr,14.30) (mippm,14.30) (vcl,14.30) (simdpp,7.84) (tsimd,5.49) (vc,5.16) (boost,0.84)};
        \label{plot5}
        % }

        \legend{\texttt{NEON},
                \texttt{SSE},
                \texttt{AVX},
                \texttt{AVX-512}}
    \end{axis}
    \end{tikzpicture}
  \end{figure}
\end{frame}
