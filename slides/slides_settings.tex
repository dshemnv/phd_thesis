%!TEX root = ./slides.tex

% Packages --------------------------------------------------------------------
\usepackage[T1]{fontenc}
\usepackage[utf8]{inputenc}
% \usepackage[frenchb]{babel}
\usepackage[english]{babel}
\usepackage{eulervm}
\usepackage{etoolbox,refcount}
\usepackage[normalem]{ulem} % strikeout with \sout{}
\usepackage{booktabs}
\usepackage{multirow}
\usepackage{multicol}
\usepackage{makecell}
\usepackage{pgfplots}
\usepackage{tikz}
\usepackage{circuitikz}
\usepackage{tikz-timing} % package pour les chronogrammes
\usepackage{caption}
\usepackage{graphicx}
\usepackage[export]{adjustbox} % align in includegraphics
\usepackage{stmaryrd} % llbracket
\usepackage{amsmath}
\usepackage{amssymb}
\usepackage{marvosym} % grosse fleche
\usepackage{calc}
\usepackage[english,onelanguage,ruled,linesnumbered,vlined]{algorithm2e}
\usepackage{bm}
\usepackage{pifont}
\usepackage{minted} % beautiful source code
\usepackage{tcolorbox}
\usepackage{etoolbox}
\usepackage[
  labelnumber  = true,
  backend      = biber,
  % style        = alphabetic,
  % citestyle    = alphabetic,
  maxnames     = 6,
  defernumbers = true,
  isbn         = false,
  doi          = true,
  url          = true,
  backref      = true]{biblatex}
\addbibresource{../tail/bibliography.bib}

% Parameters ------------------------------------------------------------------
\pgfplotsset{compat=newest}
\usepgfplotslibrary{groupplots}
\usetikzlibrary{matrix, positioning, fit, patterns, shapes, arrows, shapes.multipart, decorations.pathmorphing, calc}
\tikzset{
    invisible/.style={opacity=0},
    visible on/.style={alt={#1{}{invisible}}},
    alt/.code args={<#1>#2#3}{%
      \alt<#1>{\pgfkeysalso{#2}}{\pgfkeysalso{#3}} % \pgfkeysalso doesn't change the path
    },
    hatch distance/.store in=\hatchdistance,
    hatch distance=7pt,
    hatch thickness/.store in=\hatchthickness,
    hatch thickness=0.5pt,
}
% \SetAlFnt{\footnotesize}
\AtBeginEnvironment{frame}{\setcounter{footnote}{0}}

% package 'minted'
\setminted{bgcolor=marronUni,fontsize=\fontsize{8}{9.2},tabsize=4,numbers=left,framesep=0pt,numbersep=2pt,xleftmargin=3.5mm,xrightmargin=3.5mm,highlightcolor=marronUni!90} % package 'minted'
\usemintedstyle{monokai} % friendly fruity
% https://tex.stackexchange.com/questions/252263/alignment-of-minted-line-numbers
% \newlength{\mintednumbersep}
% \AtBeginDocument{%
%   \sbox0{\tiny00}%
%   \setlength\mintednumbersep{\parindent}%
%   \addtolength\mintednumbersep{-\wd0}%
% }
\BeforeBeginEnvironment{minted}{\begin{tcolorbox}[colback=marronUni, colframe=black, arc=0.5mm, boxsep=0mm, boxrule=0.0mm, left=0mm, right=0mm, top=0mm, bottom=0mm]}
\AfterEndEnvironment{minted}{\end{tcolorbox}}
\renewcommand{\theFancyVerbLine}{\textcolor[rgb]{1,1,1}{\tiny{\arabic{FancyVerbLine}}}}
% \renewcommand{\theFancyVerbLine}{\sffamily\textcolor[rgb]{1,1,1}{\footnotesize\oldstylenums{\arabic{FancyVerbLine}}}}

% new shapes for pgfplot
\makeatletter
\pgfdeclarepatternformonly[\hatchdistance,\hatchthickness]{flexible hatch north east}
{\pgfqpoint{0pt}{0pt}}
{\pgfqpoint{\hatchdistance}{\hatchdistance}}
{\pgfpoint{\hatchdistance-1pt}{\hatchdistance-1pt}}%
{
    \pgfsetcolor{\tikz@pattern@color}
    \pgfsetlinewidth{\hatchthickness}
    \pgfpathmoveto{\pgfqpoint{0pt}{0pt}}
    \pgfpathlineto{\pgfqpoint{\hatchdistance}{\hatchdistance}}
    \pgfusepath{stroke}
}
\makeatletter
\pgfdeclarepatternformonly[\hatchdistance,\hatchthickness]{flexible hatch north west}
{\pgfqpoint{0pt}{0pt}}
{\pgfqpoint{\hatchdistance}{\hatchdistance}}
{\pgfpoint{\hatchdistance-1pt}{\hatchdistance-1pt}}%
{
    \pgfsetcolor{\tikz@pattern@color}
    \pgfsetlinewidth{\hatchthickness}
    \pgfpathmoveto{\pgfqpoint{\hatchdistance}{0pt}}
    \pgfpathlineto{\pgfqpoint{0pt}{\hatchdistance}}
    \pgfusepath{stroke}
}

\setbeamertemplate{caption}{\raggedright\insertcaption\par}

% biblatex
% separate multi-citations by a comma instead of a semicolon in 'biblatex'
\renewcommand*{\multicitedelim}{\addcomma\addspace}
% \renewcommand\mkbibacro[1]{{\footnotesize\MakeUppercase{#1}}}
\definecolor{bluecite}{HTML}{009DE0}
\setbeamertemplate{bibliography item}{\textcolor{black}{\insertbiblabel}}
\newcommand{\enumcite}[1]{{\scriptsize\textcolor{bluecite}{\cite{#1}}\quad \fullcite{#1}}}
\newcommand{\citeblue}[1]{{\textcolor{bluecite}{\cite{#1}}}}

% Beamer template -------------------------------------------------------------
\usepackage{color}

\definecolor{Comment}{RGB}{97,161,176}

\definecolor{btfGreen}{RGB}{51,160,44}
\definecolor{btfRed}{RGB}{190,60,90}

\definecolor{bleuUni}{RGB}{0, 157, 224}
\definecolor{marronUni}{RGB}{68, 58, 49}
\definecolor{grayMarronUni}{RGB}{60, 60, 60}
\definecolor{grayBleuUni}{RGB}{118, 118, 118}

\definecolor{bluecite}{HTML}{009DE0}

\definecolor{Paired-2}{RGB}{166,206,227}
\definecolor{Paired-1}{RGB}{31,120,180}
\definecolor{Paired-4}{RGB}{178,223,138}
\definecolor{Paired-3}{RGB}{51,160,44}
\definecolor{Paired-6}{RGB}{251,154,153}
\definecolor{Paired-5}{RGB}{227,26,28}
\definecolor{Paired-8}{RGB}{253,191,111}
\definecolor{Paired-7}{RGB}{255,127,0}
\definecolor{Paired-10}{RGB}{202,178,214}
\definecolor{Paired-9}{RGB}{106,61,154}
\definecolor{Paired-12}{RGB}{255,255,153}
\definecolor{Paired-11}{RGB}{177,89,40}
\definecolor{Accent-1}{RGB}{127,201,127}
\definecolor{Accent-2}{RGB}{190,174,212}
\definecolor{Accent-3}{RGB}{253,192,134}
\definecolor{Accent-4}{RGB}{255,255,153}
\definecolor{Accent-5}{RGB}{56,108,176}
\definecolor{Accent-6}{RGB}{240,2,127}
\definecolor{Accent-7}{RGB}{191,91,23}
\definecolor{Accent-8}{RGB}{102,102,102}
\definecolor{Spectral-1}{RGB}{158,1,66}
\definecolor{Spectral-2}{RGB}{213,62,79}
\definecolor{Spectral-3}{RGB}{244,109,67}
\definecolor{Spectral-4}{RGB}{253,174,97}
\definecolor{Spectral-5}{RGB}{254,224,139}
\definecolor{Spectral-6}{RGB}{255,255,191}
\definecolor{Spectral-7}{RGB}{230,245,152}
\definecolor{Spectral-8}{RGB}{171,221,164}
\definecolor{Spectral-9}{RGB}{102,194,165}
\definecolor{Spectral-10}{RGB}{50,136,189}
\definecolor{Spectral-11}{RGB}{94,79,162}
\definecolor{Set1-1}{RGB}{228,26,28}
\definecolor{Set1-2}{RGB}{55,126,184}
\definecolor{Set1-3}{RGB}{77,175,74}
\definecolor{Set1-4}{RGB}{152,78,163}
\definecolor{Set1-5}{RGB}{255,127,0}
\definecolor{Set1-6}{RGB}{255,255,51}
\definecolor{Set1-7}{RGB}{166,86,40}
\definecolor{Set1-8}{RGB}{247,129,191}
\definecolor{Set1-9}{RGB}{153,153,153}
\definecolor{Set2-1}{RGB}{102,194,165}
\definecolor{Set2-2}{RGB}{252,141,98}
\definecolor{Set2-3}{RGB}{141,160,203}
\definecolor{Set2-4}{RGB}{231,138,195}
\definecolor{Set2-5}{RGB}{166,216,84}
\definecolor{Set2-6}{RGB}{255,217,47}
\definecolor{Set2-7}{RGB}{229,196,148}
\definecolor{Set2-8}{RGB}{179,179,179}
\definecolor{Dark2-1}{RGB}{27,158,119}
\definecolor{Dark2-2}{RGB}{217,95,2}
\definecolor{Dark2-3}{RGB}{117,112,179}
\definecolor{Dark2-4}{RGB}{231,41,138}
\definecolor{Dark2-5}{RGB}{102,166,30}
\definecolor{Dark2-6}{RGB}{230,171,2}
\definecolor{Dark2-7}{RGB}{166,118,29}
\definecolor{Dark2-8}{RGB}{102,102,102}
\definecolor{Reds-1}{RGB}{255,245,240}
\definecolor{Reds-2}{RGB}{254,224,210}
\definecolor{Reds-3}{RGB}{252,187,161}
\definecolor{Reds-4}{RGB}{252,146,114}
\definecolor{Reds-5}{RGB}{251,106,74}
\definecolor{Reds-6}{RGB}{239,59,44}
\definecolor{Reds-7}{RGB}{203,24,29}
\definecolor{Reds-8}{RGB}{165,15,21}
\definecolor{Reds-9}{RGB}{103,0,13}
\definecolor{Greens-1}{RGB}{247,252,245}
\definecolor{Greens-2}{RGB}{229,245,224}
\definecolor{Greens-3}{RGB}{199,233,192}
\definecolor{Greens-4}{RGB}{161,217,155}
\definecolor{Greens-5}{RGB}{116,196,118}
\definecolor{Greens-6}{RGB}{65,171,93}
\definecolor{Greens-7}{RGB}{35,139,69}
\definecolor{Greens-8}{RGB}{0,109,44}
\definecolor{Greens-9}{RGB}{0,68,27}
\definecolor{Blues-1}{RGB}{247,251,255}
\definecolor{Blues-2}{RGB}{222,235,247}
\definecolor{Blues-3}{RGB}{198,219,239}
\definecolor{Blues-4}{RGB}{158,202,225}
\definecolor{Blues-5}{RGB}{107,174,214}
\definecolor{Blues-6}{RGB}{66,146,198}
\definecolor{Blues-7}{RGB}{33,113,181}
\definecolor{Blues-8}{RGB}{8,81,156}
\definecolor{Blues-9}{RGB}{8,48,107}

\usecolortheme[named=bleuUni]{structure}

% Special commands : ddfrac and actionenv and compresslist
\newcommand\ddfrac[2]{\frac{\displaystyle #1}{\displaystyle #2}}

\newenvironment<>{varblock}[2][\textwidth]{%
  \setlength{\textwidth}{#1}
  \begin{actionenv}#3%
    \def\insertblocktitle{#2}%
    \par%
    \usebeamertemplate{block begin}}
  {\par%
    \usebeamertemplate{block end}%
  \end{actionenv}}

\newcommand{\compresslist}{ % Define a command to reduce spacing within itemize/enumerate environments, this is used right after \begin{itemize} or \begin{enumerate}
\setlength{\itemsep}{1pt}
\setlength{\parskip}{0pt}
\setlength{\parsep}{0pt}
}

\newcommand*\circled[1]{\tikz[baseline=(char.base)]{%
      \node[shape=circle,fill=bleuUni,inner sep=2pt] (char) {\textbf{\textcolor{white}{#1}}};}}

\newcommand{\itmsp}[1] {\setlength\itemsep{#1}}
%%%%%%%%%%%%%%%%%%%%%%%%%

%%%%% Beamer
\usepackage[bars]{beamerthemetree} % Beamer theme v 2.2
\mode<presentation>
\newcommand*\oldmacro{}%
\let\oldmacro\insertshorttitle%
\renewcommand*\insertshorttitle{%
 \oldmacro\hspace{0pt plus 1 filll}%
\insertframenumber\,/\,\inserttotalframenumber}
\setbeamertemplate{footline}[frame number]
\setbeamersize{text margin left=10pt,text margin right=10pt}
\setbeamerfont{frametitle}{size=\small}
\setbeamertemplate{frametitle}{ \nointerlineskip %
\begin{beamercolorbox}[wd=\paperwidth,ht=2.2ex,dp=.9ex,left]{frametitle} %
                       \hspace*{2ex}\strut\bfseries\color{bleuUni!15!white}\insertframetitle\strut\par %
\end{beamercolorbox}}

%\setbeamerfont{headline}{size=\footnotesize}

% \usepackage{animate}
% \usepackage{multimedia}
\usetheme{Ilmenau} % Beamer theme v 3.0
\setbeamercolor{section in head/foot}{bg=marronUni}
\useinnertheme{circles} %rectangle bullet points instead of circle ones
\usepackage{beamerthemebars}
\beamertemplatenavigationsymbolsempty
%\setbeamercolor{navigation symbols dimmed}{fg=red!80!black}
%\setbeamercolor{navigation symbols}{fg=red!80!black}
%%%%%%%%%%%%%%%%%%%%%%%%%

\setbeamertemplate{headline}{%
\begin{beamercolorbox}[colsep=1.5pt]{upper separation line head}
\end{beamercolorbox}
\begin{beamercolorbox}{section in head/foot}
    \vskip2pt\insertsectionnavigationhorizontal{\paperwidth}{}{}\vskip2pt
\end{beamercolorbox}%
\begin{beamercolorbox}[ht=10pt]{subsection in head/foot}%
    \vskip2pt\insertsubsectionnavigationhorizontal{\paperwidth}{}{}\vskip2pt
\end{beamercolorbox}%
\begin{beamercolorbox}[colsep=1.5pt]{lower separation line head}
\end{beamercolorbox}
}

\setbeamertemplate{blocks}[rounded][shadow=false]

%%%%% Contents each new sec and subsec
\AtBeginSection[]
{
  %\ifnumcomp{\value{section}}{=}{1}{}{
    \begin{frame}[c]{Plan}
      \centering
      \tableofcontents[
          currentsection,
          hideothersubsections
          %subsectionstyle=show/hide
      ]
    \end{frame}
  %}
}

\AtBeginSubsection[]
{
  \begin{frame}[c]{Plan}
    \tableofcontents[
      currentsection,
      sectionstyle=show/shaded,
      subsectionstyle=show/shaded/hide
    ]
  \end{frame}
}


% MIPP ------------------------------------------------------------------------

\newcommand{\MIPP}{MIPP\xspace}
\newcommand{\longMIPP}{MyIntrinsics++\xspace}
\newcommand{\TSIMD}{T-SIMD\xspace}
\newcommand{\xsimd}{xsimd\xspace}
\newcommand{\simdpp}{simdpp\xspace}
\newcommand{\Vc}{Vc\xspace}
\newcommand{\VCL}{VCL\xspace}
\newcommand{\BoostSIMD}{Boost.SIMD\xspace}
\newcommand{\bSIMD}{bSIMD\xspace}

\newcommand{\Cxx}{\texttt{C++}\xspace}
\newcommand{\Cxy}[1]{\texttt{C++{#1}}\xspace}
\newcommand{\CppUnit}{\texttt{CppUnit}\xspace}

% \setlength{\textfloatsep}{5pt}
% \setlength{\parsep}{0pt}
% \setlength{\parskip}{0pt}

\newcommand{\cmark}{\textcolor{btfGreen}{\ding{51}}}
\newcommand{\xmark}{\textcolor{btfRed}{\ding{55}}}
\newcommand{\Arikan}{Ar{\i}kan\xspace}

\renewcommand{\footnotesize}{\fontsize{7}{8.4}}
% % \renewcommand{\footnotesize}{\scriptsize}

\DeclareMathOperator*{\sign}{sign}