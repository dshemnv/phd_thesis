%!TEX root = ../my_thesis.tex
\graphicspath{{main/chapter3/fig/}}

\chapter{Embedded Domain Specific Language for the SDR}

\section{Introduction}

\begin{itemize}
  \item Algorithmes de synchronisation avant la démodulation
  \item Tâches séquentielles (avec état)
\end{itemize}

\section{Dataflow Model}

\begin{itemize}
  \item géneral dataflow
  \item synchronous dataflow
  \item cyclo static dataflow
\end{itemize}

\section{Related Works}

\subsection{Dedicated Languages}

\subsection{Ad Hoc Solutions}

\subsection{GNU Radio}

\section{Runtime Embedded DSL}

\subsection{Serial Blocks}

\begin{itemize}
  \item tâches = les traitements (effectués sur des trames)
  \item modules = les données internes à une ou plusieurs tâches
  \item sockets = les données échangées entre les tâches
  \item boucles = permet de répéter une sous-séquence
  \item routeurs = permet d'aiguiller vers différentes sous-séquences
\end{itemize}

\subsection{Parallel Blocks}

\begin{itemize}
  \item séquence = un enchaînement de tâches dont l'ordre est défini par les
        connections entre les sockets
  \item pipeline
\end{itemize}

\subsection{Rules}

\begin{itemize}
  \item une socket d'entrée ne peut être connectée qu'à une seule socket de
        sortie
  \item toutes les sockets d'entrée d'une tâche doivent être connectées pour
        qu'elle puisse s'exécuter
  \item une socket de sortie peut être connectée à aucune, une ou plus d'une
        socket d'entrée
  \item l'ordonnancement d'une séquence est implicitement défini par le binding,
        quand la dernière socket d'input d'une tâche est connectée, alors on va
        passer aux tâches suivantes
\end{itemize}

\section{Implementation}

\subsection{Clone}

\subsection{Processes}

\subsection{0-Copy}

\section{Application on the DVB-S2 Standard}

\subsection{Transmitter}

\subsection{Receiver}

\subsection{Evaluations}

\subsection{Discussion}