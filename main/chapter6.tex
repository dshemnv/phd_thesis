%!TEX root = ../my_thesis.tex

\graphicspath{{main/chapter6/fig/}}

\chapter{Embedded Domain Specific Language for the Software Defined Radio}

% \vspace*{\fill}
\minitoccustom
% \vspace*{\fill}

\section{Introduction}

\begin{itemize}
  \item \xmark~Algorithmes de synchronisation avant la démodulation
  \item \xmark~Tâches séquentielles (avec état)
\end{itemize}

It is now commonplace to state that Humanity has entered the era of
communication. The fifth generation of mobile networks (5G) is a perfect
illustration of this: all kinds of objects will increasingly use communication
technology to exchange information on the Internet of Things (IoT).

Despite their variety, all communication systems are based on a common
abstract model proposed by Claude Shannon. In his seminal
paper~\cite{Shannon1948}, he proposed to model a communication system with five
components: an information source, a transmitter, a channel, a receiver and a
destination.

Traditionally those communication systems are implemented on dedicated hardware
(ASIC) targeting high throughputs, low latencies and energy efficiency.
However, the implementations of such solutions have a long time to market, are
expensive and are specific by nature~\cite{Palkovic2010,Palkovic2012}.

The new standards like the 5G are coming with very large specifications and
multiple possible configurations~\cite{ETSI2018}. Small objects that need to
communicate very few data at low rates will live together with 4K video
streaming for mobile phone games which will required high throughputs as well as
low latencies~\cite{Rost2014}.

To meet those various specifications the transceivers will have to be able to
adapt quickly to new configurations. There is a growing need for flexible,
re-configurable and programmable solutions. To match those constraints, there is
a growing interest for the SDR which consists in processing both the Physical
(PHY) and Medium Access Control (MAC) layers in software~\cite{Mitola1993}, as
opposed to the traditionally hardware-based solutions. Short time to market,
lower development costs, interoperability, readiness to adapt to future updates
and new protocols are the mains advantages of the SDR~\cite{Akeela2018}.

SDR can be implemented on various targets like Field Programmable Gate Arrays
(FPGAs)~\cite{Coulton2004,Skey2006,Dutta2010,Shaik2013,Maheshwarappa2015,
Nivin2016}, Digital Signal Processors (DSPs)~\cite{Kaur2008,Karlsson2013,
Shaik2013} or General Purpose Processors (GPPs)~\cite{Yoge2012,Bang2014,
Meshram2019,Grayver2020}. This thesis focuses on GPPs targets as they are widely
spread, affordable and easy to program. More precisely, only the CPUs are
considered. Many elementary blocks of digital communication systems has been
optimized to run fast on x86 and ARM CPUs~\cite{Cassagne2015c,Cassagne2016a,
Cassagne2016b,Cassagne2018,Leonardon2019,Ghaffari2019} and have been packaged in
AFF3CT~\cite{Cassagne2017a,Cassagne2019a}, our dedicated open source toolbox.
Even if there is some interesting results in term of throughput on
GPUs~\cite{Xianjun2013,Li2014,LeGal2014a,Giard2016b,Keskin2017a}, the achieved
latency on this architecture is still too high to meet real time constraints and
cannot compete with existing CPU implementations~\cite{LeGal2015a,Cassagne2015c,
Giard2016b,Cassagne2016a,LeGal2017,Leonardon2019,LeGal2019a}. This is mainly due
to data transfers between the host (CPUs) and the device (GPUs), and to the
intrinsic nature of the GPU architecture which is not optimized for latency
efficiency.

\section{Dataflow Model}

\begin{itemize}
  \item \xmark~géneral dataflow
  \item \xmark~synchronous dataflow
  \item \xmark~cyclo static dataflow
\end{itemize}

The Shanon's communication model presented before can be refined in a way
that the transmitter and the receiver are decomposed into many processing
blocks. Those blocks are mainly connected to each other in an ordered graph.
This perfectly matches the dataflow model~\cite{Dennis1980,Ackerman1982}: the
blocks are the filters and the binding links between the blocks represent the
data movements. The dataflow model allows to describe the system from a high
level point of view and to perform optimizations independently of the system
designer. In the case of the SDR, simpler models than the generalized dataflow
can be used like the synchronous dataflow~\cite{Lee1987} or the cyclo-static
dataflow~\cite{Engels1994,Bilsen1995}. This allows to perform aggressive
simplifications like a static scheduling of the filters
execution~\cite{Parks1995}.

\section{Related Works}

\subsection{Dedicated Languages}

Many languages dedicated to streaming applications have been
developed~\cite{Buck2004,Amarasinghe2005,Liao2006,Black-Schaffer2010,Glitia2010,
Thies2010,DeOliveiraCastro2017}. Streaming applications are most of the time
represented with the dataflow model and the dedicated languages often support
the general or the cyclo-static dataflow model. They also very often come with
automatic parallelism mechanisms like pipelining and forks/joins.

\subsection{Ad Hoc Solutions}

Some preliminary works has been made to support a part of a communication
standard (DVB-S2 receiver) on CPUs and shows that it is possible to reach high
throughputs (10~Gb/s) using a small cluster~\cite{Grayver2020}. Those works
demonstrate the feasibility but lack an overall methodology. Indeed, the
approach is ad hoc for fixed parameters of the standard and a single CPU.

\subsection{GNU Radio}

However, there are few solutions targeting specifically the SDR sub-domain yet.
The most famous is GNU Radio~\cite{GNURadio} which is open source and largely
adopted by the community. The software is bundled with a large variety of
algorithms existing in real life systems. GNU Radio models the digital
communication systems at the symbol level: this philosophy is very close to the
algorithms descriptions that can be found in the signal community literature.
Thus, it allows quick implementation of new algorithms. Still, we show this is a
limitation to meet high throughputs and low latencies constraints on current
GPPs architectures, this is why, in this thesis, we propose, through AFF3CT, a
runtime Embedded Domain Specific Language (EDSL) working on sets of symbols
(aka frames). AFF3CT is a form of the synchronous dataflow model specialized to
the relevant characteristics of FEC communication chains enabling to perform
more aggressive optimizations than GNU Radio, at the cost of lower generality.

\section{Runtime Embedded DSL}

\subsection{Serial Blocks}

\begin{itemize}
  \item \xmark~tâches = les traitements (effectués sur des trames)
  \item \xmark~modules = les données internes à une ou plusieurs tâches
  \item \xmark~sockets = les données échangées entre les tâches
  \item \xmark~boucles = permet de répéter une sous-séquence
  \item \xmark~routeurs = permet d'aiguiller vers différentes sous-séquences
\end{itemize}

\subsection{Parallel Blocks}

\begin{itemize}
  \item \xmark~séquence = un enchaînement de tâches dont l'ordre est défini par les
    connections entre les sockets
  \item \xmark~pipeline
\end{itemize}

\subsection{Rules}

\begin{itemize}
  \item \xmark~une socket d'entrée ne peut être connectée qu'à une seule socket de
    sortie
  \item \xmark~toutes les sockets d'entrée d'une tâche doivent être connectées pour
    qu'elle puisse s'exécuter
  \item \xmark~une socket de sortie peut être connectée à aucune, une ou plus d'une
    socket d'entrée
  \item \xmark~l'ordonnancement d'une séquence est implicitement défini par le binding,
    quand la dernière socket d'input d'une tâche est connectée, alors on va
    passer aux tâches suivantes
\end{itemize}

\section{Implementation}

\subsection{Clone}

\subsection{Processes}

\subsection{0-Copy}

\section{Application on the DVB-S2 Standard}

\subsection{Transmitter}

\subsection{Receiver}

\subsection{Evaluations}

\subsection{Discussion}
