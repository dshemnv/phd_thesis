%!TEX root = ../my_thesis.tex

\graphicspath{{main/introduction/fig/}}

\chapter*{Introduction}
\markboth{Introduction}{Introduction}
\addcontentsline{toc}{chapter}{Introduction}

\subsubsection*{Digital Communications}

Man has sought to communicate from time immemorial. Since then, man has always
been seeking for more efficient ways to extend his communication possibilities.
Nowadays, with the advent of the Internet, the digital communications represent
the last technology advances to communicate world-wide. For instance, digital
communications enable both live video transmission and the use of a messaging
system. With the growing number of users and needs, the digital communication
systems are the subject of an important area of research. New digital
communication systems have to be able to match high throughput and low latency
constraints as well as acceptable energy consumption levels.

Traditionally, digital communication transmitters and receivers are implemented
in hardware, on dedicated chips. The required signal processing algorithms are
often very specific and repetitive. Thus, they are good candidates for
specialized architectures. However, with the growing number of use cases and
telecommunication standards, these algorithms are evolving and are becoming more
and more heterogeneous. In this context, it becomes interesting to consider
software implementations on generic architectures. This type of programmable
architectures is available in computers and is commonly referred as the Central
Process Unit (CPU). The CPUs are General Purpose Processors (GPP) that can adapt
to various types of algorithms.

\subsubsection*{Computer Architecture}

Improving the computational and the energy efficiency of these processors is one
of the main concern in computer science. As they are largely adopted for many
use cases, the CPUs take advantage of the best manufacturing processes. Thanks
to their pipelined architecture they are able to reach very high processing
frequencies ranging from 1 to 4 GHz. They also come with dedicated memory caches
that enable efficient spatial and temporal reuse of data. Nowadays, the
computational efficiency of the CPUs relies on two main parallel techniques. The
first one is the multi-core architecture: it consists in duplicating the
hardware of the ``CPU'' in multiple instances called cores. These cores are
mostly independent from each other. They are packaged together in the same chip
(called the CPU) and generally they share a fast memory: the last level cache.
The second parallel technique is the vectorized instructions. These types of
instructions are available in each core and are able to perform the same
operation on a chunk of data. This is also known as the Single Instruction
Multiple Data (SIMD) architectural model.

From an energy point of view, it is clear that CPUs are not directly competitive
compared to dedicated architectures. Their large number of instructions enables
efficient implementations of many algorithms but this is also a limitation when
targeting specific applications. Many transistors are unused and consume a
non-negligible amount of energy. On the other hand, the main strength of the GPP
architectures comes from their abilities to be used programmatically with high
level languages. Consequently, the time required to implement new algorithms is
much shorter on GPPs than on dedicated hardwares. However, even with reduced
implementation time, it is still challenging to design algorithms that take
effectively advantage of the CPUs parallelism levels.

\subsubsection*{Hardware Abstraction and Software}

The ever growing complexity of processors motivates new hardware description
level abstractions. Even if it is still possible to write assembly codes, one
should agree that this is not adapted to real-size applications. Moreover, in
general the designers of an application are not familiar with the specificities
of the CPU architectures. Thus, it becomes important to propose new models (or
abstractions) on top of the hardware. It enables the efficient use of processors
to the largest number of people. To this purpose, dedicated compilers,
languages and libraries are an important research area in computer science.

In the design of digital communication systems, it is now common to rely on
software implementations for the evaluation and the validation of signal
processing algorithms. These evaluation and validation steps consist in
the simulation of the whole communication system. Generally these type of
simulations are implemented by signal processing experts with high level
programing languages like MATLAB\R or Python. However, with the growing
complexity of the digital communication systems, these simulations are becoming
more and more compute intensive. Using high level programming languages is a
limiting factor because it can lead to large restitution times (from days to
weeks). Thus, high performance implementations based on lower level programming
languages are considered.

Moreover, software implementations are also considered for real-time uses. Their
flexibility and reduced time to market are becoming more and more attractive.
Indeed, dedicated hardware solutions require specific skills and are achieved by
electronics specialists. In general, signal processing experts do not focus on
implementing efficient software or hardware solutions. The dialog between the
two communities is not always simple as they have different concerns.

The purpose of this thesis is to ease the overall design of digital
communication systems, from the conception to the implementation. Dedicated
tools and interfaces are proposed to help the signal processing experts to
design fast software implementations. In general, this type of software
implementations are a good start to better understand the algorithms hotspots.
From this point, electronics specialists can improve proposed software solutions
and, if necessary, design adapted hardware implementations.

\subsubsection*{Contributions}

In this thesis we propose to study the most time consuming algorithms of digital
communication systems, to adapt and optimize them on General Purpose Processors
(GPPs) like the CPUs. Most of the current digital communication standards
require the implementation of such algorithms. The long simulation times and the
real-world application requirements make it desirable to have portable,
flexible, high throughput and low latency implementations. The proposed high
performance implementations are shown to be competitive with the
state-of-the-art ones. Contrary to the previous works, this thesis strives to
extract generic methodologies and strategies common to the majority of the
signal processing algorithms. The proposed implementations try to be as flexible
as possible without sacrificing too much the performance.

The signal processing algorithms come with various characteristics. Thus, it is
of interest to be able to manage this algorithmic heterogeneity. In this work,
to enable code reuse, similarities are identified into this zoo of algorithms.
The various implementations have been packaged, categorized and organized in
one single software library, namely \AFFECT. These implementations cohabit
together thanks to well-defined interfaces and an adapted software architecture
based on the Object-Oriented Programming (OOP) paradigm.

Another important concern of this work is the ability to reproduce the
scientific results. Indeed, all the proposed implementations are regrouped in
\AFFECT which is an open-source software. Specific strategies have been operated
to minimize the possible regressions based on the digital communication systems
characteristics. These non-regression strategies are automated. They ensure that
the source code remains stable even if many contributors are working together.

These contributions have been the topic of several scientific publications in
both the computer science and the signal processing communities. They are listed
at the end of the manuscript in the
``\hyperref[chap:publi]{Personal Publications}'' section. As a convention in the
document, the numeric citations are contributions of this thesis while the
alphabetic citations refer to other works in the literature.

\subsubsection*{Dissertation Organization}

This dissertation is organized in five chapters. The first chapter describes the
context and details the objectives. The next chapters present our contributions.

In Chapter~\ref{chap:ctx}, the digital communication systems are detailed. Then,
the most time consuming part of these systems is presented, namely the channel
decoders. After that, the applicative contexts of this thesis are defined. The
two main ones are the functional simulation and the Software-Defined Radio
(SDR). The functional simulation enables the evaluation and the validation of
different digital communication systems while the SDR corresponds to the
real-time execution of these systems in software. Finally, the main problematics
are exposed.

In Chapter~\ref{chap:opt}, new efficient implementations of the decoders are
proposed. First, an overall portable methodology is detailed to meet the high
throughput constraint required by both the simulations and the real-time
systems. This methodology is based on the Single Instruction Multiple Data
(SIMD) model implemented in most of the current CPUs. Depending on how the CPU
SIMD instructions are used, it is possible to maximize the throughput or the
latency of the implemented decoding algorithms. Then, specific optimized
implementations are detailed for each decoding algorithm. These implementations
focus on maximizing the flexibility, high throughput and low latency. Depending
on the implementations, some compromises have to be made and some of these
characteristics can be maximized unbeknownst to others.

In Chapter~\ref{chap:aff3ct}, \AFFECT, our toolbox dedicated to the forward
error correction (FEC) algorithms is presented. \AFFECT is unique in the domain
and it is composed by many algorithm implementations (including those presented
in Chapter~\ref{chap:opt}). \AFFECT is the software that enables the signal
processing algorithms heterogeneity thanks to a robust software architecture
based on well-defined and coherent interfaces. It enables reproducibility of the
results as it is open-source and extensively tested. \AFFECT also contains a
parallel functional simulator and enables extensive exploration/validation of
existing or new algorithms on a large combination of parameters.

\newpage
In Chapter~\ref{chap:eval}, the efficient algorithm implementations proposed
in Chapter~\ref{chap:opt} are evaluated and compared with the state-of-the-art.
The FEC software decoders hall of fame is introduced to summarize and to compare
the proposed contributions with previous works in the literature. Some metrics
are defined for ease of comparison. These metrics focus on normalized
throughput, proper use of hardware and energy efficiency. Finally, the \AFFECT
simulator efficiency is demonstrated on various multi-core CPUs and on a
multi-node cluster.

In Chapter~\ref{chap:sdr}, a new embedded Domain Specific Language (eDSL) for
the SDR is presented. The \AFFECT software suite is enriched with new blocks
dedicated to the efficient implementation of real-time digital communication
systems on multi-core CPUs. These blocks enable automatic parallelism. As an
example of use, a full physical layer of the DVB-S2 standard has been
implemented. All the digital processing are performed with \AFFECT while the
radio frequency communications is achieved with Universal Software Radio
Peripherals (USRPs). The results match the satellite real-time constraints.
