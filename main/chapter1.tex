%!TEX root = ../my_thesis.tex
\chapter{Context and State-of-the-Art}

\section{Introduction}

\begin{itemize}
  \item Shannon~\cite{Shannon1948}: système de communication (source -> émetteur
        -> canal -> récepteur -> destination)
  \item utilisé partout dans notre monde (télé, téléphone, internet, satellites,
        etc.)
  \item zoom sur l'émetteur (source -> codage canal -> modulation)
  \item zoom sur le récepteur (démodulation -> décodage canal -> destination)
  \item introduction au codage canal, nécessaire pour mieux résister aux
        perturbations dues à la traversée du signal dans un environnement
        physique (le canal)
  \item modulation: représentation d'une information numérique en analogique
        adaptée au canal
  \item couche physique (PHY 1) du modèle OSI
  \item couche physique traditionnellement implémentée en hardware (ASIC)
  \item récepteur gourmand en calcul (plus particulière l'algorithme de décodage
        et un peu la démodulation)
\end{itemize}

\section{Channel Coding}

\begin{itemize}
  \item 3 grandes familles de code approchent la capacité du canal (limite de
        Shannon) : LDPC, Polar et Turbo, les décrire et les comparer
\end{itemize}

\subsection{Channel Models}

\begin{itemize}
  \item présenter modèles de canal (BEC, BSC, AWGN)
  \item introduire LLRs
\end{itemize}

\subsection{LDPC codes}

\subsection{Turbo codes}

\subsection{Polar codes}

\section{Simulation}

\begin{itemize}
  \item besoin d'implémentations logicielles pour estimer les performances des
        algorithmes de décodage avant de les implémenter en hardware
        (Monte-Carlo sur canal AWGN)
  \item Introduire BE, FE, BER, FER
\end{itemize}

\section{Base Station and Cloud-RAN}

\begin{itemize}
  \item besoin d'implémentations software pour gagner en flexibilité et réduire
        les coûts par rapport au hardware dans les stations de base par ex.
        (SDR)
\end{itemize}

\section{Software Defined Radio (SDR)}