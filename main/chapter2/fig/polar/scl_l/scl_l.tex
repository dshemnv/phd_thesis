\documentclass{standalone}

\usepackage[T1]{fontenc}
\usepackage[utf8]{inputenc}
\usepackage{eulervm}
\usepackage{amsmath}
\usepackage{pgfplots}
\usepackage{bm}

\pgfplotsset{compat=newest}
\usepgfplotslibrary{colorbrewer}
\usepgfplotslibrary{groupplots}
\usetikzlibrary{matrix, positioning}

\usepackage{color}

\definecolor{Comment}{RGB}{97,161,176}

\definecolor{btfGreen}{RGB}{51,160,44}
\definecolor{btfRed}{RGB}{190,60,90}

\definecolor{bleuUni}{RGB}{0, 157, 224}
\definecolor{marronUni}{RGB}{68, 58, 49}
\definecolor{grayMarronUni}{RGB}{60, 60, 60}
\definecolor{grayBleuUni}{RGB}{118, 118, 118}

\definecolor{bluecite}{HTML}{009DE0}

\definecolor{Paired-2}{RGB}{166,206,227}
\definecolor{Paired-1}{RGB}{31,120,180}
\definecolor{Paired-4}{RGB}{178,223,138}
\definecolor{Paired-3}{RGB}{51,160,44}
\definecolor{Paired-6}{RGB}{251,154,153}
\definecolor{Paired-5}{RGB}{227,26,28}
\definecolor{Paired-8}{RGB}{253,191,111}
\definecolor{Paired-7}{RGB}{255,127,0}
\definecolor{Paired-10}{RGB}{202,178,214}
\definecolor{Paired-9}{RGB}{106,61,154}
\definecolor{Paired-12}{RGB}{255,255,153}
\definecolor{Paired-11}{RGB}{177,89,40}
\definecolor{Accent-1}{RGB}{127,201,127}
\definecolor{Accent-2}{RGB}{190,174,212}
\definecolor{Accent-3}{RGB}{253,192,134}
\definecolor{Accent-4}{RGB}{255,255,153}
\definecolor{Accent-5}{RGB}{56,108,176}
\definecolor{Accent-6}{RGB}{240,2,127}
\definecolor{Accent-7}{RGB}{191,91,23}
\definecolor{Accent-8}{RGB}{102,102,102}
\definecolor{Spectral-1}{RGB}{158,1,66}
\definecolor{Spectral-2}{RGB}{213,62,79}
\definecolor{Spectral-3}{RGB}{244,109,67}
\definecolor{Spectral-4}{RGB}{253,174,97}
\definecolor{Spectral-5}{RGB}{254,224,139}
\definecolor{Spectral-6}{RGB}{255,255,191}
\definecolor{Spectral-7}{RGB}{230,245,152}
\definecolor{Spectral-8}{RGB}{171,221,164}
\definecolor{Spectral-9}{RGB}{102,194,165}
\definecolor{Spectral-10}{RGB}{50,136,189}
\definecolor{Spectral-11}{RGB}{94,79,162}
\definecolor{Set1-1}{RGB}{228,26,28}
\definecolor{Set1-2}{RGB}{55,126,184}
\definecolor{Set1-3}{RGB}{77,175,74}
\definecolor{Set1-4}{RGB}{152,78,163}
\definecolor{Set1-5}{RGB}{255,127,0}
\definecolor{Set1-6}{RGB}{255,255,51}
\definecolor{Set1-7}{RGB}{166,86,40}
\definecolor{Set1-8}{RGB}{247,129,191}
\definecolor{Set1-9}{RGB}{153,153,153}
\definecolor{Set2-1}{RGB}{102,194,165}
\definecolor{Set2-2}{RGB}{252,141,98}
\definecolor{Set2-3}{RGB}{141,160,203}
\definecolor{Set2-4}{RGB}{231,138,195}
\definecolor{Set2-5}{RGB}{166,216,84}
\definecolor{Set2-6}{RGB}{255,217,47}
\definecolor{Set2-7}{RGB}{229,196,148}
\definecolor{Set2-8}{RGB}{179,179,179}
\definecolor{Dark2-1}{RGB}{27,158,119}
\definecolor{Dark2-2}{RGB}{217,95,2}
\definecolor{Dark2-3}{RGB}{117,112,179}
\definecolor{Dark2-4}{RGB}{231,41,138}
\definecolor{Dark2-5}{RGB}{102,166,30}
\definecolor{Dark2-6}{RGB}{230,171,2}
\definecolor{Dark2-7}{RGB}{166,118,29}
\definecolor{Dark2-8}{RGB}{102,102,102}
\definecolor{Reds-1}{RGB}{255,245,240}
\definecolor{Reds-2}{RGB}{254,224,210}
\definecolor{Reds-3}{RGB}{252,187,161}
\definecolor{Reds-4}{RGB}{252,146,114}
\definecolor{Reds-5}{RGB}{251,106,74}
\definecolor{Reds-6}{RGB}{239,59,44}
\definecolor{Reds-7}{RGB}{203,24,29}
\definecolor{Reds-8}{RGB}{165,15,21}
\definecolor{Reds-9}{RGB}{103,0,13}
\definecolor{Greens-1}{RGB}{247,252,245}
\definecolor{Greens-2}{RGB}{229,245,224}
\definecolor{Greens-3}{RGB}{199,233,192}
\definecolor{Greens-4}{RGB}{161,217,155}
\definecolor{Greens-5}{RGB}{116,196,118}
\definecolor{Greens-6}{RGB}{65,171,93}
\definecolor{Greens-7}{RGB}{35,139,69}
\definecolor{Greens-8}{RGB}{0,109,44}
\definecolor{Greens-9}{RGB}{0,68,27}
\definecolor{Blues-1}{RGB}{247,251,255}
\definecolor{Blues-2}{RGB}{222,235,247}
\definecolor{Blues-3}{RGB}{198,219,239}
\definecolor{Blues-4}{RGB}{158,202,225}
\definecolor{Blues-5}{RGB}{107,174,214}
\definecolor{Blues-6}{RGB}{66,146,198}
\definecolor{Blues-7}{RGB}{33,113,181}
\definecolor{Blues-8}{RGB}{8,81,156}
\definecolor{Blues-9}{RGB}{8,48,107}


\begin{document}
  \begin{tikzpicture}
    \begin{semilogyaxis}[/pgfplots/table/ignore chars={|}, %footnotesize,
                         width=1.0\linewidth, height=0.70\linewidth,
                         xlabel=$E_b/N_0~\text{(dB)}$, ylabel=FER,  grid=both, grid style={gray!30},
                         xmin=1.0, xmax=3.5, xtick={0,0.5,1,...,3.5},
                         tick align=outside, tickpos=left, %legend style={at={(0.5,-0.2)},anchor=north},
                         legend pos=south west, legend columns=2]
      \addplot[mark=diamond*,  Paired-9,     semithick, dashed, mark options={solid}                                                    ] table [x=Eb/N0, y=FER] {dat/L_1.txt};   \label{plot:line0}
      \addplot[mark=*,         Paired-1,     semithick, dashed, mark options={solid}                                                    ] table [x=Eb/N0, y=FER] {dat/L_2.txt};   \label{plot:line1}
      \addplot[mark=triangle*, Paired-3,     semithick, dashed, mark options={solid}                                                    ] table [x=Eb/N0, y=FER] {dat/L_4.txt};   \label{plot:line2}
      \addplot[mark=square*,   Paired-7,     semithick, dashed, mark options={solid}                                                    ] table [x=Eb/N0, y=FER] {dat/L_8.txt};   \label{plot:line3}
      \addplot[mark=pentagon*, Paired-11,    semithick, dashed, mark options={solid}                                                    ] table [x=Eb/N0, y=FER] {dat/L_16.txt};  \label{plot:line4}
      \addplot[mark=otimes*,   Paired-5,     semithick, dashed, mark options={solid}, every mark/.append style={solid, fill=Paired-5!40}] table [x=Eb/N0, y=FER] {dat/L_32.txt};  \label{plot:line5}
      \addplot[mark=pentagon,  Paired-11!70, semithick, dashed, mark options={solid}                                                    ] table [x=Eb/N0, y=FER] {dat/L_64.txt};  \label{plot:line6}
      \addplot[mark=square,    Paired-7!70,  semithick, dashed, mark options={solid}                                                    ] table [x=Eb/N0, y=FER] {dat/L_128.txt}; \label{plot:line7}

      % \legend{$N=2^{8~}$ $\texttt{SCL}_{L=\text{ }32}$, \texttt{SC},
      %         $N=2^{12}$ $\texttt{SCL}_{L=\text{ }32}$, \texttt{SC},
      %         $N=2^{16}$ $\texttt{SCL}_{L=\text{ }32}$, \texttt{SC},
      %         $N=2^{20}$ $\texttt{SCL}_{L=\text{ }32}$, \texttt{SC},
      %         $N=2^{20}$ $\texttt{SCL}_{L=       128}$}

      % create a (dummy) coordinate where we want to place the legend
      %
      % (The matrix cannot be placed inside the `axis' environment
      %  directly, because then a catcode error is raised.
      %  I guess that this is caused by the `matrix of nodes' feature)
      % \coordinate (legend) at (axis description cs:0.97,0.03);
      \coordinate (legend) at (axis description cs:0.985,0.33);
    \end{semilogyaxis}
    % create the legend matrix which is placed at the created (dummy) coordinate
    % and recall the plot specification using the `\ref' command
    %
    % adapt the style of that node to your needs
    % (e.g. if you like different spacings between the rows or columns
    %  or a fill color)
    \matrix [
        draw,
        matrix of nodes,
        anchor=south east,
        fill=white
    ] at (legend) {
                         &   $L$  & $\mathcal{T}_{i}$ \footnotesize{(Mb/s)} \\
        \ref{plot:line0} &   $1$  & 215.15  \\
      % \ref{plot:line0} &   $1$  &  50.74  \\
        \ref{plot:line1} &   $2$  &  26.67  \\
        \ref{plot:line2} &   $4$  &  12.54  \\
        \ref{plot:line3} &   $8$  &   6.00  \\
        \ref{plot:line4} &  $16$  &   2.96  \\
        \ref{plot:line5} &  $32$  &   1.40  \\
        \ref{plot:line6} &  $64$  &   0.66  \\
        \ref{plot:line7} & $128$  &   0.29  \\
    };
  \end{tikzpicture}
\end{document}