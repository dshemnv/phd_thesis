%!TEX root = ../my_thesis.tex

%
%   Use this file for your own custom packages, command-definitions, etc...
%

% Rubber configuration --------------------------------------------------------
% rubber: setlist arguments --shell-escape
% Packages --------------------------------------------------------------------
\usepackage[english]{minitoc} % rappel du plan du chapitre en cours
\usepackage{array}
\usepackage{longtable} % pour la liste des notations
\usepackage[nomain,nonumberlist,acronym,nogroupskip=true,nopostdot]{glossaries} % glossaire avec accronymes
\usepackage[toc,page]{appendix}
\usepackage{listings}
\usepackage[english,onelanguage,ruled,linesnumbered,vlined]{algorithm2e} % TODO Re enable this and changes algorithms
% \usepackage{algorithm, algorithmic}
\usepackage[
	labelnumber  = true,
	backend      = biber,
	style        = alphabetic,
	citestyle    = alphabetic,
	maxnames     = 99,
	defernumbers = true,
	isbn         = false,
	doi          = true,
	backref      = true]{biblatex}
\addbibresource{tail/biblio.bib}
\usepackage{verbatim}
\usepackage{booktabs}
\usepackage{multirow}
\usepackage{makecell}
\usepackage{multicol}
\usepackage{xspace}
\usepackage{pifont}
\usepackage{hyperref}
\usepackage[cache=false]{minted} % beautiful source code
% \usepackage{sourcecodepro} % beautiful source code (works with 'minted')
\usepackage{chngcntr}
\usepackage{bm} % bold math
\usepackage{geometry} % reduce margins
\usepackage{adjustbox} % rotate big table 90 degree
\usepackage[strict]{changepage}
\usepackage{tipa} % to make the phonetic "ə" for the AFF3CT pronunciation
\usepackage{subcaption} % multi figures
\usepackage{siunitx} % SI units
\usepackage{todonotes}
\usepackage{import}
% \usepackage{asymptote} % Nice figures with asymptote
% \def\asydir{asy_cache}

% Parameters ------------------------------------------------------------------
\setcounter{secnumdepth}{3} % package 'minitoc'
\setcounter{tocdepth}{2} % package 'minitoc'
\newlength{\Oldarrayrulewidth} % package 'array'
\setminted{fontsize=\small,framesep=2mm,tabsize=4} % package 'minted'
\usemintedstyle{friendly} % package 'minted'
\counterwithin{listing}{chapter} % package 'chngcntr'
% \counterwithin{algocf}{chapter} % package 'chngcntr'
% \SetKwComment{Comment}{$\triangleright$\ }{} % package 'algorithm2e'
% \newcommand\mycommfont[1]{\small\ttfamily\textcolor{Comment}{#1}} % package 'algorithm2e'
% \SetCommentSty{mycommfont} % package 'algorithm2e'
\renewcommand*{\bibfont}{\footnotesize} % package 'biblatex'
\renewcommand*{\labelalphaothers}{\textsuperscript{+}} % package 'biblatex'


% add column types for big table
\newcolumntype{L}[1]{>{\raggedright\let\newline\\\arraybackslash\hspace{0pt}}m{#1}}
\newcolumntype{C}[1]{>{\centering\let\newline\\\arraybackslash\hspace{0pt}}m{#1}}
\newcolumntype{R}[1]{>{\raggedleft\let\newline\\\arraybackslash\hspace{0pt}}m{#1}}
\newlength{\simcolwidth}
\setlength{\simcolwidth}{0.5cm}

% delete title from accronyms
\renewcommand{\glossarysection}[2][]{}


% configure a custom mini table of contents (% package 'minitoc')
\renewcommand{\mtctitle}{}
\tightmtctrue
\nomtcrule
\newcommand{\minitoccustom}{\vspace*{1em}{\color{bleuUni}\hrule}\vspace*{-1ex}\minitoc\vspace*{-1ex}{\color{bleuUni}\hrule}}

% package 'hyperref'
\ifthenelse{\boolean{twosidedoc}}
{
	\hypersetup{
		pdfborder  = {0 0 0},
		colorlinks = true,
		linkcolor  = black,
		citecolor  = black,
		urlcolor   = grayMarronUni}
	\urlstyle{same}
}
{
	\hypersetup{
		pdfborder  = {0 0 0},
		colorlinks = true,
		linkcolor  = marronUni,
		citecolor  = bleuUni,
		urlcolor   = marronUni}
	\urlstyle{same}
}

% separate multi-citations by a comma instead of a semicolon in 'biblatex'
\renewcommand*{\multicitedelim}{\addcomma\addspace}
% configure 'biblatex' for the multibib
\DeclareFieldFormat{labelnumberwidth}{\mkbibbrackets{#1}}
\renewbibmacro*{cite}{%
	\printtext[bibhyperref]{%
		\printfield{labelprefix}%
		\ifkeyword{Cassagne}
		{\printfield{labelnumber}}
		{\printfield{labelalpha}%
			\printfield{extraalpha}}}}
% configure 'biblatex' for the multibib
\defbibenvironment{bibliographyNUM}
{\list
	{\printtext[labelnumberwidth]{%
			\printfield{labelprefix}%
			\printfield{labelnumber}}}
	{\setlength{\labelwidth}{\labelnumberwidth}%
		\setlength{\leftmargin}{\labelwidth}%
		\setlength{\labelsep}{\biblabelsep}%
		\addtolength{\leftmargin}{\labelsep}%
		\setlength{\itemsep}{\bibitemsep}%
		\setlength{\parsep}{\bibparsep}}%
	\renewcommand*{\makelabel}[1]{\hss##1}}
{\endlist}
{\item}
% % configure 'biblatex' back references text
% \DefineBibliographyStrings{english}{%
%     backrefpage  = {see p.}, % for single page number
%     backrefpages = {see pp.} % for multiple page numbers
% }

% reduce the space between the lstlisting and its caption
\AtEndEnvironment{listing}{\vspace{-10pt}}

% add vertical spaces between chapters in the list of listings
\let\Chapter\chapter
\def\chapter{\addtocontents{lol}{\protect\addvspace{10pt}}\Chapter}

% combine the list of algorithms with the list of listings
% \makeatletter
% \AtBeginDocument{%
%   \let\c@algocf\c@lstlisting
% }
% \renewcommand{\algocf@list}{lol}%
% \renewcommand*\l@algocf{\@dottedtocline{1}{1.5em}{2.3em}}
% \makeatother

% don't externalize todo
\makeatletter
\renewcommand{\todo}[2][]{\tikzexternaldisable\@todo[#1]{#2}\tikzexternalenable}
\makeatother

% Math operators --------------------------------------------------------------
\DeclareMathOperator{\card}{card}
\DeclareMathOperator{\diag}{diag}
\DeclareMathOperator{\prob}{P}
\DeclareMathOperator{\hardDec}{h_d}
\DeclareMathOperator*{\maxstar}{max*}
\DeclareMathOperator*{\argmax}{arg\,max}
\DeclareMathOperator*{\decide}{decide}
\DeclareMathOperator*{\computeGamma}{computeGamma}
\DeclareMathOperator*{\computeAlpha}{computeAlpha}
\DeclareMathOperator*{\computeBeta}{computeBeta}
\DeclareMathOperator*{\computeExtrinsic}{computeExtrinsic}
\DeclareMathOperator*{\initAlpha}{initAlpha}
\DeclareMathOperator*{\initBeta}{initBeta}
\DeclareMathOperator*{\sizeof}{sizeof}
\DeclareMathOperator*{\argmin}{arg\,min}
\DeclareMathOperator*{\sign}{sign}
\DeclareMathOperator*{\SCLDecode}{SCLDecode}
\DeclareMathOperator*{\updatePaths}{updatePaths}
\DeclareMathOperator*{\selectBestPath}{selectBestPath}

% Commands --------------------------------------------------------------------
\newcommand{\R}{\textsuperscript{\textregistered}\xspace}
\newcommand{\TM}{\textsuperscript{\texttrademark}\xspace}
\newcommand{\cmark}{\textcolor{btfGreen}{\ding{51}}}
\newcommand{\xmark}{\textcolor{btfRed}{\ding{55}}}
\newcommand{\Arikan}{Ar{\i}kan\xspace}
\newcommand{\AFFECT}{AFF3CT\xspace}
\newcommand{\MIPP}{MIPP\xspace}
\newcommand{\longMIPP}{\textsc{MyIntrinsics++}\xspace}
\newcommand{\TSIMD}{{T-SIMD}\xspace}
\newcommand{\xsimd}{{xsimd}\xspace}
\newcommand{\simdpp}{{simdpp}\xspace}
\newcommand{\Vc}{{Vc}\xspace}
\newcommand{\VCL}{{VCL}\xspace}
\newcommand{\BoostSIMD}{{Boost.SIMD}\xspace}
\newcommand{\bSIMD}{{bSIMD}\xspace}
\newcommand{\C}{\texttt{C}\xspace}
\newcommand{\Cxx}{\texttt{C++}\xspace}
\newcommand{\Cxy}[1]{\texttt{C++{#1}}\xspace}
\newcommand{\CppUnit}{\texttt{CppUnit}\xspace}
\newcommand{\Cline}[2]{
	\noalign{\global\setlength{\Oldarrayrulewidth}{\arrayrulewidth}}
	\noalign{\global\setlength{\arrayrulewidth}{#1}}\cline{#2}
	\noalign{\global\setlength{\arrayrulewidth}{\Oldarrayrulewidth}}}
\renewcommand\listoflistingscaption{List of Algorithms and Source Codes}
\def\bl{big.\textsc{LITTLE}\xspace}
\def\bigARM{big\xspace}
\def\little{\textsc{LITTLE}\xspace}
\def\odr{\textsc{Odroid}\xspace}
\def\odrx{\textsc{Odroid-XU+E}\xspace}
\def\juno{\textsc{Juno}\xspace}
\def\mus{\si{\micro\second}}

% ---- Abomination code to make first letter bold
\makeatletter
\newcommand*{\BoldAbbrv}[1]{%
	\qrr@BoldAbbrv#1 \relax
}
\newcommand*{\eBoldAbbrv}[1]{%
	\edef\qrr@BoldAbbrv@Arg{#1 }%
	\expandafter\qrr@BoldAbbrv\qrr@BoldAbbrv@Arg\relax
}
\def\qrr@BoldAbbrv#1 #2\relax{%
	\textbf#1\relax
	\ifx\relax#2\else
		\space\qrr@BoldAbbrv#2\relax
	\fi
}
\makeatother