%%%%%%%%%%%%%%%%%%%%%%%%%%%%%%%%%%%%%%%%%%%%%%
%
%		Thesis Settings
%		Custom settings
%
%		2011
%
%%%%%%%%%%%%%%%%%%%%%%%%%%%%%%%%%%%%%%%%%%%%%%

%
%   Use this file for your own custom packages, command-definitions, etc...
%


% the following lines are for creating a simplified TO-DO box. However since boites is not per default installed with all latex-distributions, we have removed this example again
% if you want to use it and do not have "boites" installed, you can get it from here: http://www.ctan.org/tex-archive/macros/latex/contrib/boites
%
%\usepackage{boites,boites_exemples}
%\newcommand{\todolist}[1]{\begin{boiteepaisseavecuntitre}{TO DO in this chapter} #1 \end{boiteepaisseavecuntitre}}  % creates a little box
% %\newcommand{\todolist}[1]{}  % to be used when to do is not to be printed

\setcounter{secnumdepth}{4}

\usepackage[british]{minitoc} %rappel du plan du chapitre en cours
\renewcommand{\mtctitle}{}
\tightmtctrue
\nomtcrule
\renewcommand{\mtcSfont}{\small}

\newcommand{\minitocTITI}{\vspace*{1em}{\color{bleuUni}\hrule}\vspace*{-1ex}\minitoc\vspace*{-1ex}{\color{bleuUni}\hrule}%
}


%\usepackage{geometry} %pour la page de garde

\usepackage{tikz} %de même
\usetikzlibrary{shapes,arrows}
\usetikzlibrary{calc}

\usepackage{array}
\usepackage{longtable} %pour la liste des notations

\usepackage[xindy]{glossaries}
\usepackage[toc,page]{appendix}
%\makeglossaries
%\makenoidxglossaries
% \input{tail/glossaireEntries}

\usepackage{stmaryrd} % ll and rrbrackets
\usepackage{circuitikz}			% création d'images et de circuits électriques
\usepackage{pgfplots}				% package pour les graphiques
\usepackage{tikz-timing}			% package pour les chronogrammes
\usepackage{trig}				% fourni les fonctions trigonométriques

\usepackage{lettrine}
\pgfplotsset{compat=newest}

\usetikzlibrary{circuits, matrix, positioning}
% porte logique a plusieurs entrées
\usetikzlibrary{circuits.logic.CDH}
%\usetikzlibrary{circuits.ee.IEC}
\usepgfplotslibrary{groupplots}
\usepgfplotslibrary{colorbrewer}

\usepackage{pgffor}


\definecolor{Paired-2}{RGB}{166,206,227}
\definecolor{Paired-1}{RGB}{31,120,180}
\definecolor{Paired-4}{RGB}{178,223,138}
\definecolor{Paired-3}{RGB}{51,160,44}
\definecolor{Paired-6}{RGB}{251,154,153}
\definecolor{Paired-5}{RGB}{227,26,28}
\definecolor{Paired-8}{RGB}{253,191,111}
\definecolor{Paired-7}{RGB}{255,127,0}
\definecolor{Paired-10}{RGB}{202,178,214}
\definecolor{Paired-9}{RGB}{106,61,154}
\definecolor{Paired-12}{RGB}{255,255,153}
\definecolor{Paired-11}{RGB}{177,89,40}
\definecolor{Accent-1}{RGB}{127,201,127}
\definecolor{Accent-2}{RGB}{190,174,212}
\definecolor{Accent-3}{RGB}{253,192,134}
\definecolor{Accent-4}{RGB}{255,255,153}
\definecolor{Accent-5}{RGB}{56,108,176}
\definecolor{Accent-6}{RGB}{240,2,127}
\definecolor{Accent-7}{RGB}{191,91,23}
\definecolor{Accent-8}{RGB}{102,102,102}
\definecolor{Spectral-1}{RGB}{158,1,66}
\definecolor{Spectral-2}{RGB}{213,62,79}
\definecolor{Spectral-3}{RGB}{244,109,67}
\definecolor{Spectral-4}{RGB}{253,174,97}
\definecolor{Spectral-5}{RGB}{254,224,139}
\definecolor{Spectral-6}{RGB}{255,255,191}
\definecolor{Spectral-7}{RGB}{230,245,152}
\definecolor{Spectral-8}{RGB}{171,221,164}
\definecolor{Spectral-9}{RGB}{102,194,165}
\definecolor{Spectral-10}{RGB}{50,136,189}
\definecolor{Spectral-11}{RGB}{94,79,162}
\definecolor{Set1-1}{RGB}{228,26,28}
\definecolor{Set1-2}{RGB}{55,126,184}
\definecolor{Set1-3}{RGB}{77,175,74}
\definecolor{Set1-4}{RGB}{152,78,163}
\definecolor{Set1-5}{RGB}{255,127,0}
\definecolor{Set1-6}{RGB}{255,255,51}
\definecolor{Set1-7}{RGB}{166,86,40}
\definecolor{Set1-8}{RGB}{247,129,191}
\definecolor{Set1-9}{RGB}{153,153,153}
\definecolor{Set2-1}{RGB}{102,194,165}
\definecolor{Set2-2}{RGB}{252,141,98}
\definecolor{Set2-3}{RGB}{141,160,203}
\definecolor{Set2-4}{RGB}{231,138,195}
\definecolor{Set2-5}{RGB}{166,216,84}
\definecolor{Set2-6}{RGB}{255,217,47}
\definecolor{Set2-7}{RGB}{229,196,148}
\definecolor{Set2-8}{RGB}{179,179,179}
\definecolor{Dark2-1}{RGB}{27,158,119}
\definecolor{Dark2-2}{RGB}{217,95,2}
\definecolor{Dark2-3}{RGB}{117,112,179}
\definecolor{Dark2-4}{RGB}{231,41,138}
\definecolor{Dark2-5}{RGB}{102,166,30}
\definecolor{Dark2-6}{RGB}{230,171,2}
\definecolor{Dark2-7}{RGB}{166,118,29}
\definecolor{Dark2-8}{RGB}{102,102,102}
\definecolor{Reds-1}{RGB}{255,245,240}
\definecolor{Reds-2}{RGB}{254,224,210}
\definecolor{Reds-3}{RGB}{252,187,161}
\definecolor{Reds-4}{RGB}{252,146,114}
\definecolor{Reds-5}{RGB}{251,106,74}
\definecolor{Reds-6}{RGB}{239,59,44}
\definecolor{Reds-7}{RGB}{203,24,29}
\definecolor{Reds-8}{RGB}{165,15,21}
\definecolor{Reds-9}{RGB}{103,0,13}
\definecolor{Greens-1}{RGB}{247,252,245}
\definecolor{Greens-2}{RGB}{229,245,224}
\definecolor{Greens-3}{RGB}{199,233,192}
\definecolor{Greens-4}{RGB}{161,217,155}
\definecolor{Greens-5}{RGB}{116,196,118}
\definecolor{Greens-6}{RGB}{65,171,93}
\definecolor{Greens-7}{RGB}{35,139,69}
\definecolor{Greens-8}{RGB}{0,109,44}
\definecolor{Greens-9}{RGB}{0,68,27}
\definecolor{Blues-1}{RGB}{247,251,255}
\definecolor{Blues-2}{RGB}{222,235,247}
\definecolor{Blues-3}{RGB}{198,219,239}
\definecolor{Blues-4}{RGB}{158,202,225}
\definecolor{Blues-5}{RGB}{107,174,214}
\definecolor{Blues-6}{RGB}{66,146,198}
\definecolor{Blues-7}{RGB}{33,113,181}
\definecolor{Blues-8}{RGB}{8,81,156}
\definecolor{Blues-9}{RGB}{8,48,107}

\definecolor{falseframe}{RGB}{254,178,76}
\definecolor{correctframe}{RGB}{44,162,95}
\definecolor{falsebit}{RGB}{227,74,51}
\definecolor{correctbit}{RGB}{67,162,202}

\DeclareMathOperator{\card}{card}
\DeclareMathOperator*{\maxstar}{max*}
\DeclareMathOperator*{\argmax}{arg\,max}
\DeclareMathOperator*{\decide}{decide}

%\let\emph\relax % there's no \RedeclareTextFontCommand
%\DeclareTextFontCommand{\emph}{\bfseries\em}
\usetikzlibrary{patterns}

\usepackage{version}
\usepackage[export]{adjustbox}
\reversemarginpar

\newcommand{\e}{\mathrm{e}}
\usepackage{multirow}

\usepackage{arydshln}

\makeatletter
\def\adl@drawiv#1#2#3{%
        \hskip.5\tabcolsep
        \xleaders#3{#2.5\@tempdimb #1{1}#2.5\@tempdimb}%
                #2\z@ plus1fil minus1fil\relax
        \hskip.5\tabcolsep}
\newcommand{\cdashlinelr}[1]{%
  \noalign{\vskip\aboverulesep
           \global\let\@dashdrawstore\adl@draw
           \global\let\adl@draw\adl@drawiv}
  \cdashline{#1}
  \noalign{\global\let\adl@draw\@dashdrawstore
           \vskip\belowrulesep}}
\makeatother

\usepackage{scalerel}

\def\thumbsup{\includegraphics[width=15pt]{head/down.png}}
\def\thumbsdown{\includegraphics[width=15pt]{head/up.png}}
\def\thumbsmid{\includegraphics[width=15pt]{head/mid.png}}

\usepackage[french,onelanguage, ruled, linesnumbered, vlined]{algorithm2e}
\SetAlFnt{\small}
\usepackage{multibib}
\newcites{mine}{Publications}

\DecimalMathComma

\newcommand\ddfrac[2]{\frac{\displaystyle #1}{\displaystyle #2}}

\makeatletter
\AtBeginDocument{\let\book@l@chapter\l@chapter}
\newcommand{\demotechaptersintoc}{%
  \addtocontents{toc}{\let\protect\l@chapter\protect\l@section}%
}
\newcommand{\promotechaptersintoc}{%
  \addtocontents{toc}{\let\protect\l@chapter\protect\book@l@chapter}%
}
