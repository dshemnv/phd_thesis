%!TEX root = ../my_thesis.tex

\cleardoublepage
\addcontentsline{toc}{chapter}{Abstracts}
\chapter*{Résumé}
\vskip1em

Une radio logicielle en anglais \emph{Software-Defined Radio} ou SDR est un
système de communications numériques reconfigurable utilisant des techniques de
traitement numérique du signal sur des architectures numériques programmables.
Avec l'émergence de nouveaux standards de communications complexes et la
puissance de calcul grandissante des processeurs généralistes, il devient
intéressant d'échanger l'efficacité énergétique des architectures dédiées par la
souplesse et la facilité d'implémentation sur processeurs généralistes.

Même lorsque l'implémentation d'un traitement numérique est finalement faite sur
une puce dédiée, une version logicielle de ce traitement s'avère nécessaire en
amont pour s'assurer des bonnes propriétés de la fonctionnalité. Cela est
généralement réalisé via la simulation. Le simulations sont cependant souvent
coûteuses en temps de calcul. Il n'est pas rare de devoir attendre plusieurs
jours voire plusieurs semaines pour évaluer les performances du modèle
fonctionnel d'un système.

Dans ce contexte, cette thèse propose d'étudier les algorithmes les plus
coûteux en temps de calcul dans les chaînes de communication numériques
actuelles. Ces algorithmes sont le plus souvent présents dans des décodeurs de
codes correcteurs d'erreurs au niveau récepteur. Le rôle du codage canal est
d’accroître la robustesse vis à vis des erreurs qui peuvent apparaître lorsque
l'information transite au travers d'un canal de transmission. Trois grandes
familles codes correcteurs d'erreurs sont étudiées dans nos travaux, à savoir
les codes LDPC, les codes polaires et les turbo codes. Ces trois familles de
code sont présentes dans la plupart des standards de communication actuels comme
le Wi-Fi, l’Ethernet, les réseaux mobiles 3G, 4G et 5G, la télévision numérique,
etc. Les décodeurs qui en découlent proposent le meilleur compromis entre la
résistance aux erreurs et la vitesse de décodage. Chacune de ces familles repose
sur des algorithmes de décodage spécifiques. Un des enjeux principal de cette
thèse est de proposer des implémentations logicielles optimisées pour chacune
des trois familles. Des réponses sont apportées de façon spécifique et des
stratégies d'optimisation plus générales sont aussi discutées. L'idée est
d'abstraire des stratégies d'optimisation possibles en étudiant un sous-ensemble
représentatif de décodeurs.

Enfin, la dernière partie de cette thèse propose la mise en œuvre d'un système
de communications numériques complet à l'aide de la radio logicielle. En
s’appuyant sur les implémentations rapides de décodeurs proposées précédemment,
un émetteur et un récepteur compatibles avec le standard DVB-S2 sont
implémentés. Ce standard est typiquement utilisé pour la diffusion de contenu
multimédia par satellite. À cette occasion, un langage dédié à la radio
logicielle a été développé pour tirer parti de l'architecture parallèle des
processeurs généralistes actuels. Le système atteint des débits suffisants pour
être déployé en condition opérationnelle.

Les différentes contributions des travaux de thèse ont été faites dans une
dynamique d'ouverture, de partage et de réutilisabilité. Il en résulte une
bibliothèque à code source ouvert nommée AFF3CT pour \emph{A Fast Forward Error
Correction Toolbox}. Ainsi, tous les résultats proposés dans cette thèse peuvent
aisément être reproduits et étendus. Cette philosophie est détaillée dans un
chapitre spécifique du manuscrit de thèse.

\vskip0.5cm
\emph{Mots clefs :} Radio logicielle, Simulation fonctionnelle, Codes
                    correcteurs d'erreurs, Implémentation logicielle,
                    Optimisation, Parallélisation, Code source ouvert
