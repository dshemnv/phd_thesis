%!TEX root = ../my_thesis.tex

\cleardoublepage
\phantomsection
\chapter*{Résumé étendu}
\markboth{Extended Abstract in French}{Résumé étendu (\emph{Extended Abstract in French})}
\addcontentsline{toc}{chapter}{Extended Abstract in French}
\vskip1em

\newcommand{\vskipSectionResume}{\vskip0.5em}
\newcommand{\vskipSSSectionResume}{\vskip0.25em}

\section*{Chapitre~\ref{chap:ctx} - Contexte et objectifs}
\vskipSectionResume

\subsubsection*{Organisation}
\vskipSSSectionResume

Ce chapitre présente le contexte des systèmes de communication numérique. Il a
pour but de définir les notions qui seront réutilisées dans le manuscrit et de
donner une vue globale. Il définit aussi les principaux objectifs de cette
thèse.

La première partie présente le principe des systèmes de communication numérique
avec ses différentes composantes : l'émetteur, le canal et le récepteur. Les
métriques les plus couramment utilisées dans les communications numériques sont
présentées avec notamment la définition du taux d'erreur binaire (\emph{Bit
Error Rate}, BER) et du taux d'erreur trame (\emph{Frame Error Rate}, FER). La
deuxième partie détaille le modèle de canal par ajout de bruit blanc gaussien
(\emph{Additive White Gaussian Noise}, AWGN) et la modulation numérique binaire
par changement de phase (\emph{Binary Phase-Shift Keying}, BPSK) utilisées tout
au long du manuscrit. Une caractérisation du rapport signal sur bruit
(\emph{Signal-to-Noise Ratio}, SNR) est donnée ainsi que la notion de
probabilité à la sortie du canal. La troisième partie présente les familles de
code correcteur d'erreurs considérées dans ce manuscrit : à savoir les codes
LDPC, les codes polaires et les turbo-codes. Les traitements de codage (situé
dans l’émetteur) et de décodage (situé dans le récepteur) correspondants sont
détaillés pour chaque famille de code. Ces familles de code sont utilisées dans
la plupart des standards actuels et viennent avec une complexité calculatoire
élevée. Par conséquent, ce sont de bons candidats d'étude. Dans la dernière
partie, deux contextes applicatifs majeurs sont détaillés pour les familles de
code considérées. La simulation fonctionnelle permet la conception et la
validation d'un schéma de codage. La radio logicielle (\emph{Software-Defined
Radio}, SDR) est un système de communication radio où tous les composants sont
implémentés en logiciel (par opposition à des implémentations matérielles plus
couramment utilisées).

\subsubsection*{Objectifs de la thèse}
\vskipSSSectionResume

À l'aube de la cinquième génération des standards pour la téléphonie mobile
(5G), le défi consiste maintenant à concevoir des systèmes de communication
capables de transmettre d'énormes quantités de données en peu de temps, à un
faible coût énergétique et dans des environnements très variés. Les chercheurs
s'efforcent d'affiner encore les schémas de codage existants, afin d'obtenir de
faibles taux d'erreur résiduels grâce à des processus de décodage rapides,
souples et les moins complexes possible.

\paragraph{Simulation fonctionnelle}

La validation d'un système de codage nécessite d'estimer son taux d'erreur. En
général, il n'existe pas de modèle mathématique simple pour décrire ces
performances. La seule solution pratique consiste à effectuer une simulation
Monte Carlo de l'ensemble de la chaîne. Cela signifie que certaines données sont
générées, encodées, modulées, bruitées, décodées de manière aléatoire, et que
les performances sont ensuite estimées en mesurant le taux d'erreur binaire
(BER) et le taux d'erreur trame (FER) à la fin de la chaîne de communication
(après avis du décodeur). Ce processus a l'avantage d'être universel mais il
entraîne également trois problèmes principaux :

\begin{enumerate}
  \item \textbf{Temps de simulation :}
    $\sim 100$ trames erronées doivent être simulées pour estimer avec précision
    les BER/FER. Ainsi, la mesure d'un FER de $10^{-7}$ nécessite la simulation
    de la transmission de $\sim100\times 10^7=10^9$ trames. En supposant une
    trame de 1000~bits, le simulateur doit alors calculer la transmission de
    $10^{12}$~bits. En gardant à l'esprit que la complexité de calcul de
    l'algorithme de décodage peut être importante, plusieurs semaines ou mois
    peuvent être nécessaires pour estimer avec précision les BER/FER d'un schéma
    de codage (en particulier si le taux d'erreur est faible).

  \item \textbf{Hétérogénéité algorithmique :}
    un grand nombre de codes correcteurs d'erreur ont été conçus au fil des
    années. Pour chaque famille de code, plusieurs configurations de décodage
    sont possibles. S'il est simple de décrire un schéma de codage unique, il
    est  plus difficile d'avoir une description logicielle unifiée qui prenne en
    charge tous les schémas de codage et les algorithmes associés. Cette
    difficulté provient de l'hétérogénéité des structures de données nécessaires
    pour décrire les différents schémas de codage canal : les turbo-codes sont
    basés sur des treillis, les codes LDPC sont bien définis sur des graphes
    bipartis et les codes polaires sont décodés efficacement à l'aide d'arbres
    binaires.

  \item \textbf{Reproductibilité :}
    il est généralement fastidieux de reproduire des résultats issus de la
    littérature. Cela peut s'expliquer par la grande quantité de paramètres
    empiriques nécessaires pour définir un système de communication, et par le
    fait que tous ne sont pas toujours rapportés dans les publications. En
    outre, le code source des simulateurs est rarement accessible au public.
    Par conséquent, beaucoup de temps est passé à ``réinventer la roue'' juste
    pour pouvoir comparer de nouveaux résultats avec l'état de l'art.
\end{enumerate}

\paragraph{Radio logicielle}

Le paradigme de la radio logicielle (\emph{Software-Defined Radio}, SDR) est
désormais considéré pour des systèmes de communication réels. Pour répondre
aux contraintes posées par les systèmes temps réel, voici les principaux défis à
relever :

\begin{enumerate}
  \item \textbf{Haut débit :}
    les nouvelles applications comme le \emph{streaming} vidéo, peuvent être
    très gourmandes en données. En conséquence, les tâches de calcul intensif de
    l'émetteur et du récepteur doivent être bien optimisées pour atteindre des
    niveaux de performance comparables aux implémentation matérielles.

  \item \textbf{Faible latence :}
    atteindre un débit élevé n'est pas toujours la contrainte majeure, par
    exemple, dans les applications d'audio-conférence, il est inconfortable de
    percevoir un retard lorsque les gens parlent.

  \item \textbf{Flexibilité :}
    les implémentations logicielles doivent pouvoir s'adapter à diverses
    configurations. Par exemple, lorsque le rapport signal sur bruit (SNR)
    change, le décodeur doit être capable se s'adapter ``à la volée'' à de
    nouveaux taux de codage $R$.

  \item \textbf{Portabilité :}
    les solutions proposées peuvent être déployées sur des serveurs haut de
    gamme ainsi que dans des systèmes embarqués à faible consommation d'énergie.
    De plus, de nombreux systèmes d'exploitation coexistent, et il est important
    de pouvoir supporter les plus communs comme Windows, macOS et Linux.
\end{enumerate}

\section*{Chapitre~\ref{chap:opt} - Stratégies d'optimisation}
\vskipSectionResume

\subsubsection*{Organisation}
\vskipSSSectionResume

Ce chapitre se concentre sur les stratégies d'optimisation dédiées aux
algorithmes de communication numérique. Nos contributions sont divisées en deux
parties : 1) les stratégies génériques et 2) les optimisations spécifiques.
La première partie décrit les stratégies génériques que nous avons proposées
pour optimiser les algorithmes présents dans les récepteurs de systèmes de
communication numérique. Il s'avère que la vectorisation est une des clefs pour
implémenter des solutions logicielles efficaces. Une bibliothèque dédiée ainsi
que des niveaux de parallélisme génériques sont proposés. La seconde partie est
consacrée à l'implémentation logicielle efficace d'un sous-ensemble
représentatif de décodeurs pour les trois grandes familles abordées plus tôt, à
savoir : les codes LDPC, les codes polaires et les turbo-codes.

\subsubsection*{Principaux résultats}
\vskipSSSectionResume

En premier lieu, des stratégies génériques pour l'implémentation efficace
d'algorithmes sur processeurs généralistes (CPUs) sont présentées. Une
contribution principale de ce chapitre est la proposition de \MIPP : une
bibliothèque qui encapsule les instructions vectorielles. L'idée est d'abstraire
les types de données et les multiples jeux d’instructions vectoriels existants
afin de proposer des implémentations logicielles ``universelles'' et efficaces
des algorithmes présents dans les récepteurs de systèmes de communication
numérique. Nous montrons que \MIPP n'introduit presque pas de surcoût par
rapport aux fonctions intrinsèques (ou du code assembleur). \MIPP fonctionne
aussi bien sur des représentations en virgule flottante et que sur des
représentations en virgule fixe. Pour les algorithmes présents dans les
récepteurs de systèmes de communication numérique, les représentations en
virgule fixe sont très intéressantes car elles permettent de traiter un plus
grand nombre d'éléments dans les registres vectoriels, avec un impact modéré sur
les performances de décodage. Pour résumer, \textbf{\MIPP améliore la
flexibilité et la portabilité du code source tout en conservant le même niveau
de performance.} Notez que la bibliothèque \MIPP a été valorisée suite à une
publication dans une conférence scientifique~\cite{Cassagne2018}.

Dans une deuxième partie, deux grandes stratégies de vectorisation sont
explicitement définies et présentées. La stratégie intra-trame fonctionne
sur une seule trame en s'appuyant sur le parallélisme inhérent à l'algorithme,
tandis que la stratégie inter-trames fonctionne sur plusieurs trames en
même temps. La stratégie intra-trame peut à la fois augmenter le débit et
diminuer la latence. Au contraire, la stratégie inter-trames n'améliore pas la
latence, mais elle s'accompagne d'une efficacité vectorielle potentiellement
plus élevée et peut conduire à des débits très élevés. Ces deux stratégies
peuvent être appliquées à tous les blocs de traitement des chaînes de
communication numérique. \textbf{Les stratégies intra-trame et inter-trames
constituent donc un point clé pour résoudre le problème de l'hétérogénéité
algorithmique.}

Les dernières parties du chapitre se concentrent sur la conception
d'implémentations logicielles efficaces des algorithmes de décodage présentés
dans le chapitre précédent. Les décodeurs LDPC, les décodeurs polaires et le
turbo-décodeur sont compatibles avec la stratégie inter-trames, tandis que les
décodeurs polaires sont aussi compatibles avec la stratégie intra-trame. En
fonction des familles de codes, nous nous concentrons sur différentes
contraintes. \textbf{Les décodeurs LDPC ont été mis en œuvre pour prendre en
charge de nombreuses variantes et donc pour maximiser la flexibilité} au prix de
débits plus faibles et de latences plus élevées par rapport à d'autres travaux.
Ce choix permet d'évaluer les performances de décodage de nombreuses
combinaisons algorithmiques. \textbf{Dans les décodeurs polaires, la flexibilité
ainsi que des optimisations agressives sont considérées et comparées. Ces
dernières permettent d'atteindre de très faibles latences.} \textbf{Enfin, le
turbo-décodeur se concentre sur l'obtention de débits les plus élevés possibles}
et certaines spécialisations sont faites pour la norme LTE. Il est à noter que
la plupart des implémentations logicielles proposées ont fait l'objet de
publications dans des conférences et des revues
scientifiques~\cite{Ghaffari2019,Leonardon2019,Cassagne2015c,Cassagne2016b,
Cassagne2016a}.

\section*{Chapitre~\ref{chap:aff3ct} - AFF3CT : une boîte à outils pour le codage canal}
\vskipSectionResume

\subsubsection*{Organisation}
\vskipSSSectionResume

Ce chapitre est consacré à la présentation de notre boîte à outils open-source
nommée \AFFECT. La première partie décrit les principaux prérequis en fonction
de quatre objectifs : l'implémentation d'un logiciel hautement performant, la
prise en charge de l'hétérogénéité algorithmique, la portabilité et la
reproductibilité. Dans la deuxième partie, \AFFECT est comparé aux autres
bibliothèques logicielles de codage canal \verb|C|/\Cxx existantes. La troisième
partie présente \AFFECT comme une bibliothèque dédiée aux algorithmes de
communication numérique. L'architecture et les fonctionnalités du logiciel sont
décrites. Ensuite, des exemples d'utilisation de la bibliothèque sont donnés en
\Cxx et en MATLAB\R. La quatrième partie se concentre sur le simulateur \AFFECT
qui est livré dans la boîte à outils. Un aperçu des explorations possibles est
donné et notre comparateur de BER/FER est présenté. À la fin, la stratégie de
test d'\AFFECT est expliquée. Une dernière partie est consacrée à l'impact
d'\AFFECT dans les contextes industriels et universitaires. Une revue des
publications scientifiques qui ont utilisé \AFFECT est donnée.

\subsubsection*{Principaux résultats}
\vskipSSSectionResume

Tout d'abord, l'accent est mis sur \textbf{la bibliothèque \AFFECT qui vient
avec une architecture logicielle qui permet l'hétérogénéité algorithmique}. De
nombreuses familles de codes correcteurs d'erreur sont supportées comme les
codes LDPC, les codes polaires, les turbo-codes, les turbo-codes produit, les
codes convolutifs, les codes BCH, les codes RS, etc. \textbf{À notre
connaissance, \AFFECT est la bibliothèque qui offre le support le plus complet
pour les algorithmes de codage canal.} \AFFECT est également livré avec des
modèles de canaux multiples (AWGN, Rayleigh, BEC, BSC, etc.) et des schémas de
modulation (PSK, QAM, PAM, OOK, CPM, SCMA, etc.). Toutes ces implémentations
d'algorithmes efficaces peuvent être utilisées à partir d'interfaces. Des
exemples d'utilisation de la bibliothèque sont donnés en \Cxx natif ou en
utilisant l'encapsulation MATLAB\R. La boîte à outils \AFFECT a fait l'objet de
publications dans une conférence et une revue scientifiques~\cite{Cassagne2017a,
Cassagne2019a}.

\textbf{\AFFECT est également fourni avec un simulateur fonctionnel de BER/FER.}
Toutes les caractéristiques précédemment énumérées peuvent être simulées sur
différents paramètres. \textbf{Sa capacité à explorer une grande variété de
paramètres est démontrée.} De nombreux paramètres peuvent être modifiés comme le
nombre d'itérations de décodage, les approximations dans l'implémentation des
algorithmes, la quantification des données dans les décodeurs, etc. Certains de
ces paramètres sont présentés pour les décodeurs introduits dans les chapitres
précédents. Il est à noter que ce sujet a été valorisé par un article dans une
conférence nationale~\cite{Cassagne2017}.

\textbf{\AFFECT est conçu pour permettre la reproductibilité des résultats
scientifiques.} Un outil de comparaison des performances de décodage (BER/FER) a
été ajouté pour permettre une recherche facile dans une base de données
d'environ 500 références pré-simulées. Toutes ces références sont des résultats
simulés avec \AFFECT qui peuvent être reproduits. À cette fin, un pipeline de
tests a été mis en place. Chaque fois qu'il y a une modification du code source,
la base de données des références est rejouée pour éviter les régressions. Ces
tests sont également effectués sur plusieurs architectures (x86 et ARM\R) et
systèmes d'exploitation (Windows, macOS et Linux) afin de garantir que la
portabilité soit toujours conservée.

La dernière partie du chapitre traite de l'impact d'\AFFECT dans la communauté.
\textbf{Il est montré que de plus en plus d'utilisateurs adoptent la boîte à
outils \AFFECT aussi bien dans l'industrie que dans les milieux académiques.}
Les contextes applicatifs sont variés et vont de la validation des performances
de décodage à l'utilisation de sous-parties spécifiques de la bibliothèque. Les
contributions externes sont cependant encore rares.

\section*{Chapitre~\ref{chap:eval} - Évaluation et comparaison des performances}
\vskipSectionResume

\subsubsection*{Organisation}
\vskipSSSectionResume

Ce chapitre propose d'évaluer les différentes contributions exposées dans les
chapitres précédents. Les trois premières parties se concentrent sur les
implémentations efficaces des décodeurs LDPC, des décodeurs polaires et du
turbo-décodeur. Le débit, la latence et l'efficacité énergétique sont étudiés
et comparés avec d'autres travaux. La quatrième partie résume les
implémentations de décodeurs logiciels les plus efficaces que nous avons
trouvées dans la littérature. Cet état de l'art est décomposé en trois
catégories distinctes : une pour les décodeurs LDPC, une pour les décodeurs
polaires et une pour les turbo-décodeurs. Certaines métriques sont définies pour
faciliter la comparaison entre les différentes publications. La dernière partie
est consacrée à une analyse des performances du simulateur \AFFECT. Une chaîne
de communication numérique représentative est définie et évaluée en séquentiel
et en parallèle. Cette chaîne utilise un décodeur polaire rapide évalué plus tôt
dans le chapitre.

\subsubsection*{Principaux résultats}
\vskipSSSectionResume

Pour les décodeurs LDPC et les turbo-décodeurs, la stratégie inter-trames a été
appliquée et permet d'obtenir des débits comparables aux meilleurs travaux de la
littérature. Toutefois, les latences ne sont pas compétitives avec les
meilleures implémentations de type intra-trame que l'on trouve dans la
littérature. L'implémentation inter-trames proposée est alors davantage
orientées vers la simulation ou vers des applications en temps réel qui ne
nécessitent pas une faible latence comme le \emph{streaming} vidéo, par exemple.
Pour les décodeurs polaires, les stratégies inter-trames et intra-trame ont
toute le deux été implémentées. Il en résulte un \emph{framework} complet qui
peut s'adapter à de nombreux contextes applicatifs. \textbf{Les décodeurs
proposés sont parmi les plus rapides de la littérature. Ils peuvent également
être très flexibles avec les implémentations dynamiques ou spécialisés pour des
performances maximales avec la technique de génération de code source.} Pour
tous les décodeurs proposés (LDPC, polar et turbo), le niveau de généricité est
l'une de nos principales contributions. \textbf{Les implémentations sont
capables de s'adapter à différentes architectures de CPU ainsi que de supporter
de nombreuses variantes algorithmiques.} De plus, chacune des implémentations
présentées est capable de travailler à un niveau proche des performances de
décodage de référence. La plupart des résultats obtenus ont été publiés dans des
conférences et des revues scientifiques~\cite{Ghaffari2019,Leonardon2019,
Cassagne2015c,Cassagne2016b,Cassagne2016a}.

Les ``Temples de la renommée'' (\emph{Hall of Fames}, HoFs) des décodeurs
logiciels sont ensuite présentés. Ces HoFs représentent des états de l'art
complets par famille de code correcteur d'erreur. Les implémentations de
décodeur proposées dans la thèse sont comparées avec les autres travaux de la
littérature. Ces HoFs permettent de comparer les implémentations CPU et GPU.
Certaines mesures telles que le débit normalisé, le TNDC et la consommation
d'énergie sont définies. Les résultats montrent que ces dernières années, les
implémentations CPU sont plus efficaces que les implémentations GPU en termes de
débit, de latence et d'efficacité énergétique. L'un des principaux problèmes des
implémentations basées sur GPU est le temps de transfert nécessaire entre le
CPU et le GPU. Un autre problème majeur vient de l'architecture intrinsèque des
GPUs qui nécessite beaucoup de parallélisme pour être efficace. Il n'est pas
toujours possible de tirer parti de ce niveau élevé de parallélisme dans les
algorithmes de décodage canal. \textbf{Par conséquent, en général, les CPUs sont
plus adaptés pour des implémentations à faible latence que les GPUs.}

La dernière partie de ce chapitre est consacrée aux performances du simulateur
\AFFECT. Une chaîne de communication numérique entièrement vectorisée est
proposée pour l'évaluation. Les performances sur un seul cœur de calcul CPU sont
d'abord présentées. Il en résulte qu'\AFFECT fonctionne le plus rapidement sur
les derniers processeurs Intel\R Gold qui supportent le jeu d'instruction
vectoriel ``AVX-512''. Ensuite, les performances sur plusieurs cœurs de calcul
sont mises à l'épreuve et les processeurs AMD\R EPYC sont les plus performants :
le débit utile de la chaîne atteint 11 Gb/s. Même si les processeurs AMD\R EPYC
ne prennent en charge que les instructions de type ``AVX'', il semble que
l'architecture Zen 2 soit bien équilibrée entre la puissance de calcul et la
vitesse de la mémoire. Enfin, la capacité multi-nœuds du simulateur \AFFECT est
testée et une accélération linéaire est observée sur 32 nœuds. Le débit de
pointe en multi-nœuds est de 32 Gb/s. \textbf{Ces débits élevés permettent
l'exploration de nombreuses combinaisons à un niveau de taux d'erreur très
faible.} Une partie de ces résultats ont été publiés dans une revue
scientifique~\cite{Cassagne2019a}. \textbf{À l'heure actuelle et à notre
connaissance, \AFFECT est l'un des simulateurs de codes correcteurs d'erreurs
le plus rapide.}

\section*{Chapitre~\ref{chap:sdr} - Langage embarqué et dédié à la radio logicielle}
\vskipSectionResume

\subsubsection*{Organisation}
\vskipSSSectionResume

Ce chapitre présente un nouveau langage embarqué et dédié (\emph{embedded Domain
Specific Language}, eDSL) à la radio logicielle (SDR). La première partie
décrit les modèles et solutions existants. Elle motive également le besoin
d'un nouveau langage dédié à la radio logicielle. Dans une deuxième partie, une
description de l'eDSL proposé est donnée et détaillée en deux sous-parties. Dans
un premier temps, les composants élémentaires sont présentés, puis, dans un
second temps, les composants parallèles sont décrits. La troisième partie se
concentre sur l'implémentation des composants présentés précédemment. Entre
autres, la technique de duplication des séquences et l'implémentation du
pipeline sont discutées. Enfin, la dernière partie présente un cas concret
d'utilisation de l'eDSL sur une norme bien répandue dans les communications
numériques: la norme DVB-S2. Un émetteur-récepteur entièrement numérique a été
conçu en logiciel. La norme DVB-S2 est présentée d'un point de vue applicatif
(émetteur et récepteur) et est ensuite évaluée sur une cible CPU spécifique.

\subsubsection*{Principaux résultats}
\vskipSSSectionResume

Les principaux composants de l'eDSL ont été conçus pour répondre aux besoins de
la SDR en termes 1) d'expressivité avec des séquences, des tâches et des
boucles ; 2) de performance avec la technique de duplication de séquences et la
stratégie de pipeline. Nous avons évalué l'eDSL proposé dans un contexte
applicatif : l'implémentation logicielle de la norme DVB-S2. \textbf{Les
résultats démontrent l'efficacité de l'eDSL d'\AFFECT. En effet, la solution
proposée répond aux contraintes de temps réel des satellites (30 $\thicksim$
50 Mb/s).} Ceci est la conséquence de deux facteurs principaux : 1) les
optimisations au niveau des tâches, par exemple un décodeur LDPC rapide a été
utilisé (ce décodeur est détaillé dans les chapitres précédents) ; 2) l'eDSL a
un très faible surcoût à l'utilisation. Cela est notamment possible grâce à une
implémentation efficace de la technique du pipeline.
