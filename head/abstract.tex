%!TEX root = ../my_thesis.tex

\cleardoublepage
\chapter*{Abstract}
\vskip1em

% Since their introduction in the 90's, turbo codes are considered as one of the
% most powerful error-correcting code. Thanks to their excellent trade-off between
% computational complexity and decoding performance, they were chosen in many
% communication standards.

% One way to characterize error-correcting codes is the evolution of the bit error
% rate as a function of signal-to-noise ratio (SNR). The turbo code error rate
% performance is divided in two different regions: the waterfall region and the
% error floor region. In the waterfall region, a slight increase in SNR results in
% a significant drop in error rate. In the error floor region, the error rate
% performance is only slightly improved as the SNR grows. This error floor can
% prevent turbo codes from being used in applications with low error rates
% requirements. Therefore various constructions optimizations that lower the error
% floor of turbo codes has been proposed in recent years by scientific community.
% However, these approaches can not be considered for already standardized turbo
% codes.

% This thesis addresses the problem of lowering the error floor of turbo codes
% without allowing any modification of the digital communication chain at the
% transmitter side. For this purpose, the state-of-the-art post-processing
% decoding method for turbo codes is detailed. It appears that efficient solutions
% are expensive to implement due to the required multiplication of computational
% resources or can strongly impact the overall decoding latency.

% Firstly, two decoding algorithms based on the monitoring of decoder's internal
% metrics are proposed. The waterfall region is enhanced by the first algorithm.
% However, the second one marginally lowers the error floor. Then, the study shows
% that in the error floor region, frames decoded by the turbo decoder are really
% close to the word originally transmitted. This is demonstrated by a proposition
% of an analytical prediction of the distribution of the number of bits in errors
% per erroneous frame. This prediction rests on the distance spectrum of turbo
% codes. Since the appearance of error floor region is due to only few bits in
% errors, an identification metric is proposed. This lead to the proposal of an
% algorithm that can correct residual errors. This algorithm, called
% Flip-and-Check, rests on the generation of candidate words, followed by
% verification according to an error-detecting code. Thanks to this decoding
% algorithm, the error floor of turbo codes encountered in different standards
% (LTE, CCSDS, DVB-RCS and DVB-RCS2) is lowered by one order of magnitude. This
% performance improvement is obtained without considering an important
% computational complexity overhead.

% Finally, a hardware decoding architecture implementing the Flip-and-Check
% algorithm is presented. A preliminary study of the impact of the different
% parameters of this algorithm is carried out. It leads to the definition of
% optimal values for some of these parameters. Others has to be adapted according
% to the gains targeted in terms of decoding performance. The possible integration
% of this algorithm along with existing turbo decoders is demonstrated thanks to
% this hardware architecture. This therefore enables the lowering of the error
% floors of standardized turbo codes.

\vskip0.5cm
\emph{Keywords:} Software-Defined Radio, Simulation, Error Correcting Codes,
                 Software Implementation, Optimization, Parallelization,
                 Open Source Code
