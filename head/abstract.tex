%!TEX root = ../my_thesis.tex

\cleardoublepage
\chapter*{Abstract}
\vskip1em

Software-defined radio is a radio communication system where components that
have been traditionally implemented in hardware are instead implemented by means
of software. With the growing number of complex digital communication standards
and the general purpose processors increasing power, it becomes interesting to
trade the energy efficiency of the dedicated architectures by the flexibility
and the reduce time to market time on general purpose processors.

Even if at the end the implementation of signal processing is made on a
application-specific integrated circuit, the software version of this processing
is necessary to evaluate and verify the properties of the algorithm. This is
generally the role of the simulation: the digital communication system designers
first bench the signal processing algorithms by means of software simulations.
The latter are often expensive in terms of computational time. To evaluate the
global performance of a communication system, one can wait from few days to few
weeks.

In this context, this thesis proposes to study the most time consuming
algorithms in the today digital communication chains. These algorithms often are
the channel decoders located in the receivers. The role of the channel channel
coding is to improve the errors resilience of the system. Indeed, errors can
occur at the channel level during the transmission between the transmitter and
the receiver. Three main channel coding families are then presented: the LDPC
codes, the polar codes and the turbo codes. These three code families are used
in most of the current digital communication standards like the Wi-Fi, the
Ethernet, the 3G, 4G and 5G mobile networks, the digital television, etc. The
resulting decoders offer the best compromise between error resistance and
decoding speed known to date. Each of these families comes with different
decoding algorithms. One of the main challenge of this thesis is to propose
optimized software implementations for each of them. Specific efficient
implementations are proposed as well as more general optimization strategies.
The idea is to extract the generic optimization strategies from a representative
sub-set of decoders.

The last part of the thesis focuses on the implementation of a complete digital
communication system in software. Thanks to the efficient decoding
implementations proposed before, a full transceiver, compatible with the DVB-S2
standard, is implemented. This standard is typically used for broadcasting
multimedia contents via satellite. To this purpose, an embedded domain specific
language targeting the software-defined radio is introduced. The main objective
of this language is to take advantage of the parallel architecture of the
current general purpose processors. The results show that the system achieves
sufficient throughputs to be deployed in real-world conditions.

These contributions have been made in a dynamic of openness and sharing, it
results in an open source library named AFF3CT for A Fast Forward Error
Correction Toolbox. Thus, the results proposed in this thesis can easily be
reproduced. This philosophy is detailed in a chapter of the manuscript.

\vskip0.5cm
\emph{Keywords:} Software-Defined Radio, Simulation, Error Correcting Codes,
                 Software Implementation, Optimization, Parallelization,
                 Open Source Code
